\documentclass[oneside]{amsart}

\usepackage[all]{xy}
\usepackage{tikz-cd}
\usepackage[T1]{fontenc}
\usepackage{xstring}
\usepackage{xparse}
\usepackage{xr-hyper}
\usepackage{xcolor}
\definecolor{brightmaroon}{rgb}{0.76, 0.13, 0.28}
\usepackage[linktocpage=true,colorlinks=true,hyperindex,citecolor=blue,linkcolor=brightmaroon]{hyperref}
\usepackage[nameinlink]{cleveref}
\usepackage[left=1.25in,right=1.25in,top=0.75in,bottom=0.75in]{geometry}
%\usepackage[charter,greekfamily=didot]{mathdesign}
%\usepackage{Baskervaldx}
\usepackage{amssymb}
\usepackage{stmaryrd}
\usepackage{mathrsfs}
\usepackage{mathpazo}
\linespread{1.05}

\usepackage[nobottomtitles]{titlesec}
\usepackage{marginnote}
\usepackage{enumerate}
\usepackage{longtable}
\usepackage{aurical}
\usepackage{microtype}

\newtheoremstyle{ega-env-style}%
{}{}{\rmfamily}{}{\bfseries}{.}{ }{\thmnote{(#3)}}%

\newtheoremstyle{ega-thm-env-style}%
{}{}{\itshape}{}{\bfseries}{. --- }{ }{\thmname{#1}\thmnumber{ #2}\thmnote{ (#3)}}%

\newtheoremstyle{ega-defn-env-style}%
{}{}{\rmfamily}{}{\bfseries}{. --- }{ }{\thmname{#1}\thmnumber{ #2}\thmnote{ (#3)}}%

\theoremstyle{ega-env-style}
\newtheorem*{env}{---}

\theoremstyle{ega-thm-env-style}
\newtheorem{theorem}{Theorem}[subsection]
\newtheorem{proposition}{Proposition}[subsection]
\newtheorem{lemma}{Lemma}[subsection]
\newtheorem{corollary}{Corollary}[subsection]
\newtheorem{stheorem}{Theorem}[section]
\newtheorem{slemma}[stheorem]{Lemma}
\newtheorem{skey}[stheorem]{Key Formula}
\newtheorem{conjecture}{Conjecture}[subsection]

\theoremstyle{ega-defn-env-style}
\newtheorem*{definition}{Definition}
\newtheorem{example}{Example}[subsection]
\newtheorem*{examples}{Examples}
\newtheorem*{remark}{Remark}
\newtheorem*{remarks}{Remarks}
\newtheorem*{notation}{Notation}
\newtheorem*{exercise}{Exercise}
\newtheorem*{properties}{Properties}
\newtheorem*{consequences}{Consequences}
\newtheorem*{note}{Note}
\newtheorem*{fact}{Fact}

\makeatletter
\def\l@subsection{\@tocline{2}{0pt}{2.5pc}{2.2pc}{}}
\def
\section{\@startsection{section}{1}%
	\z@{.7\linespacing\@plus\linespacing}{.5\linespacing}%
{\normalfont\bfseries\Large\scshape\centering}}
\renewcommand{\@seccntformat}[1]{%
	\ifnum\pdfstrcmp{#1}{section}=0\textsection\fi%
\csname the#1\endcsname.~}
\makeatother

\def\mathcal{\mathscr}
%% Fonts

\def\sh{\mathcal}                   % sheaf font
\def\bb{\mathbf}                    % bold font
\def\cat{\mathsf}                   % category font

%% Font Letters

\def\CL{\mathcal{L}}
\def\OO{\mathcal{O}}
\def\CX{\mathcal{X}}

%% Cohomology

\def\CH{\mathrm{H}}                 % cohomology H
\def\CHH{\check{\HH}}               % Čech cohomology H
\def\RD{\mathrm{R}}                 % right derived R
\def\LD{\mathrm{L}}                 % left derived L
\def\dual#1{{#1}^\vee}              % dual
\def\Tor{\operatorname{Tor}}        % Tor
\def\Ext{\operatorname{Ext}}        % Ext
\def\HdR{\mathrm{H}_{\mathrm{dR}}}  % de Rham cohomology
\def\Zc{\underline{\mathbb{Z}}}     % constant sheaf with integer coeffs

%% Categories

\def\A{\cat{A}}                     % category A, usually abelian
\def\C{\cat{C}}                     % category C
\def\op{^\cat{op}}                  % opposite category
\def\Set{\cat{Sets}}                % category of sets
\def\Grp{\cat{Gps}}                 % category of groups
\def\Alg{\cat{Alg}}                 % category of algebras
\def\QCoh{\cat{QCoh}}               % category of quasicoherent sheaves
\def\supertilde{{\,\widetilde{\,}\,}}   % use \supertilde instead of ^\sim
\def\Hom{\operatorname{Hom}}        % morphisms
\def\End{\operatorname{End}}        % endomorphisms
\def\Aut{\operatorname{Aut}}        % automorphisms
\def\ker{\operatorname{ker}}        % kernel
\def\img{\operatorname{im}}         % image
\def\coker{\operatorname{coker}}    % cokernel
\def\pr{\operatorname{pr}}          % projection
\def\EssIm{\operatorname{EssIm}}    % essential image
\DeclareMathOperator*{\colim}{colim}   % colimit


%% Schemes

\def\Proj{\operatorname{Proj}}      % Proj
\def\Supp{\operatorname{Supp}}      % support
\def\Spec{\operatorname{Spec}}      % Spec
\def\Spf{\operatorname{Spf}}        % formal Spec
\def\Aff{\mathbb A}                 % affine space
\def\P{\mathbb{P}}                  % projective space
\def\Pic{\operatorname{Pic}}        % Picard group

%% Standard Operators

\def\codim{\operatorname{codim}}    % codimension
\def\id{\operatorname{id}}          % identity

%% Arrows
\renewcommand{\to}{\mathchoice{\longrightarrow}{\rightarrow}{\rightarrow}{\rightarrow}}
\newcommand{\from}{\mathchoice{\longleftarrow}{\leftarrow}{\leftarrow}{\leftarrow}}
\let\mapstoo\mapsto
\renewcommand{\mapsto}{\mathchoice{\longmapsto}{\mapstoo}{\mapstoo}{\mapstoo}}
\def\isoto{\simeq}                  % isomorphism
\def\simto{\xrightarrow{\sim}}      % isomorphism arrow
\def\surjto{\twoheadrightarrow}     % surjetion
\def\injto{\hookrightarrow}         % injection

%% Under/Over Accents

\newcommand{\wh}[1]{\widehat{#1}}   % hat
\newcommand{\wt}[1]{\widetilde{#1}}    % tilde
\def\ul{\underline}                % underline

%% Spaces

\def\Z{\mathbb Z}                  % integers
\def\Q{\mathbb Q}                  % rationals
\def\N{\mathbb{N}}                 % naturals

%% Groups

\def\GL{\bb{GL}}                   % general linear group
\def\SL{\bb{SL}}                   % special linear group
\def\det{\operatorname{det}}       % determinant

%% Sub/Superscripts

\def\et{^\text{\'et}}              % \'etale
\def\an{^{\text{an}}}              % analytic

%% Misc

\newcommand{\defn}[1]{\textbf{#1}}  % definition highlighting
\def\defeq{:=}                     % definition equation
\def\eps{\varepsilon}              % correct epsilon
\newcommand{\gen}[1]{\left\langle\!\left\langle #1 \right\rangle\!\right\rangle}



\def\shHom{\sh{H}\textup{\kern-2.2pt{\Fontauri\slshape om}}}   % sheaf Hom
\def\shProj{\sh{P}\textup{\kern-2.2pt{\Fontauri\slshape roj}}} % sheaf Proj
\def\shExt{\sh{E}\textup{\kern-2.2pt{\Fontauri\slshape xt}}}   % sheaf Ext
\def\shGr{\sh{G}\textup{\kern-2.2pt{\Fontauri\slshape r}}}     % sheaf Gr
\def\shDer{\sh{D}\,\textup{\kern-2.2pt{\Fontauri\slshape er}}} % sheaf Der
\def\shDiff{\sh{D}\,\textup{\kern-2.2pt{\Fontauri\slshape if{}f}}\,} % sheaf Diff
\def\shHomcont{\sh{H}\textup{\kern-2.2pt{\Fontauri\slshape om.\,cont}}}   % sheaf Hom.cont
\def\shAut{\sh{A}\textup{\kern-2.2pt{\Fontauri\slshape ut}}}   % sheaf Aut
\def\shSym{\sh{S}\textup{\kern-2.2pt{\Fontauri\slshape ym}}}   % sheaf Sym

\makeatletter
\newcommand{\cbigoplus}{\DOTSB\cbigoplus@\slimits@}
\newcommand{\cbigoplus@}{\mathop{\widehat{\bigoplus}}}
\makeatother


\def\i{\operatorname{inv}}
\def\Gm{\mathbb{G}_m}
\def\Ga{\mathbb{G}_a}
\def\Diag{\operatorname{Diag}}
\def\Rep{\operatorname{Rep}}
\def\Repf{\operatorname{Rep}^{\text{fd}}}
\def\Res{\operatorname{Res}}
\def\Ind{\operatorname{Ind}}
\def\XX{\mathbb{X}}
\def\kg{\mathfrak g}
\def\Lie{\operatorname{Lie}}
\def\S{\mathfrak{S}}
\def\car{\operatorname{char}}
\def\ch{\operatorname{ch}}
\def\Frac{\operatorname{Frac}}
\def\Fr{\operatorname{Fr}}
\def\F{\mathbb{F}}
\def\res{^{\mathrm{res}}}
\def\aff{_{\mathrm{aff}}}
\def\R{\bb{R}}
\def\Stab{\operatorname{Stab}}

\title{Representations of Reductive Algebraic Groups}
\author{Lecturer: Simon Riche,\quad Typesetter: Micha{\l} Mruga{\l}a}
\begin{document}
\maketitle
\tableofcontents

\newpage

\section{Generalities on affine group schemes and smooth representations}
\subsection{Affine group schemes}
Fix $k$ a base field.

Recall that an \defn{affine $k$-group scheme} is one of the following data:
\begin{enumerate}[(1)]
	\item An affine scheme $G$ over $k$ endowed with morphisms of $k$-schemes
		\begin{itemize}
			\item $m:G\times G\to G$;
			\item $e:\Spec(k)\to G$ ;
			\item $\i:G\to G$;
		\end{itemize}
		which satisfy the usual axioms of groups (with $m$ multiplication, $e$ the unit, $\i$
		the inverse).
	\item A functor $\Alg_k = \{k\text{-algebras}\} \xrightarrow{F} \Grp$ such that the
		composition $\Alg_k \xrightarrow{F}\Grp\to \Set$ is representable.
	\item A commutative Hopf algebra over $k$, i.e. a commutative algebra $A$ with morphisms
		of $k$-algebras
		\begin{itemize}
			\item $\Delta:A \to A\otimes A$ ;
			\item $\eps:A\to k$ ;
			\item $S:A\to A$ ;
		\end{itemize}
		such that
		\begin{align*}
			(\id \otimes\Delta) \circ \Delta = (\Delta \otimes \id) &: A \to A\otimes A \otimes A \\
			(\id\otimes \eps) \circ \Delta = \id = (\eps\otimes\id)\circ\Delta &: A \to A \\
			(\id,S)\circ \Delta = \Delta \circ \eps = (S,\id)\circ\Delta &:A \to A
		\end{align*}
\end{enumerate}
\begin{notation}
	\begin{align*}
		A &\to \Spec(A) \\
		G &\to \OO(G), \Delta_G.
	\end{align*}
\end{notation}
\begin{example} \leavevmode
	\begin{enumerate}[(1)]
		\item \emph{Diagonalizable groups:} If $\Lambda$ is an abstract commutative group we
			have the affine $k$-group scheme $\Diag(\Lambda)\defeq\Spec(k[\Lambda])$ with
			\[
				\Delta(\lambda) = \lambda \otimes \lambda,\quad \eps(\lambda)=1,\quad
				S(\lambda)=\lambda^{-1}\quad (\forall \lambda\in \Lambda).
			\]
			In particular for $\Lambda=\Z$, $k[\Lambda]=k[x,x^{-1}]$ and $\Diag(\Lambda)=\Gm$ (the
			\defn{multiplicative group}).

			A \defn{(split) torus} is a group scheme of the form $\Diag(\Lambda)$ with $\Lambda$ a
			finitely generated free abelian group.
		\item \emph{Additive group:} $\Ga\defeq\Spec(k[x])$ with
			\[
				\Delta(x) = x\otimes 1 + 1\otimes x,\quad \eps(x)=0,\quad S(x)=-x.
			\]
			More generally for $V$ a $k$-vector space we have the functor $V_{a}:R\mapsto
			(R\otimes V,-)$ which is an affine $k$-group scheme if $V$ is finite dimensional.
		\item If $V$ is a $k$-vector space, $\GL(V)$ is the functor $R\mapsto \Aut_R(R\otimes
			V)$. If $V$ is finite dimensional this is an affine $k$-group scheme.

			In particular, if $V=k^{n}$ we get
			\[
				\GL_n = \Spec(k[x_{ij},1\le i,j\le n][\det^{-1}])
			\]
			with
			\[
				\Delta(x_{ij}) = \sum_l x_{il}\otimes x_{lj},\quad \eps(x_{ij}) = \delta_{ij}
			.\]
			Similarly we have $\SL(V), \SL_n$.
		\item For any abstract group $\Gamma$ we have the functor (??)
	\end{enumerate}
\end{example}

\subsection{Representations}

If $G$ is an affine $k$-group scheme, a \defn{representation} of $G$ is the datum of a
$k$-vector space $V$ and a morphism of group valued functors $G\to \GL(V)$.
[Equivalently, an action of $G$ on $V_{a}$ such that $G(R)$ acts $R$-linearly on $R\otimes V$.]

This datum is equivalent ot that of a \defn{comodule} for $\OO(G)$, i.e. a $k$-vector
space $V$ and a $k$-linear map $\Delta_V:V\to V\otimes \OO(G)$ such that
\begin{align*}
	(\Delta_V\otimes\id_{\OO(G)}\circ \Delta_V = (\id_V\otimes \Delta_G)\circ \Delta_V
		&:V\to V\otimes\OO(G)\otimes\OO(G) \\
		(\id_V\otimes \eps)\otimes \Delta_V = \id_V &:V\to V.
	\end{align*}
	[$\Delta_V$ corresponds to the image of $\id_{\OO(G)}\in
	G(\OO(G))=\End_{k\text{-alg}}(\OO(G))$ in $\End_{\OO(G)}(\OO(G)\otimes V)$.]
	\begin{example} \leavevmode
		\begin{enumerate}[(1)]
			\item \emph{(Right) Regular representation:} $V=\OO(G)$ with $\Delta_V=\Delta_G$. More
				generally, given an action of $G$ on an affine scheme $X$ we get a representation
				with underlying vector space $\OO(X)$.
			\item If $V$ is a finite dimensional vector space, $V$ is a representation of $\GL(V)$.
			\item For any $G$ we have the trivial representation $k$.
		\end{enumerate}
	\end{example}
	\begin{notation}
		$\Rep(G)$ is the abelian category of representations of $G$. $\Repf(G)$ is the full
		subcategory of finite dimensional representations.
	\end{notation}

	If $V\in \Rep(G)$ then $V$ is the union of its finite dimensional subrepresentations.

	\begin{example}[Representations of diagonalizable group schemes]
		Let $\Lambda$ be a commutative group, $G=\Diag(\Lambda)$. If $V\in \Rep(G)$ we have
		\[
			\Lambda_V:V\to V\otimes\OO(G) = \bigoplus_{\lambda\in \Lambda}V\otimes\lambda
		.\]
		Hence there are morphisms ($\rho_\lambda:\lambda\in \Lambda$) in $\End(V)$ such that
		\[
			\Delta_V(v) = \sum_{\lambda\in \Lambda} \rho_\lambda(v)\otimes\lambda,\qquad \forall v\in V.
		\]
		(Here $\rho_\lambda(v)=0$ for all but finitely many $\lambda$s.)

		It is easy to see that
		\[
			\rho_\lambda\circ\rho_\mu =
			\begin{cases}
				\rho_\lambda & \text{if }\lambda=\mu \\
				0 & \text{otherwise}
			\end{cases}
		\]
		and $\id = \sum_{\lambda_\Lambda}\rho_\lambda$.

		Hence $V=\bigoplus_{\lambda\in \Lambda}\rho_\lambda(V)$ with
		\[
			\rho_\lambda(V) = \{v\in V : \Delta_V(v) = v\otimes \lambda\} = V_\lambda
		.\]
		Hence $\Rep(G)$ is isomorphic to the category of $\Lambda$-graded vector spaces (correct?).
	\end{example}
	\subsection{Induction} Let $G$ be an affine $k$-group scheme.

	A \defn{subgroup} of $G$ is a closed subscheme $H\subset G$ such that
	$e,\i|_{H},m|_{H\times H}$ factor through $H$. Then $H$ is an affine $k$-group scheme.
	In this setting we have the restriction functor $\Res_{H}^{G}:\Rep(G)\to \Rep(H)$.

	\begin{proposition}
		The functor $\Res_{H}^{G}$ has a right adjoint $\Ind_{H}^{G}:\Rep(H)\to \Rep(G)$.
	\end{proposition}
	Explicitly, we have
	\[
		\Ind_{H}^{G}(V) = (V\otimes\OO(G))^H
	\]
	with $H$ acting diagonally via the right-regular representation on $\OO(G)$ and $G$
	acting on the fixed points via the left regular representation
	\[
		\Ind_{H}^{G}(V) = \left\{
			\begin{array}{c}
				\text{morphisms of functors} \\
				f :G\to V_{a}
			\end{array} \middle|
			\begin{array}{c}
				f(gh)=h^{-1}f(g) \\
				\forall g\in G(R), h\in H(R), R\in \cat{Alg}_k
		\end{array} \right\}
	.\]
	The canonical isomorphism
	\[
		\Hom_{\Rep(G)}(V,\Ind_{H}^{G}(V')) \isoto \Hom_{\Rep(H)}(V,V')
	\]
	is called \defn{Frobenius reciprocity}.

	\begin{properties} \leavevmode
		\begin{itemize}
			\item \emph{Transitivity:} Given subgroups $H_1\subset H_2\subset G$ we have
				\[
					\Ind_{H_1}^{G} \isoto \Ind_{H_2}^{G}\circ \Ind_{H_1}^{H_2}.
				\]
			\item \emph{Tensor identity:} For $V_1\in \Rep(H),V_2\in \Rep(G)$
				\[
					\Ind_{H}^{G}(V_1\otimes\Res_{H}^{G}(V_2)) \isoto \Ind_{H}^{G}(V_1) \otimes V_2.
				\]
			\item $\Ind_H^{G}$ sends injective objects of $\Rep(H)$ to injective objects of
				$\Rep(G)$. In particular,
				\[
					\Ind_H^{G}(k) = \OO(G)
				\]
				is injective.
			\item $\Rep(G)$ has enough injectives.
		\end{itemize}
	\end{properties}

	\emph{Geometric interpretation:} We assume $G$ is an algebraic group (over $k$), i.e. an
	affine $k$-group scheme such that $\OO(G)$ is a finitely generated $k$-algebra. In this
	setting, for $H\subset G$ a subgroup we have a quotient scheme $G /H$ of finite type
	over $k$ with a faithfully flat quotient map $\pi:G\to G /H$. For $V\in \Rep(H)$, we
	have a quasicoherent sheaf $\CL_{G /H}(V)\in \QCoh(G /H)$ with
	\[
		\Gamma(V,\CL_{G /H}(V)) = \left\{ \text{morphisms } f:\pi^{-1}(V)\to V \middle|
		f(x,h)=h^{-1}f(x)\text{ for all }(?)  \right\}
	.\]
	We have $\Ind_{H}^{G}(V) = \Gamma(G /H,\CL_{G /H}(V))$. If $V$ is finite dimensional,
	then $\CL_{G /H}(V)$ is coherent.

	\begin{consequences} \leavevmode
		\begin{itemize}
			\item If $G /H$ is affine then $\Ind_{H}^{G}$ is exact.
			\item If $G /H$ is projective then $\Ind_{H}^{G}$ preserves finite dimensionality.
		\end{itemize}
	\end{consequences}

	Since $\Rep(H)$ has enough injectives we can consider the derived functor
	\[
		\RD\Ind_H^{G}:D^{b}\Rep(H)\to D^{b}\Rep(G).
	\]
	The functor $\CL_{G /H}:\Rep(H)\to \cat{QCoh}(G /H)$ is exact, hence we have
	\[
		\CL_{G /H}:D^{b}\Rep(H)\to D^{b}\cat{QCoh}(G /H).
	\]
	One can check that
	\[
		\RD\Ind_H^{G}(V) \isoto R\Gamma(G /H, \CL_{G /H}(V)).
	\]
	(??)

	\begin{consequences} \leavevmode
		\begin{itemize}
			\item We have $\RD^{n}\Ind_H^{G}(V)=0$ for all $V\in \Rep(H)$ if $n>\dim(G /H)$.
			\item If $G /H$ is projective, then $\RD^{n}\Ind_H^{G}(V)$ is finite-dimensional for
				all $V\in \Repf(H), n\in \Z$.
		\end{itemize}
	\end{consequences}

	\newpage

	\section{Reductive algebraic groups}
	From now on $k$ is algebraically closed.

	\subsection{Definition}
	A $k$-algebraic group $G$ is called \defn{unipotent} if every non-zero representation
	admits a non-zero fixed vector. [Equivalent condition: $G$ is unipotent if and only if
	it is isomorphic to a subgroup of unipotent upper-triangular matrices in $\GL_n$ for some $n$.]
	\begin{example}
		$\Ga$ is unipotent as
		\[
			\Ga \isoto
			\begin{pmatrix} 1 & * \\ 0 & 1
			\end{pmatrix} .
		\]
	\end{example}
	If $G$ is a smooth, connected algebraic group, the smooth, connected, unipotent, normal
	subgroups of $G$ there is a largest element called the \defn{unipotent radical} of $G$,
	denoted $R_u(G)$. An algebraic group $G$ is called \defn{reductive} if it is smooth,
	connected and $R_u(G)$ is trivial.

	One possible motivation for studying representations of reductive algebraic groups is
	that any simple representation of a smooth connected algebraic group $G$ factors through
	a simple representation of $G /R_u(G)$, which is a reductive algebraic group.

	\begin{example} \leavevmode
		\begin{enumerate}[(1)]
			\item \emph{Tori:} If $\Lambda$ is a finitely generated, free abelian group, then
				$\Diag(\Lambda)$ is a reductive algebraic group.
			\item For any finite-dimensional $k$-vector space $V$, $\GL(V)$ and $\SL(V)$ are
				reductive algebraic groups.
			\item Symplectic groups, special orthogonal groups.
		\end{enumerate}
	\end{example}
	\subsection{Structure} From now on $G$ is a redutive algebraic group.

	We denote by $B$ a \defn{Borel subgroup} (a maximal, connected, smooth, solvable
	subgroup). Note that:
	\begin{itemize}
		\item a Borel subgroup is unique up to conjugation;
		\item the quotient $G /B$ is a smooth, projective variety.
	\end{itemize}
	\begin{example}[Main Example]
	 For $G=\GL_{n,k}$ one can take
	 \[
			B = \left\{
				\begin{pmatrix} * & & 0 \\ & \ddots & \\ * & & *
			\end{pmatrix}  \right\} .
	 \]
	 In this case $G /B$ parametrizes flags in $k^{n}$, i.e. data
	 \[
	  \{0\} \subset V_1\subset \dots\subset V_{n-1}\subset k^{n}
	 \]
	 with $V_i$ a subspace of dimension $i$.
	\end{example}
	Let $T$ be a maximal torus contained in $B$.
	\begin{example}[Main Example Continued]
		We take
		\[
			T = \left\{
				\begin{pmatrix} t_1 & & 0 \\ & \ddots \\ 0 & & t_n
			\end{pmatrix}  \right\} .
		\]
	\end{example}
	Note that
	\[
		T\isoto \Diag(\XX),\quad \XX = \{\text{morphisms }T\to \Gm\}
	\]
	we call elements of $\XX$ \defn{weights}.
	\begin{example}[Main Example Continued]
		We have $\XX\isoto \Z^{n}$ via
		\[
			(\lambda_1,\dots,\lambda_n) \leftrightarrow
			\begin{pmatrix} t_1 & & 0 \\ & \ddots \\ 0 & & t_n
			\end{pmatrix} \mapsto \prod_{i=1}^{n} t_i^{\lambda_i} .
		\]
	\end{example}
	The \defn{roots} $R\subset \XX$ are the non-zero weights appearing in the action of $T$
	on $\kg=\Lie(G)$.
	\begin{example}[Main Example Continued]
		We have
		\[
			R = \{\eps_i - \eps_j: 1\le i\neq j \le n\} .
		\]
	\end{example}
	We define the \defn{positive roots} $R_+\subset R$: the weights appearing in the action
	of $T$ on $\kg /\Lie(B)$, and the \defn{simple roots} $R_s\subset R_+$: positive roots
	that cannot be written as a sum of two positive roots.
	\begin{note}
		$R=R_+ \coprod -R_{+}$. Any element of $R_+$ can be uniquely written as a sum of simple roots.
	\end{note}
	\begin{example}[Main Example Continued]
		In our case
		\begin{align*}
			R_+ &= \{\eps_i - \eps_j:1\le i<j<n\} \\
			R_s &= \{\eps_i - \eps_{i+1}\} .
		\end{align*}
	\end{example}

	The \defn{cocharacters} of $G$ are
	\begin{align*}
		\XX^{\vee} &= \Hom_\Z(\XX,\Z) \\
		&= \{\text{morphisms }\Gm\to T\} .
	\end{align*}
	We have \defn{coroots} $R^{\vee}\subset \XX^{\vee}$ and a bijection
	\begin{align*}
		R &\longrightarrow R^{\vee} \\
		\alpha &\longmapsto \alpha ^{\vee}.
	\end{align*}
	Then $(\XX,R,\XX^{\vee},R^{\vee})$ together with the identification
	$\XX^{\vee}=\Hom(\XX,\Z)$ and the bijection $R\to R^{\vee}$ is the \defn{root datum} of
	$G$. It determines $G$ up to isomorphism.

	There is an opposite Borel subgroup $B^{+}\subset G$ containing $T$ such that the
	non-zero weights of $T$ acting on $\Lie(B^{+})$ are $R_+$.

	$W=N_G(T) / T$ is a the \defn{Weyl group}, it is a constant group scheme, associated
	with a finite group also denoted $W$. $W$ acts faithfully on $\XX$. For $\alpha\in R$
	there is an element $s_\alpha\in W$ which acts on $\XX$ via
	\[
		\lambda \mapsto \lambda - \left<\lambda,\alpha ^{\vee} \right>\alpha.
	\]
	If we set
	\[
		S = \{s_\alpha : \alpha\in R_s\} \subset W,
	\]
	then $(W,S)$ is a Coxeter system. In particular, we have the length function
	\begin{align*}
		\ell: W &\longrightarrow \Z_{\ge 0} \\
		w &\longmapsto \min \{r\ge 0 | \text{there exist }s_1,\dots,s_r\in S\text{ such that
		}w=s_1\cdots s_r\} .	
	\end{align*}
	\begin{example}[Main Example Continued]
		For example, $W=\S_n$ is the symmetric group via permutation matrices. The action on
		$\XX=\Z^{n}$ is by permuting entries
		\[
			S = \{(i,i+1):1\le i<n\}
		\]
		The length function counts inversions of permutations.
	\end{example}
	\begin{example}
		Let $G=\SL_2$,
		\begin{align*}
			B &= \left\{
				\begin{pmatrix} * & 0 \\ * & *
			\end{pmatrix}  \right\} \subset \SL_{2,k} \\
			T &= T = \left\{
				\begin{pmatrix} t & 0 \\ 0 & t^{-1}
			\end{pmatrix} \middle| t\in k^{\times } \right\} \isoto \Gm.
		\end{align*}
		We have $\XX\isoto \Z$ via
		\[
			\lambda \leftrightarrow \left[
				\begin{pmatrix} t & 0 \\ 0 & t^{-1}
			\end{pmatrix} \mapsto t ^{\lambda} \right] .
		\]
		We have $R=\{2,-2\} $, $R_+=\{2\} =R_s$ and $W=\S_2=\Z /2\Z$.
	\end{example}

	\subsection{Classification of simple representations}
	The Borel is a semidirect product $B=T\ltimes U$ with $U=R_u(B)$. Similarly
	$B^{+}=T\ltimes U^{+}$ with $U^{+}=R_u(B^{+})$. In particular, $T\simto B /U$, so any
	$\lambda\in \XX$ provides a morphism $B\to \Gm$, which is a one-dimensional
	representation $k_B(\lambda)$. Define
	\[
		\nabla(\lambda) = \Ind_B^{G}(k_B(\lambda)).
	\]
	It's easy to see that:
	\begin{itemize}
		\item $\dim(\nabla(\lambda))<\infty$ for all $\lambda\in \XX$ (because $G /B$ is projective).
		\item The action of $T$ on $\nabla(\lambda)$ determines an $\XX$-grading
			\[
				\nabla(\lambda) = \bigoplus_{\mu\in \XX}\nabla(\lambda)_\mu.
			\]
			Here if $\nabla(\lambda)\neq 0$, we have
			\begin{itemize}
				\item $\nabla(\lambda)_\lambda = \nabla(\lambda)^{U^{+}}$ and this is one-dimensional,
				\item if $\nabla(\lambda)_\mu \neq 0$ then $\lambda-\mu\in \Z_{\ge 0}R_s$.
			\end{itemize}
			This follows from the open embedding
			\[
				U^{+}\times B\injto G
			\]
			induced by multiplication.
	\end{itemize}
	\begin{corollary}
		We have a bijection
		\begin{align*}
			\left\{ \lambda\in \XX \middle| \nabla(\lambda)\neq 0 \right\} &\simto \left\{
			\text{simple objects in }\Rep(G) \right\} /\isoto \\
			\lambda &\mapsto L(\lambda) = \text{ unique simple subrepresentation in }\nabla(\lambda).
		\end{align*}
	\end{corollary}
	It's less easy to show:
	\begin{proposition}
		For $\lambda\in \XX$, we have
		\[
			\nabla(\lambda) \neq 0 \quad \iff\quad \forall \alpha\in R_s, \left<\lambda,\alpha
			^{\vee} \right> \ge 0.
		\]
	\end{proposition}
	\begin{proof}[Idea of the proof]
		The forward direction is easy using the fact that $W$ permutes
		\[
			\left\{ \mu\in \XX \middle| \nabla(\lambda)_\mu \neq 0 \right\} .
		\]

		Conversely, one can construct a function
		\[
			\bigcup_{\alpha\in R_s} s_\alpha U^{+}B\to k
		\]
		and then use the fact that the LHS has complement of codimension 2 in $G$, cf. Bruhat
		decomposition.
	\end{proof}
	We set
	\[
		\XX_+ = \left\{ \lambda\in \XX \middle| \forall \alpha\in R_s, \left<\lambda,\alpha
		^{\vee} \right> \ge 0 \right\}
	\]
	the \defn{dominant weights}.
	\begin{example}
		\begin{enumerate}[(1)]
			\item $\nabla(0)=k$ is the trivial representation (because $G /B$ is connected and projective).
			\item Let $G=\GL_{n,k}$
				\[
					\XX_+ = \left\{ (\lambda_1,\dots,\lambda_n) \middle| \lambda_1\ge \lambda_2\ge
					\dots\ge \lambda_n \right\} .
				\]
				For $r\ge 0$
				\begin{align*}
					\nabla(r,0,\dots,0) \simeq S^{r}(V)
				\end{align*}
				with $V=k^{n}$ the natural representation, and
				\[
					\nabla(0,\dots,0,-r) \simeq S^{r}(V^{*})
				\]
				(cf. sections of line bundles on $\P^{n}$).

				For $s \in \left\{ 1,\dots,n \right\} $,
				\begin{align*}
					\nabla(\underbrace{1,\dots,1}_{s},0,\dots,0) &= \bigwedge^{s}(V) \\
					&= L(\underbrace{1,\dots,1}_s,0,\dots,0).
				\end{align*}
				For $r\in \Z$
				\[
					\nabla(r,\dots,r) = k_{\det ^{r}} = L(r,\dots,r).
				\]
			\item Let $G=\SL_2$, then $\XX_+ = \Z_{\ge 0}$. For $r\ge 0$
				\[
					\nabla(r) = S^{r}(k^2)
				\]
				(cf. sections of line bundles on $G /B=\P^{1}$). If $\car(k)=0$ then $\nabla(r)$ is
				simple for all $r\ge 0$. If $\car(k)=p>0$ this is not always true:
				\[
					\nabla(p) = kx^{p} \oplus kx^{p-1}y \oplus \dots \oplus kxy^{p-1} \oplus ky^{p}
				\]
				with $x,y$ a canonical basis of $k^2$. Then $kx^{p}\oplus ky^{p}$ is a non-trivial
				$G$-stable subspace. In fact, $L(p) = kx^{p}\oplus ky^{p}$.

				More generally, $\nabla(r)$ is simple if and only if $r\le p-1$.
			\item For all $\lambda\in \XX_+$, $L(\lambda)^{*}\simeq L(-w_0\lambda)$ where $w_0\in
				W$ is the longest element.
		\end{enumerate}
	\end{example}

	\subsection{Characters}
	If $V\in \Repf(G)$ then the action of $T$ determines a grading $V=\bigoplus_{\lambda\in
	\XX}V_\lambda$ with
	\[
		V_\lambda = \left\{ v\in V \middle| \forall t\in T,tv = \lambda(t)v \right\} .
	\]
	We set
	\[
		\ch(V) = \sum_{\lambda\in \XX}\dim(V_\lambda)e^{\lambda}\in \Z[\XX].
	\]
	It's easy to check that:
	\begin{itemize}
		\item $\ch$ factors through $K^{0}(\Repf(G))\to \Z[\XX]$.
		\item $\ch(V\otimes V') = \ch(V)\ch(V')$, so the map above is a \emph{ring morphism}.
		\item $\ch$ takes values in $\Z[\XX]^{W}$.
	\end{itemize}
	\begin{proposition}
		$\ch$ induces an isomorphism
		\[
			K^{0}(\Repf(G)) \simto \Z[\XX]^{W}.
		\]
	\end{proposition}
	\begin{proof}[Proof idea]
		Show that
		\[
			\left\{ \ch(L(\lambda)) \middle| \lambda\in \XX_+ \right\}
		\]
		is a basis of $\Z[\XX]^{W}$.
	\end{proof}

	\section{Some general results about ? of reductive algebraic groups}
	\subsection{Kempf's vanishing theorem}
	\begin{theorem}
		If $\lambda\in \XX_+$ then
		\[
			\RD^{n}\Ind_B^{G}(k_B(\lambda)) = 0 \quad \forall n>0.
		\]
	\end{theorem}
	\begin{note}
		We have
		\[
			\RD^{n}\Ind_B^{G}\left( k_B(\lambda) \right) = \CH^{n}(G /B, \CL_{G /B}(k_B(\lambda))),
		\]
		where $\CL_{G /B}(k_B(\lambda))$ is a line bundle equal to $\OO_{G /B}(\lambda)$.
	\end{note}
	So we are in fact computing cohomology of some line bundles on $G /B$.

	\emph{Closely related fact:} (??)

	In fact, in case $p=0$, we get Kempf's vanishing theorem from this proposition using the
	\emph{Kodaira vanishing theorem}.
	\begin{example}\leavevmode
		\begin{enumerate}[(1)]
			\item $\CH^{n}(G /B,\OO_{G /B})=0$ for $n>0$, this is the $\lambda=0$ case.
			\item For $G=\SL_{2,k}, \XX=\Z, G /B=\P^{1}$ and $\OO_{G
				/B}(\lambda)=\OO_{\P^{1}}(\lambda)$. So we recover the fact that
				\[
					\CH^{n}(\P^{1},\OO_{\P^{1}}(m)) = 0
				\]
				if $n>0$ and $m\ge 0$.
		\end{enumerate}
	\end{example}
	\begin{remark}
		Serre duality for $G /B$:  $\omega_{G /B}\isoto \OO_{G /B}(-\rho)$ where
		\[
			\rho = \frac{1}{2}\sum_{\alpha\in R_+}\alpha
		\]
		implies that for $\lambda\in \XX$ we have
		\[
			\RD^{n}\Ind_{B}^{G}(k_B(\lambda)) \isoto \left( \RD^{|R^{+}|-n}\Ind_B^{G}\left(
			-(\lambda+2\rho) \right)  \right) ^{*}
		\]
		(note that $|R_+| = \dim(G /B)$). So if $\lambda\in -2\rho-\XX_+$ then
		$\RD^{n}\Ind_B^{G}(k_B(\lambda))=0$ if $n\neq |R_+|$.

		For $\SL_2$, this says
		\[
			\CH^{n}(\P^{1},\OO_{\P^{1}}(m)) = 0
		\]
		for $n\neq 1$ if $m\le -2$.
	\end{remark}

	Here is an interesting application. For $\lambda\in \XX_+$ we set
	\[
		\Delta(\lambda) = \left( \nabla (-w_0\lambda \right) )^{*} =
		\RD^{|R|_+}\Ind_B^{G}\left( k_B(w_0\lambda-2\rho) \right)
	\]
	($w_0$ is longest length in $W$). These modules are called \defn{Weyl modules}. We have
	\[
		\Delta(\lambda) \surjto L(\lambda).
	\]

	\begin{proposition}
		For $\lambda,\mu\in \XX_+$ we have
		\[
			\Ext_{\Rep(G)}^{n}\left( \Delta(\lambda),\nabla(\mu) \right) =
			\begin{cases}
				k & \text{if }\lambda=\mu, n=0 \\
				0 & \text{otherwise}.
			\end{cases}
		\]
	\end{proposition}
	The unique (up to scalar) non-zero morphism for $\lambda=\mu$ and $n=0$ is the composition
	\[
		\Delta(\lambda) \surjto L(\lambda) \injto \nabla(\lambda).
	\]
	This statement says that $\Rep(G)$ is a ``highest weight category''.

	\subsection{Borel-Bott-Weil theorem}
	We consider the action of $W$ on $X$ given by
	\[
		w\cdot \lambda = w(\lambda+\rho) - \rho.
	\]
	More precisely, this defines an action on $\Q\otimes_\Z\XX$, which stabilizes $\XX$
	since $w\rho-\rho\in \Z R$ for all $w\in W$.

	Set
	\[
		\overline{C} =
		\begin{cases}
			\left\{ \lambda\in \XX \middle| \forall \beta\in R_+, \left<\lambda+\rho,\beta^{\vee}
			\right> \ge 0 \right\} & \text{if }p=0 \\
			\left\{ \lambda\in \XX \middle| \forall \beta\in R_+, 0\le
			\left<\lambda+\rho,\beta^{\vee} \right> \right\} & \text{if }p>0.
		\end{cases}
	\]
	\begin{example}\leavevmode
		\begin{enumerate}[(1)]
			\item Let $G=\SL_2, \XX=\Z$, then
				\[
					\overline{C} =
					\begin{cases}
						\{ -1,0,1,\dots \} & \text{if }p=0 \\
						\{ -1,0,\dots,p-1 \} & \text{if }p>0.
					\end{cases}
				\]
			\item Let $G=\SL_3$, then (picture).
		\end{enumerate}
	\end{example}
	\begin{theorem}[Borel-Bott-Weil]\leavevmode
		\begin{enumerate}[(1)]
			\item If $\lambda\in \overline{C}\setminus \XX_+$ then
				\[
					\RD^{n}\Ind_B^{G}\left( k_B(w\cdot \lambda) \right) = 0
				\]
				for all $w\in W, n\in \Z$.
			\item If $\lambda\in \overline{C}\cap \XX_+$, then for $n\in \Z, w\in W$, we have
				\[
					\RD^{n}\Ind_B^{G}\left( k_B(w\cdot \lambda) \right) =
					\begin{cases}
						\nabla(\lambda) & \text{if }n=\ell(w) \\
						0 & \text{otherwise}.
					\end{cases}
				\]
		\end{enumerate}
	\end{theorem}
	The proof is by induction on $\ell(w)$, the case $\ell(w)=0$ follows from Kempf's
	vanishing theorem. This uses a decomposition of
	$\RD^{n}\Ind_B^{P(\alpha)}(k_B(\lambda))$ for $\alpha\in R_+$, which is a $\SL_2$ computation (?)

	\begin{remark}
		If $p=0$ then $W\cdot C=\XX$, so we understand all $\RD^{n}\Ind_B^{G}(k_B(\lambda))$.
		For $p>0$, this then only describes a small number of such spaces. In particular, in
		this case there can exist $\lambda\in \XX$ such that
		$\RD^{n}\Ind_B^{G}(k_B(\lambda))\neq 0$ for several $n$'s.
	\end{remark}
	\begin{corollary}\leavevmode
		\begin{enumerate}[(1)]
			\item If $\lambda\in \overline{C}\cap \XX_+$ then $\nabla(\lambda)=L(\lambda)$.
			\item If $\lambda,\mu\in \overline{C}\cap \XX_+$ then
				\[
					\Ext_{\Rep(G)}^{1}(L(\lambda),L(\mu)) = 0.
				\]
				In particular, if $p=0$ the category $\Rep(G)$ is semisimple.
		\end{enumerate}
	\end{corollary}
	\begin{proof}\leavevmode
		\begin{enumerate}[(1)]
			\item By BBW and Serre duality
				\[
					\nabla(\lambda)^{*} \isoto \nabla(-w_0\lambda).
				\]
				Hence $\nabla(\lambda)$ has a unique simple quotient isomorphic to $L(\lambda)$.
				Since $L(\lambda)$ is also the unique simple submodule of $\nabla(\lambda)$ and has
				multiplicity 1 as a composition factor (because $\dim \nabla(\lambda)_\lambda=1$) we
				must have $\nabla(\lambda)\isoto L(\lambda)$.
			\item We have
				\[
					\Ext ^{1}(L(\lambda),L(\mu)) \isoto \Ext ^{1}(\Delta(\lambda), \nabla(\mu)) \isoto 0.
				\]
				Then we use local finiteness of representations.
		\end{enumerate}
	\end{proof}
	\begin{remark}\leavevmode
		\begin{enumerate}[(1)]
			\item In Milne's book (§22.C) there is a different proof of semisimplicity using the
				action of $\kg$ and Casimir operators.
			\item For a connected algebraic group $H$ over $k$ of characteristic 0 one proves that
				$H$ is reductive if and only if every finite dimensional representation is
				semisimple. [Thoerem 22.42 in Milne's book.]
		\end{enumerate}
	\end{remark}
	\begin{example}
		For $G=\SL_2$, write $V=k^2$ for the natural representation. One recovers that
		$S^{r}(V)$ is simple if $p=0$ or $p>0$ and $r<p$.
	\end{example}

	\subsection{Weyl's character formula}
	For $\lambda\in \XX$ we set
	\[
		\chi(\lambda) = \sum_{n\ge 0}(-1)^{n}\ch\left( \RD^{n}\Ind_B^{G}(k_B(\lambda)) \right)
	\]
	\begin{note}
		If $\lambda\in \XX_+$, by Kempf's vanishing theorem we have
		\[
			\chi(\lambda) = \ch(\nabla(\lambda)).
		\]
	\end{note}
	\begin{theorem}
		For $\lambda\in \XX$ we have
		\[
			\chi\left( \lambda \right) = \frac{\sum_{w\in W}(-1)^{\ell(w)}e^{w\cdot
			\lambda}}{\sum_{w\in W}(-1)^{\ell(w)}e^{w\cdot 0}} \in \Frac(\Z[X])
		\]
	\end{theorem}
	In particular, this gives a formula for $\ch\left( \nabla(\lambda) \right) $ (this
	formula does not depend on $p$!). Along the way to prove this theorem, one proves that
	$\ch(\Lambda(\lambda)) = \ch(\nabla(\lambda))$.

	\begin{remark}
		The proof in general follows from an analysis of
		$\ch\RD^{1}\Ind_B^{P(\alpha)}(k_B(\lambda))$ for $\alpha\in R_+$ already used for the BBW theorem.

		When $p=0$, there is an alternative proof using a Lefschetz-type fixed point formula
		[cf. reference in Milne's book or §6.1.16 in Chriss-Ginzburg \emph{Representation
		theory and complex geometry}].
	\end{remark}

	\section{The case of positive characteristic}
	We assume that $p>0$.
	\subsection{Frobenius morphism and Frobenius kernel}
	Set
	\[
		G^{(1)} = \Spec(k)\times_{\Spec(k)} G
	\]
	where $\Spec(k)\to \Spec(k)$ corresponds to
	\begin{align*}
		k &\longrightarrow k \\
		x &\longmapsto x^{p}.
	\end{align*}
	In other words, $\OO(G^{(1)}=\OO(G)$ with $k$ acting by $\lambda\cdot f = \lambda^{1
		/p}f$ for $\lambda\in k, f\in \OO(G)$.

		We have a Frobenius morphism $\Fr:G\to G^{(1)}$ associated with
		\begin{align*}
			\OO(G^{(1)}) &\longrightarrow O(G) \\
			f &\longmapsto f^{p}.
		\end{align*}
		Here $G^{(1)}$ is an affine $k$-group scheme and $\Fr_G$ is a morphism of $k$-group schemes.

		We have
		\[
			T^{(1)} \subset B^{(1)}\subset G^{(1)}
		\]
		a maximal torus and Borel subgroup and $G^{(1)}$ is again reductive.

		We have
		\begin{align*}
			\phi: X^{*}(T^{(1)}) &\longrightarrow \XX \\
			\lambda &\longmapsto \lambda\circ\Fr_T.
		\end{align*}
		This morphism is injective, with image $p\XX$. The roots of $G^{(1)}$ with respect to
		$T^{(1)}$ are $pR$.
		\begin{remark}
			$G\isoto G^{(1)}$ as $k$-group schemes. A choice of such isomorphism amounts to
			choosing a ``lift'' of $G$ to an $\F_p$-group scheme.
		\end{remark}
		The \defn{Frobenius kernel} is $G_1=\ker(\Fr_G)$ (scheme-theoretic kernel). Here
		$\OO(G_1)$ is a finite-dimensional Hopf-algebra and
		\[
			\OO(G_1)^{*} \isoto U\kg / \left<x^{p} - x^{[p]} \right>
		\]
		where $(-)^{[p]}:\kg\to \kg$ is the restricted $p$-th power operation. The Frobenius
		morphism induces an isomorphism $G /G_1\simto G^{(1)}$.

		\subsection{Curtis' and Steinberg's theorems}
		Let
		\[
			\XX_+\res = \left\{ \lambda\in \XX \middle| \forall \alpha \in R_s, 0\le
			\left<\lambda,\alpha ^{\vee} \right>\le p-1 \right\} .	
		\]
		\begin{theorem}[Curtis]\leavevmode
			\begin{enumerate}[(1)]
				\item For $\lambda\in \XX_+\res$, teh representation $\Res_{G_1}^{G}(L(\lambda))$ is
					a simple $G_1$-representation.
				\item If $G$ is semisimple and simply connected, then
					\[
						\lambda\mapsto \Res_{G_1}^{G}(L(\lambda))
					\]
					induces a bijection
					\[
						\XX_+\res\simto \{\text{simple }G_1\text{-modules}\} /\sim.
					\]
			\end{enumerate}
		\end{theorem}
		\begin{remark}
			If $G$ is semisimple and simply connected, $\XX_+\res\simto \XX /p\XX$. For general
			$G$, simple $G_1$-modules are parametrized by $\XX /p\XX$.
		\end{remark}
		A closely related statement is:
		\begin{theorem}[Steinberg]
			If $\lambda\in \XX_+\res, \mu\in X^{*}(T^{(1)})_+$ we have
			\[
				L(\lambda)\otimes \Fr_G^{*}(L^{(1)}(\mu)) \isoto L(\lambda+\phi(\mu)).
			\] 	
		\end{theorem}
		If $G,B,T$ are obtained from similar data over $\F_p$ we have
		\[
			\begin{tikzcd}
				T \ar[r,hook] \ar[d,"\isoto"] & B \ar[r,hook] \ar[d,"\isoto"] & G \ar[d,"\isoto"] \\
				T^{(1)} \ar[r,hook] & B^{(1)} \ar[r,hook] & G^{(1)}
			\end{tikzcd}
		\]
		The formula becomes
		\[
			L(\lambda) \otimes \Fr_G^{*}(L(\mu)) \isoto L(\lambda+\rho\mu)
		\]
		for $\lambda\in \XX_+\res, \mu\in\XX_{+}$.

		More generally, for $\lambda_0,\dots,\lambda_r\in \XX_+\res,\mu\in \XX_+$.
		\[
			L(\lambda_0+p\lambda_1+p^2\lambda_2+\dots+p^{r}\lambda_r + p^{r+1}\mu) \isoto
			L(\lambda_0)\otimes \Fr_G^{*}L(\lambda_1) \otimes\dots\otimes
			(\Fr_G^{r})^{*}L(\lambda_r)\otimes(\Fr_G^{r+1})^{*}(\mu).
		\]
		Note that $L(p\lambda)\isoto \Fr_G^{*}L(\lambda)$ for all $\lambda\in \XX_+$.

		\begin{remark}
			If $G$ is semisimple and simply connected any $\lambda\in \XX_+$ can be written uniquely as
			\[
				\lambda_0+p\lambda_1+\dots+p^{r}\lambda_r
			\]
			with $\lambda_0,\dots,\lambda_r\in \XX_+\res$. This reduces the description of all
			simple representations to those corresponding to elements in $\XX_+\res$.
		\end{remark}
		\subsection{Linkage principle}
		\subsubsection{Affine Weyl group}
		Define
		\[
			W\aff \defeq W\ltimes \Z R
		\]
		which acts (affinely, not linearly) on $\R\otimes_\Z \XX$. A fundamental domain is
		\[
			\overline{A_0} = \left\{ v\in V \middle| \forall \alpha\in R_+, 0\le \left<v,\alpha
			^{\vee} \right>\le 1 \right\} .
		\]
		Set
		\[
			S\aff\defeq \left\{ w\in W\aff \middle| V^{W}\cap \overline{A_0}\text{ has codimension
			1 in }V \right\}.
		\]
		\begin{fact}
			The pair $(W\aff,S\aff)$ is a Coxeter system.
		\end{fact}
		\begin{example}\leavevmode
			\begin{enumerate}[(1)]
				\item Take $G=\SL_2, V=\R$, then $S\aff=\{s,s_0\} , S=\{s\} $. (Picture) $W\aff$ is
					the infinite dihedral group generated by $s,s_0$.
				\item Take $G=\SL_3$ and $\dim V=2$ (Picture).
			\end{enumerate}
		\end{example}
		What is relevant for the study of $\Rep(G)$ is the \defn{dot-action} of $W\aff$ on $\XX$
		\[
			(w,\lambda)\bullet \mu = w(\mu + p\lambda + \rho) - \rho.
		\]
		\begin{theorem}[Linkage principle]
			For $\lambda,\mu\in \XX_+$ we have
			\[
				\Ext_{\Rep(G)}^{1}(L(\lambda), L(\mu)) \neq 0 \implies W\aff\bullet \lambda = W\aff\bullet \mu.
			\]
		\end{theorem}
		For $c\in \XX(W\aff,\bullet)$ we set
		\[
			\Rep_c(G) = \left\{ M\in \Rep(G) \middle|
				\begin{array}{c}
					\text{all composition factors of }M \\
					\text{are of the form }L(\lambda)\text{ with }\lambda\in \XX_+\cap c
			\end{array}\right\}
		\]
		a full subcategory of $\Rep(G)$.
		\begin{corollary}
			We have
			\[
				\Rep(G) = \prod_{c\in \XX / (W\aff,\bullet)}\Rep_c(G).
			\]
		\end{corollary}
		\begin{remark}\leavevmode
			\begin{enumerate}[(1)]
				\item The linkage principle was conjectured by Verma. Proved by Humphreys,
					Carter-Lusztig, Jatzen, Andersen.
				\item Under mild assumptions on $p$, the linkage principle follows from a ``central
					character'' argument and the description of the center of $G$ and of $U\kg$.
				\item In fact, what Andersen proves is a ``strong linkage principle'': if $\lambda\in
					\XX$ satisfies
					\[
						\left<\lambda, \alpha ^{\vee} \right> \ge -1
					\]
					for all $\alpha\in R_s, w\in W$, if $L(\mu)$ is a composition factor of
					$\RD^{i}\Ind_B^{G}(k_B(w\bullet\lambda))$ then $\mu\in W\aff\bullet \lambda$.
			\end{enumerate}
		\end{remark}
		\begin{example}
			Let $G=\SL_2$, then
			\[
				\begin{tikzcd}
					0 \ar[r] & L(p) \ar[r] \ar[d,equals] & \nabla(p) \ar[r] \ar[d,equals] & L(p-2) \ar[r] & 0 \\
					& k\left<x^{p},y^{p} \right> & k\left<x^{p},x^{p-1}y,\dots,xy^{p-1},y^{p} \right>
				\end{tikzcd}
			\]
			(Picture)
		\end{example}
		\subsection{Translation functors}
		Recall
		\[
			\overline{C} = \left\{ \lambda\in \XX \middle| \forall \alpha\in R_+,0\le
			\left<\lambda+\rho,\beta^{\vee} \right>\le p \right\} .
		\]
		Then $\overline{C}$ is a fundamental domain for $W\aff$ acting on $\XX$. For
		$\lambda\in \overline{C}$ we consider the projection
		\[
			\pr_\lambda:\Rep(G) = \prod_{\mu\in \overline{C}}\Rep_{W\aff\bullet\mu}(G) \to
			\Rep_{W\aff\bullet\lambda}(G).
		\]
		Given $\lambda,\mu\in \overline{C}$, denote by $\nu$ the unique element in
		$W(\mu-\lambda)\cap \XX_+$ and set
		\begin{align*}
			T_\lambda^{\mu}: \Rep(G) &\longrightarrow \Rep(G) \\
			M &\longmapsto \pr_\mu\left( L(\nu)\otimes\pr_\lambda M \right) .
		\end{align*}
		\begin{fact}\leavevmode
			\begin{enumerate}[(1)]
				\item $T_\lambda^{\mu}$ is exact.
				\item $T_\lambda^{\mu}$ is left and right adjoint to $T_\mu^{\lambda}$.
			\end{enumerate}
		\end{fact}
		A subset $I\subset S\aff$ is called \defn{finitary} if $\left<I \right>\subset W\aff$
		is finite. Then
		\[
			\overline{C} = \coprod_{\substack{I\subset S\aff \\ \text{finitary}}} \overline{C}_I
		\]
		where
		\[
			\overline{C}_I = \left\{ \lambda\in \overline{C} \middle|
			\Stab_{(W\aff\bullet)}(\lambda)=\left<I \right> \right\} .
		\]
		\begin{example}
			Let $G=\SL_3$ (Picture)
		\end{example}
		\begin{theorem}
			If $\lambda,\mu\in \overline{C}_I$with $I$ finitary, then $T_\lambda^{\mu}$ and
			$T_\mu^{\lambda}$ restrict to quasi-inverse equivalences
			\[
				\Rep_{W\aff\bullet\lambda}(G)\simto \Rep_{W\aff\bullet\mu}(G).
			\]
		\end{theorem}
		For $I\subset S\aff$ finitary we set
		\[
			W\aff^{I} = \left\{ w\in W\aff \middle| \forall x\in W,y\in \left<I \right>, \ell(xwy)
			= \ell(x)+\ell(w)+\ell(y) \right\} .
		\]
		\begin{fact}
			For $\lambda\in \overline{C}_I$ we have an isomorphism
			\begin{align*}
				W\aff^{I} &\longrightarrow (W\aff\bullet\lambda)\cap \XX_+ \\
				w &\longmapsto w\bullet\lambda.
			\end{align*}
		\end{fact}
		So $W\aff^{I}$ induces the simples/induced modules in $\Rep_{W\aff\bullet\lambda}(G)$.
		For $\lambda,\mu$ as in the theorem
		\begin{align*}
			T_{\lambda}^{\mu}L(w\bullet\lambda) &\isoto L(w\bullet\mu) \\
			T_{\lambda}^{\mu}\nabla(w\bullet\lambda) &\isoto \nabla(w\bullet\mu).
		\end{align*}
		We have $C_{\emptyset}\neq \emptyset$ if and only if $p\ge h$ (Coxeter number of $G$).
		In this case, $0\in \overline{C}_{\emptyset}$. For $\lambda\in \overline{C}_I$, consider
		\[
			T_0^{\lambda}:\Rep_{W\aff\bullet 0}(G) \to \Rep_{W\aff\bullet\lambda}(G).
		\]
		\begin{proposition}\leavevmode
			\begin{enumerate}[(1)]
				\item For $W\aff^{\emptyset}$ we have
					\[
						T_0^{\lambda}\nabla(w\bullet 0) =
						\begin{cases}
							\nabla(w\bullet\lambda) & \text{if }w\bullet\lambda\in \XX_+ \\
							0 & \text{otherwise}.
						\end{cases}
					\]
				\item For $y\in W\aff^{I}$, $T_\lambda^{0}$ has a filtration with subquotients
					\[
						\{\nabla(yx\bullet 0)| x\in ?\} .
					\]
				\item For $w\in W\aff^{\emptyset}$ we have
					\[
						T_0^{\lambda}L(w\bullet 0) =
						\begin{cases}
							L(w\bullet 0) & \text{if }w\in W\aff^{I} \\
							0 & \text{otherwise}.
						\end{cases}
					\]
			\end{enumerate}
		\end{proposition}
		This reduces the study of $\Rep(G)$ to that of $\Rep_{W\aff\bullet 0}(G)$.

		\end{document}
