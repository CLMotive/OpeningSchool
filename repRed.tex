\documentclass[oneside]{amsart}

\usepackage[all]{xy}
\usepackage{tikz-cd}
\usepackage[T1]{fontenc}
\usepackage{xstring}
\usepackage{xparse}
\usepackage{xr-hyper}
\usepackage{xcolor}
\definecolor{brightmaroon}{rgb}{0.76, 0.13, 0.28}
\usepackage[linktocpage=true,colorlinks=true,hyperindex,citecolor=blue,linkcolor=brightmaroon]{hyperref}
\usepackage[nameinlink]{cleveref}
\usepackage[left=1.25in,right=1.25in,top=0.75in,bottom=0.75in]{geometry}
%\usepackage[charter,greekfamily=didot]{mathdesign}
%\usepackage{Baskervaldx}
\usepackage{amssymb}
\usepackage{stmaryrd}
\usepackage{mathrsfs}
\usepackage{mathpazo}
\linespread{1.05}

\usepackage[nobottomtitles]{titlesec}
\usepackage{marginnote}
\usepackage{enumerate}
\usepackage{longtable}
\usepackage{aurical}
\usepackage{microtype}

\newtheoremstyle{ega-env-style}%
{}{}{\rmfamily}{}{\bfseries}{.}{ }{\thmnote{(#3)}}%

\newtheoremstyle{ega-thm-env-style}%
{}{}{\itshape}{}{\bfseries}{. --- }{ }{\thmname{#1}\thmnumber{ #2}\thmnote{ (#3)}}%

\newtheoremstyle{ega-defn-env-style}%
{}{}{\rmfamily}{}{\bfseries}{. --- }{ }{\thmname{#1}\thmnumber{ #2}\thmnote{ (#3)}}%

\theoremstyle{ega-env-style}
\newtheorem*{env}{---}

\theoremstyle{ega-thm-env-style}
\newtheorem{theorem}{Theorem}[subsection]
\newtheorem{proposition}{Proposition}[subsection]
\newtheorem{lemma}{Lemma}[subsection]
\newtheorem{corollary}{Corollary}[subsection]
\newtheorem{stheorem}{Theorem}[section]
\newtheorem{slemma}[stheorem]{Lemma}
\newtheorem{skey}[stheorem]{Key Formula}
\newtheorem{conjecture}{Conjecture}[subsection]

\theoremstyle{ega-defn-env-style}
\newtheorem*{definition}{Definition}
\newtheorem{example}{Example}[subsection]
\newtheorem*{examples}{Examples}
\newtheorem*{remark}{Remark}
\newtheorem*{remarks}{Remarks}
\newtheorem*{notation}{Notation}
\newtheorem*{exercise}{Exercise}
\newtheorem*{properties}{Properties}
\newtheorem*{consequences}{Consequences}
\newtheorem*{note}{Note}
\newtheorem*{fact}{Fact}

\makeatletter
\def\l@subsection{\@tocline{2}{0pt}{2.5pc}{2.2pc}{}}
\def
\section{\@startsection{section}{1}%
	\z@{.7\linespacing\@plus\linespacing}{.5\linespacing}%
{\normalfont\bfseries\Large\scshape\centering}}
\renewcommand{\@seccntformat}[1]{%
	\ifnum\pdfstrcmp{#1}{section}=0\textsection\fi%
\csname the#1\endcsname.~}
\makeatother

\def\mathcal{\mathscr}
%% Fonts

\def\sh{\mathcal}                   % sheaf font
\def\bb{\mathbf}                    % bold font
\def\cat{\mathsf}                   % category font

%% Font Letters

\def\CL{\mathcal{L}}
\def\OO{\mathcal{O}}
\def\CX{\mathcal{X}}

%% Cohomology

\def\CH{\mathrm{H}}                 % cohomology H
\def\CHH{\check{\HH}}               % Čech cohomology H
\def\RD{\mathrm{R}}                 % right derived R
\def\LD{\mathrm{L}}                 % left derived L
\def\dual#1{{#1}^\vee}              % dual
\def\Tor{\operatorname{Tor}}        % Tor
\def\Ext{\operatorname{Ext}}        % Ext
\def\HdR{\mathrm{H}_{\mathrm{dR}}}  % de Rham cohomology
\def\Zc{\underline{\mathbb{Z}}}     % constant sheaf with integer coeffs

%% Categories

\def\A{\cat{A}}                     % category A, usually abelian
\def\C{\cat{C}}                     % category C
\def\op{^\cat{op}}                  % opposite category
\def\Set{\cat{Sets}}                % category of sets
\def\Grp{\cat{Gps}}                 % category of groups
\def\Alg{\cat{Alg}}                 % category of algebras
\def\QCoh{\cat{QCoh}}               % category of quasicoherent sheaves
\def\supertilde{{\,\widetilde{\,}\,}}   % use \supertilde instead of ^\sim
\def\Hom{\operatorname{Hom}}        % morphisms
\def\End{\operatorname{End}}        % endomorphisms
\def\Aut{\operatorname{Aut}}        % automorphisms
\def\ker{\operatorname{ker}}        % kernel
\def\img{\operatorname{im}}         % image
\def\coker{\operatorname{coker}}    % cokernel
\def\pr{\operatorname{pr}}          % projection
\def\EssIm{\operatorname{EssIm}}    % essential image
\DeclareMathOperator*{\colim}{colim}   % colimit


%% Schemes

\def\Proj{\operatorname{Proj}}      % Proj
\def\Supp{\operatorname{Supp}}      % support
\def\Spec{\operatorname{Spec}}      % Spec
\def\Spf{\operatorname{Spf}}        % formal Spec
\def\Aff{\mathbb A}                 % affine space
\def\P{\mathbb{P}}                  % projective space
\def\Pic{\operatorname{Pic}}        % Picard group

%% Standard Operators

\def\codim{\operatorname{codim}}    % codimension
\def\id{\operatorname{id}}          % identity

%% Arrows
\renewcommand{\to}{\mathchoice{\longrightarrow}{\rightarrow}{\rightarrow}{\rightarrow}}
\newcommand{\from}{\mathchoice{\longleftarrow}{\leftarrow}{\leftarrow}{\leftarrow}}
\let\mapstoo\mapsto
\renewcommand{\mapsto}{\mathchoice{\longmapsto}{\mapstoo}{\mapstoo}{\mapstoo}}
\def\isoto{\simeq}                  % isomorphism
\def\simto{\xrightarrow{\sim}}      % isomorphism arrow
\def\surjto{\twoheadrightarrow}     % surjetion
\def\injto{\hookrightarrow}         % injection

%% Under/Over Accents

\newcommand{\wh}[1]{\widehat{#1}}   % hat
\newcommand{\wt}[1]{\widetilde{#1}}    % tilde
\def\ul{\underline}                % underline

%% Spaces

\def\Z{\mathbb Z}                  % integers
\def\Q{\mathbb Q}                  % rationals
\def\N{\mathbb{N}}                 % naturals

%% Groups

\def\GL{\bb{GL}}                   % general linear group
\def\SL{\bb{SL}}                   % special linear group
\def\det{\operatorname{det}}       % determinant

%% Sub/Superscripts

\def\et{^\text{\'et}}              % \'etale
\def\an{^{\text{an}}}              % analytic

%% Misc

\newcommand{\defn}[1]{\textbf{#1}}  % definition highlighting
\def\defeq{:=}                     % definition equation
\def\eps{\varepsilon}              % correct epsilon
\newcommand{\gen}[1]{\left\langle\!\left\langle #1 \right\rangle\!\right\rangle}



\def\shHom{\sh{H}\textup{\kern-2.2pt{\Fontauri\slshape om}}}   % sheaf Hom
\def\shProj{\sh{P}\textup{\kern-2.2pt{\Fontauri\slshape roj}}} % sheaf Proj
\def\shExt{\sh{E}\textup{\kern-2.2pt{\Fontauri\slshape xt}}}   % sheaf Ext
\def\shGr{\sh{G}\textup{\kern-2.2pt{\Fontauri\slshape r}}}     % sheaf Gr
\def\shDer{\sh{D}\,\textup{\kern-2.2pt{\Fontauri\slshape er}}} % sheaf Der
\def\shDiff{\sh{D}\,\textup{\kern-2.2pt{\Fontauri\slshape if{}f}}\,} % sheaf Diff
\def\shHomcont{\sh{H}\textup{\kern-2.2pt{\Fontauri\slshape om.\,cont}}}   % sheaf Hom.cont
\def\shAut{\sh{A}\textup{\kern-2.2pt{\Fontauri\slshape ut}}}   % sheaf Aut
\def\shSym{\sh{S}\textup{\kern-2.2pt{\Fontauri\slshape ym}}}   % sheaf Sym

\makeatletter
\newcommand{\cbigoplus}{\DOTSB\cbigoplus@\slimits@}
\newcommand{\cbigoplus@}{\mathop{\widehat{\bigoplus}}}
\makeatother


\usepackage{stmaryrd}
\usepackage[nameinlink]{cleveref}

\def\i{\operatorname{inv}}
\def\Gm{\mathbb{G}_m}
\def\Ga{\mathbb{G}_a}
\def\Diag{\operatorname{Diag}}
\def\Rep{\operatorname{Rep}}
\def\Repf{\operatorname{Rep}^{\text{fd}}}
\def\Res{\operatorname{Res}}
\def\Ind{\operatorname{Ind}}

\title{Representations of Reductive Algebraic Groups}
\author{Lecturer: Simon Riche, Typesetter: Micha{\l} Mruga{\l}a}
\begin{document}
\maketitle
\tableofcontents

\newpage

\section{Generalities on affine group schemes and smooth representations}
\subsection{Affine group schemes}
Fix $k$ a base field.

Recall that an \defn{affine $k$-group scheme} is one of the following data:
\begin{enumerate}[(1)]
	\item An affine scheme $G$ over $k$ endowed with morphisms of $k$-schemes
		\begin{itemize}
			\item $m:G\times G\to G$; 
			\item $e:\Spec(k)\to G$ ;
			\item $\i:G\to G$;
		\end{itemize}
		which satisfy the usual axioms of groups (with $m$ multiplication, $e$ the unit, $\i$ the inverse).
	\item A functor $\Alg_k = \{k\text{-algebras}\} \xrightarrow{F} \Grp$ such that the composition $\Alg_k \xrightarrow{F}\Grp\to \Set$ is representable.
	\item A commutative Hopf algebra over $k$, i.e. a commutative algebra $A$ with morphisms of $k$-algebras
		\begin{itemize}
			\item $\Delta:A \to A\otimes A$ ;
			\item $\eps:A\to k$ ;
			\item $S:A\to A$ ;
		\end{itemize}
		such that
		\begin{align*}
			(\id \otimes\Delta) \circ \Delta = (\Delta \otimes \id) &: A \to A\otimes A \otimes A \\
			(\id\otimes \eps) \circ \Delta = \id = (\eps\otimes\id)\circ\Delta &: A \to A \\
			(\id,S)\circ \Delta = \Delta \circ \eps = (S,\id)\circ\Delta &:A \to A
		\end{align*}
\end{enumerate}
\begin{notation}
\begin{align*}
	A &\to \Spec(A) \\
	G &\to \OO(G), \Delta_G.
\end{align*}
\end{notation}
\begin{example} \leavevmode
	\begin{enumerate}[(1)]
		\item \emph{Diagonalizable groups:} If $\Lambda$ is an abstract commutative group we have the affine $k$-group scheme $\Diag(\Lambda)\defeq\Spec(k[\Lambda])$ with
			\[
			\Delta(\lambda) = \lambda \otimes \lambda,\quad \eps(\lambda)=1,\quad S(\lambda)=\lambda^{-1}\quad (\forall \lambda\in \Lambda).
			\] 
			In particular for $\Lambda=\Z$, $k[\Lambda]=k[x,x^{-1}]$ and $\Diag(\Lambda)=\Gm$ (the \defn{multiplicative group}).

			A \defn{(split) torus} is a group scheme of the form $\Diag(\Lambda)$ with $\Lambda$ a finitely generated free abelian group.
		\item \emph{Additive group:} $\Ga\defeq\Spec(k[x])$ with
			\[
			\Delta(x) = x\otimes 1 + 1\otimes x,\quad \eps(x)=0,\quad S(x)=-x.
			\]
			More generally for $V$ a $k$-vector space we have the functor $V_{a}:R\mapsto (R\otimes V,-)$ which is an affine $k$-group scheme if $V$ is finite dimensional.
		\item If $V$ is a $k$-vector space, $\GL(V)$ is the functor $R\mapsto \Aut_R(R\otimes V)$. If $V$ is finite dimensional this is an affine $k$-group scheme.

			In particular, if $V=k^{n}$ we get 
			\[
				\GL_n = \Spec(k[x_{ij},1\le i,j\le n][\det^{-1}])
			\] 
			with
			\[
			\Delta(x_{ij}) = \sum_l x_{il}\otimes x_{lj},\quad \eps(x_{ij}) = \delta_{ij}
			.\] 
			Similarly we have $\SL(V), \SL_n$.
		\item For any abstract group $\Gamma$ we have the functor (??)
	\end{enumerate}
\end{example}

\subsection{Representations}

If $G$ is an affine $k$-group scheme, a \defn{representation} of $G$ is the datum of a $k$-vector space $V$ and a morphism of group valued functors $G\to \GL(V)$. [Equivalently, an action of $G$ on $V_{a}$ such that $G(R)$ acts $R$-linearly on $R\otimes V$.]

This datum is equivalent ot that of a \defn{comodule} for $\OO(G)$, i.e. a $k$-vector space $V$ and a $k$-linear map $\Delta_V:V\to V\otimes \OO(G)$ such that
\begin{align*}
	(\Delta_V\otimes\id_{\OO(G)}\circ \Delta_V = (\id_V\otimes \Delta_G)\circ \Delta_V &:V\to V\otimes\OO(G)\otimes\OO(G) \\
	(\id_V\otimes \eps)\otimes \Delta_V = \id_V &:V\to V.
\end{align*}
[$\Delta_V$ corresponds to the image of $\id_{\OO(G)}\in G(\OO(G))=\End_{k\text{-alg}}(\OO(G))$ in $\End_{\OO(G)}(\OO(G)\otimes V)$.]
\begin{example} \leavevmode
	\begin{enumerate}[(1)]
		\item \emph{(Right) Regular representation:} $V=\OO(G)$ with $\Delta_V=\Delta_G$. More generally, given an action of $G$ on an affine scheme $X$ we get a representation with underlying vector space $\OO(X)$.
		\item If $V$ is a finite dimensional vector space, $V$ is a representation of $\GL(V)$.
		\item For any $G$ we have the trivial representation $k$.
	\end{enumerate}
\end{example}
\begin{notation}
	$\Rep(G)$ is the abelian category of representations of $G$. $\Repf(G)$ is the full subcategory of finite dimensional representations.
\end{notation}

If $V\in \Rep(G)$ then $V$ is the union of its finite dimensional subrepresentations.

\begin{example}[Representations of diagonalizable group schemes]
	Let $\Lambda$ be a commutative group, $G=\Diag(\Lambda)$. If $V\in \Rep(G)$ we have
	\[
	\Lambda_V:V\to V\otimes\OO(G) = \bigoplus_{\lambda\in \Lambda}V\otimes\lambda
	.\] 
	Hence there are morphisms ($\rho_\lambda:\lambda\in \Lambda$) in $\End(V)$ such that
	\[
	\Delta_V(v) = \sum_{\lambda\in \Lambda} \rho_\lambda(v)\otimes\lambda,\qquad \forall v\in V.
	\] 
	(Here $\rho_\lambda(v)=0$ for all but finitely many $\lambda$s.)

	It is easy to see that
	\[
	\rho_\lambda\circ\rho_\mu = \begin{cases}
		\rho_\lambda & \text{if }\lambda=\mu \\
		0 & \text{otherwise}
	\end{cases}
	\]
	and $\id = \sum_{\lambda_\Lambda}\rho_\lambda$.

	Hence $V=\bigoplus_{\lambda\in \Lambda}\rho_\lambda(V)$ with
	\[
	\rho_\lambda(V) = \{v\in V : \Delta_V(v) = v\otimes \lambda\} = V_\lambda
	.\] 
	Hence $\Rep(G)$ is isomorphic to the category of $\Lambda$-graded vector spaces (correct?).
\end{example}
\subsection{Induction} Let $G$ be an affine $k$-group scheme.

A \defn{subgroup} of $G$ is a closed subscheme $H\subset G$ such that $e,\i|_{H},m|_{H\times H}$ factor through $H$. Then $H$ is an affine $k$-group scheme. In this setting we have the restriction functor $\Res_{H}^{G}:\Rep(G)\to \Rep(H)$.

\begin{proposition}
	The functor $\Res_{H}^{G}$ has a right adjoint $\Ind_{H}^{G}:\Rep(H)\to \Rep(G)$.
\end{proposition}
Explicitly, we have
\[
\Ind_{H}^{G}(V) = (V\otimes\OO(G))^H
\] 
with $H$ acting diagonally via the right-regular representation on $\OO(G)$ and $G$ acting on the fixed points via the left regular representation
\[
	\Ind_{H}^{G}(V) = \left\{ \begin{array}{c} 
		\text{morphisms of functors} \\ 
	f :G\to V_{a}
	\end{array} \middle| \begin{array}{c} 
		f(gh)=h^{-1}f(g) \\
		\forall g\in G(R), h\in H(R), R\in \cat{Alg}_k 
	\end{array} \right\} 
.\] 
The canonical isomorphism
\[
\Hom_{\Rep(G)}(V,\Ind_{H}^{G}(V')) \simeq \Hom_{\Rep(H)}(V,V')
\] 
is called \defn{Frobenius reciprocity}.

\begin{properties} \leavevmode
\begin{itemize}
	\item \emph{Transitivity:} Given subgroups $H_1\subset H_2\subset G$ we have
		\[
		\Ind_{H_1}^{G} \simeq \Ind_{H_2}^{G}\circ \Ind_{H_1}^{H_2}.
		\] 
	\item \emph{Tensor identity:} For $V_1\in \Rep(H),V_2\in \Rep(G)$ 
		\[
		\Ind_{H}^{G}(V_1\otimes\Res_{H}^{G}(V_2)) \simeq \Ind_{H}^{G}(V_1) \otimes V_2.
		\] 
	\item $\Ind_H^{G}$ sends injective objects of $\Rep(H)$ to injective objects of $\Rep(G)$. In particular, 
		\[
			\Ind_H^{G}(k) = \OO(G)
		\]
		is injective.
	\item $\Rep(G)$ has enough injectives.
\end{itemize}
\end{properties}

\emph{Geometric interpretation:} We assume $G$ is an algebraic group (over $k$), i.e. an affine $k$-group scheme such that $\OO(G)$ is a finitely generated $k$-algebra. In this setting, for $H\subset G$ a subgroup we have a quotient scheme $G /H$ of finite type over $k$ with a faithfully flat quotient map $\pi:G\to G /H$. For $V\in \Rep(H)$, we have a quasicoherent sheaf $\CL_{G /H}(V)\in \QCoh(G /H)$ with
\[
\Gamma(V,\CL_{G /H}(V)) = \left\{ \text{morphisms } f:\pi^{-1}(V)\to V \middle| f(x,h)=h^{-1}f(x)\text{ for all }(?)  \right\} 
.\] 
We have $\Ind_{H}^{G}(V) = \Gamma(G /H,\CL_{G /H}(V))$. If $V$ is finite dimensional, then $\CL_{G /H}(V)$ is coherent.

\begin{consequences} \leavevmode
\begin{itemize}
	\item If $G /H$ is affine then $\Ind_{H}^{G}$ is exact.
	\item If $G /H$ is projective then $\Ind_{H}^{G}$ preserves finite dimensionality.
\end{itemize}
\end{consequences}

Since $\Rep(H)$ has enough injectives we can consider the derived functor
\[
	\RD\Ind_H^{G}:D^{b}\Rep(H)\to D^{b}\Rep(G).
\]
The functor $\CL_{G /H}:\Rep(H)\to \cat{QCoh}(G /H)$ is exact, hence we have 
\[
	\CL_{G /H}:D^{b}\Rep(H)\to D^{b}\cat{QCoh}(G /H).
\]
One can check that
\[
\RD\Ind_H^{G}(V) \simeq R\Gamma(G /H, \CL_{G /H}(V)).
\] 
(??)

\begin{consequences} \leavevmode
\begin{itemize}
	\item We have $\RD^{n}\Ind_H^{G}(V)=0$ for all $V\in \Rep(H)$ if $n>\dim(G /H)$.
	\item If $G /H$ is projective, then $\RD^{n}\Ind_H^{G}(V)$ is finite-dimensional for all $V\in \Repf(H), n\in \Z$.
\end{itemize}
\end{consequences}

\newpage

\section{Reductive algebraic groups}
From now on $k$ is algebraically closed.

\subsection{Definition}
A $k$-algebraic group $G$ is called \defn{unipotent} if every non-zero representation admits a non-zero fixed vector. [Equivalent condition: $G$ is unipotent if and only if it is isomorphic to a subgroup of unipotent upper-triangular matrices in $\GL_n$ for some $n$.]
\begin{example}
	$\Ga$ is unipotent as
	\[
		\Ga \simeq \begin{pmatrix} 1 & * \\ 0 & 1 \end{pmatrix} .
	\] 
\end{example}
If $G$ is a smooth, connected algebraic group, the smooth, connected, unipotent, normal subgroups of $G$ there is a largest element called the \defn{unipotent radical} of $G$, denoted $R_u(G)$. An algebraic group $G$ is called \defn{reductive} if it is smooth, connected and $R_u(G)$ is trivial. 

One possible motivation for studying representations of reductive algebraic groups is that any simple representation of a smooth connected algebraic group $G$ factors through a simple representation of $G /R_u(G)$, which is a reductive algebraic group.

\begin{example} \leavevmode
	\begin{enumerate}[(1)]
		\item \emph{Tori:} If $\Lambda$ is a finitely generated, free abelian group, then $\Diag(\Lambda)$ is a reductive algebraic group.
		\item For any finite-dimensional $k$-vector space $V$, $\GL(V)$ and $\SL(V)$ are reductive algebraic groups.
		\item Symplectic groups, special orthogonal groups.
	\end{enumerate}
\end{example}
\subsection{Structure} From now on $G$ is a redutive algebraic group.

We denote by $B$ a \defn{Borel subgroup} (a maximal, connected, smooth, solvable subgroup). Note that:
\begin{itemize}
	\item a Borel subgroup is unique up to conjugation;
	\item the quotient $G /B$ is a smooth, projective variety.
\end{itemize}
\begin{example}
	 For $G=\GL_{n,k}$ one can take
	 \[
		 B = \left\{ \begin{pmatrix} * & & 0 \\ & \ddots & \\ * & & *\end{pmatrix}  \right\} .
	 \] 
	 In this case $G /B$ parametrizes flags in $k^{n}$, i.e. data
	 \[
	  \{0\} \subset V_1\subset \dots\subset V_{n-1}\subset k^{n}
	 \]
	 with $V_i$ a subspace of dimension $i$.
\end{example}

\end{document}
