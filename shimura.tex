\documentclass[oneside]{amsart}

\usepackage[all]{xy}
\usepackage{tikz-cd}
\usepackage[T1]{fontenc}
\usepackage{xstring}
\usepackage{xparse}
\usepackage{xr-hyper}
\usepackage{xcolor}
\definecolor{brightmaroon}{rgb}{0.76, 0.13, 0.28}
\usepackage[linktocpage=true,colorlinks=true,hyperindex,citecolor=blue,linkcolor=brightmaroon]{hyperref}
\usepackage[nameinlink]{cleveref}
\usepackage[left=1.25in,right=1.25in,top=0.75in,bottom=0.75in]{geometry}
%\usepackage[charter,greekfamily=didot]{mathdesign}
%\usepackage{Baskervaldx}
\usepackage{amssymb}
\usepackage{stmaryrd}
\usepackage{mathrsfs}
\usepackage{mathpazo}
\linespread{1.05}

\usepackage[nobottomtitles]{titlesec}
\usepackage{marginnote}
\usepackage{enumerate}
\usepackage{longtable}
\usepackage{aurical}
\usepackage{microtype}

\newtheoremstyle{ega-env-style}%
  {}{}{\rmfamily}{}{\bfseries}{.}{ }{\thmnote{(#3)}}%

\newtheoremstyle{ega-thm-env-style}%
  {}{}{\itshape}{}{\bfseries}{. --- }{ }{\thmname{#1}\thmnumber{ #2}\thmnote{ (#3)}}%

\newtheoremstyle{ega-defn-env-style}%
  {}{}{\rmfamily}{}{\bfseries}{. --- }{ }{\thmname{#1}\thmnumber{ #2}\thmnote{ (#3)}}%

\theoremstyle{ega-env-style}
\newtheorem*{env}{---}

\theoremstyle{ega-thm-env-style}
\newtheorem{theorem}{Theorem}[subsection]
\newtheorem{proposition}{Proposition}[subsection]
\newtheorem{lemma}{Lemma}[subsection]
\newtheorem{corollary}{Corollary}[subsection]
\newtheorem{stheorem}{Theorem}[section]
\newtheorem{slemma}[stheorem]{Lemma}
\newtheorem{skey}[stheorem]{Key Formula}

\theoremstyle{ega-defn-env-style}
\newtheorem*{definition}{Definition}
\newtheorem{example}{Example}[subsection]
\newtheorem*{examples}{Examples}
\newtheorem*{remark}{Remark}
\newtheorem*{remarks}{Remarks}
\newtheorem*{notation}{Notation}
\newtheorem*{exercise}{Exercise}
\newtheorem*{properties}{Properties}
\newtheorem*{consequences}{Consequences}
\newtheorem*{note}{Note}
\newtheorem*{fact}{Fact}

\makeatletter
\def\l@subsection{\@tocline{2}{0pt}{2.5pc}{2.2pc}{}}
\def\section{\@startsection{section}{1}%
  \z@{.7\linespacing\@plus\linespacing}{.5\linespacing}%
  {\normalfont\bfseries\Large\scshape\centering}}
\renewcommand{\@seccntformat}[1]{%
  \ifnum\pdfstrcmp{#1}{section}=0\textsection\fi%
  \csname the#1\endcsname.~}
\makeatother

\def\mathcal{\mathscr}
%% Fonts

\def\sh{\mathcal}                   % sheaf font
\def\bb{\mathbf}                    % bold font
\def\cat{\mathsf}                   % category font

%% Font Letters

\def\CA{\mathcal{A}}
\def\CE{\mathcal{E}}
\def\CF{\mathcal{F}}
\def\CG{\mathcal{G}}
\def\CL{\mathcal{L}}
\def\OO{\mathcal{O}}
\def\CX{\mathcal{X}}
\def\b1{\mathbb{1}}

%% Cohomology

\def\CH{\mathrm{H}}                 % cohomology H
\def\CHH{\check{\HH}}               % Čech cohomology H
\def\RD{\mathrm{R}}                 % right derived R
\def\LD{\mathrm{L}}                 % left derived L
\def\dual#1{{#1}^\vee}              % dual
\def\Tor{\operatorname{Tor}}        % Tor
\def\Ext{\operatorname{Ext}}        % Ext
\def\HdR{\mathrm{H}_{\mathrm{dR}}}  % de Rham cohomology
\def\Zc{\underline{\mathbb{Z}}}     % constant sheaf with integer coeffs

%% Categories

\def\A{\cat{A}}                     % category A, usually abelian
\def\C{\cat{C}}                     % category C
\def\op{^\cat{op}}                  % opposite category
\def\Set{\cat{Sets}}                % category of sets
\def\Grp{\cat{Gps}}                 % category of groups
\def\Alg{\cat{Alg}}                 % category of algebras
\def\QCoh{\cat{QCoh}}               % category of quasicoherent sheaves
\def\supertilde{{\,\widetilde{\,}\,}}   % use \supertilde instead of ^\sim
\def\Map{\operatorname{Map}}        % morphisms (any category)
\def\Hom{\operatorname{Hom}}        % morphisms (additive category)
\def\iHom{\underline{\Hom}}         % interal hom
\def\End{\operatorname{End}}        % endomorphisms
\def\Aut{\operatorname{Aut}}        % automorphisms
\def\ker{\operatorname{ker}}        % kernel
\def\img{\operatorname{im}}         % image
\def\coker{\operatorname{coker}}    % cokernel
\def\pr{\operatorname{pr}}          % projection
\def\EssIm{\operatorname{EssIm}}    % essential image
\DeclareMathOperator*{\colim}{colim}   % colimit

\def\DAet{\cat{DA}\et}              
\def\SH{\cat{SH}}                   
\def\Sh{\cat{Sh}}                   % category of sheaves
\def\PSh{\cat{PSh}}                 % category of presheaves

%% Schemes

\def\Proj{\operatorname{Proj}}      % Proj
\def\Supp{\operatorname{Supp}}      % support
\def\Spec{\operatorname{Spec}}      % Spec
\def\Spf{\operatorname{Spf}}        % formal Spec
\def\Aff{\mathbb A}                 % affine space
\def\P{\mathbb{P}}                  % projective space
\def\Pic{\operatorname{Pic}}        % Picard group

%% Standard Operators

\def\codim{\operatorname{codim}}    % codimension
\def\id{\operatorname{id}}          % identity

%% Arrows
\renewcommand{\to}{\mathchoice{\longrightarrow}{\rightarrow}{\rightarrow}{\rightarrow}}
\newcommand{\from}{\mathchoice{\longleftarrow}{\leftarrow}{\leftarrow}{\leftarrow}}
\let\mapstoo\mapsto
\renewcommand{\mapsto}{\mathchoice{\longmapsto}{\mapstoo}{\mapstoo}{\mapstoo}}
\def\isoto{\simeq}                  % isomorphism
\def\simto{\xrightarrow{\sim}}      % isomorphism arrow
\def\surjto{\twoheadrightarrow}     % surjetion
\def\injto{\hookrightarrow}         % injection

%% Under/Over Accents

\newcommand{\wh}[1]{\widehat{#1}}   % hat
\newcommand{\wt}[1]{\widetilde{#1}}    % tilde
\def\ul{\underline}                % underline

%% Spaces

\def\Z{\mathbb Z}                  % integers
\def\Q{\mathbb Q}                  % rationals
\def\N{\mathbb N}                 % naturals
\def\H{\mathcal H}

%% Groups

\def\GL{\bb{GL}}                   % general linear group
\def\SL{\bb{SL}}                   % special linear group
\def\Sp{\bb{Sp}}
\def\det{\operatorname{det}}       % determinant

%% Sub/Superscripts

\def\et{^\text{\'et}}              % \'etale
\def\an{^{\text{an}}}              % analytic

%% Misc

\newcommand{\defn}[1]{\textbf{#1}}  % definition highlighting
\def\defeq{:=}                     % definition equation
\def\eps{\varepsilon}              % correct epsilon
\newcommand{\gen}[1]{\left\langle\!\left\langle #1 \right\rangle\!\right\rangle}



\def\shHom{\sh{H}\textup{\kern-2.2pt{\Fontauri\slshape om}}}   % sheaf Hom
\def\shProj{\sh{P}\textup{\kern-2.2pt{\Fontauri\slshape roj}}} % sheaf Proj
\def\shExt{\sh{E}\textup{\kern-2.2pt{\Fontauri\slshape xt}}}   % sheaf Ext
\def\shGr{\sh{G}\textup{\kern-2.2pt{\Fontauri\slshape r}}}     % sheaf Gr
\def\shDer{\sh{D}\,\textup{\kern-2.2pt{\Fontauri\slshape er}}} % sheaf Der
\def\shDiff{\sh{D}\,\textup{\kern-2.2pt{\Fontauri\slshape if{}f}}\,} % sheaf Diff
\def\shHomcont{\sh{H}\textup{\kern-2.2pt{\Fontauri\slshape om.\,cont}}}   % sheaf Hom.cont
\def\shAut{\sh{A}\textup{\kern-2.2pt{\Fontauri\slshape ut}}}   % sheaf Aut
\def\shSym{\sh{S}\textup{\kern-2.2pt{\Fontauri\slshape ym}}}   % sheaf Sym

\makeatletter
\newcommand{\cbigoplus}{\DOTSB\cbigoplus@\slimits@}
\newcommand{\cbigoplus@}{\mathop{\widehat{\bigoplus}}}
\makeatother


\usepackage{stmaryrd}
\usepackage[nameinlink]{cleveref}

\newcommand{\SC}{\mathcal{C}}
\newcommand{\OO}{\mathcal{O}}
\def\coker{\operatorname{coker}}
\def\bL{\mathbf{L}}
\def\CT{\mathcal{T}}
\def\Aut{\operatorname{Aut}}
\def\wt{\widetilde}
\def\ko{\mathfrak{o}}
\def\kt{\mathfrak{t}}
\newcommand{\CX}{\mathcal{X}}
\def\coker{\operatorname{coker}}
\def\bL{\mathbf{L}}
\def\wh{\widehat}
\def\CI{\mathcal{I}}
\def\et{^\text{\'et}}
\def\sep{^\text{sep}}
\def\pr{\operatorname{pr}}
\def\img{\operatorname{im}}
\def\Sym{\operatorname{Sym}}
\def\CG{\mathcal{G}}
\def\N{\mathbb{N}}
\def\th{^{\mathrm{th}}}
\def\CN{\mathcal{N}}
\def\Inf{\operatorname{Inf}}
\def\rank{\operatorname{rank}}
\def\Pic{\operatorname{Pic}}
\def\i{\operatorname{inv}}
\def\Diag{\operatorname{Diag}}
\def\G{\mathbb{G}}
\def\End{\operatorname{End}}
\def\det{\operatorname{det}}
\def\SL{\bb{SL}}
\def\Rep{\operatorname{Rep}}
\def\Repf{\operatorname{Rep}^{\text{fd}}}
\def\Res{\operatorname{Res}}
\def\Ind{\operatorname{Ind}}
\def\CL{\mathcal{L}}
\def\Zc{\underline{\mathbb{Z}}}
\def\DAet{\operatorname{DA}^{\text{\'et}}}
\def\SH{\operatorname{SH}}
\def\Sh{\operatorname{Sh}}
\def\PSh{\operatorname{PSh}}
\def\Cone{\operatorname{Cone}}
\def\A{\cat{A}}
\def\Ch{\operatorname{Ch}}
\def\EssIm{\operatorname{EssIm}}
\def\colim{\operatorname{colim}}
\def\ft{^{\text{(f.t)}}}
\def\an{^{\text{an}}}
\def\P{\mathbb{P}}
\def\CA{\mathcal{A}}
\def\Mat{\operatorname{Mat}}
\def\H{\mathcal{H}}
\def\Sp{\bb{Sp}}
\def\Stab{\operatorname{Stab}}
\def\CE{\mathcal{E}}

\title{Introduction to Shimura Varieties and Their Cohomology}
\author{Micha{\l} Mruga{\l}a}

\begin{document}
\maketitle
\emph{Plan:}
\begin{enumerate}[I)]
	\item Siegel modular varieties
	\item General Shimura varieties
	\item (\'Etale) Cohomology: Kottwitz conjecture
\end{enumerate}
(The cohomology gives a realization of the global Langlands correspondence for certain number fields.)

The main reference are Sophie Morel's notes from the IHES '22 summer school.

\section{Siegel Modular Varieties}
Analytically: we want to parametrize complex tori. This is fairly simple. Let $V$ be a $\bb{C}$-vector space of dimension $m\ge 1$, $\Lambda\subset V$ a lattice (a discrete subgroup such that $V /\Lambda$ is compact), then $X=V /\Lambda$ is a complex Lie group, which is a complex torus. 

\begin{exercise}
	A morphism $f:X=V /\Lambda \to X' = V' /\Lambda'$ of complex Lie groups is given by a $\bb{C}$-linear map $V\to V'$ mapping $\Lambda$ to $\Lambda'$.
\end{exercise}

\emph{Question:} Which complex tori are algebraizable, i.e. $X\injto \P^{n}(\bb{C})$ (equivalent to $X\simeq \underline{X}\an$ for some projective $\underline{X}$ by Chow). Can we find a parametrization?

\begin{example}
	Let $n=1$ complex tori are always algebraic. There is the Weierstrass $\wp$-function
	\begin{align*}
		\wp: V /\Lambda &\longrightarrow \P^{1}(\bb{C}) \\
		v &\longmapsto \wp(v) = \frac{1}{v^2} + \sum_{\lambda\in \Lambda-0}\frac{1}{(\lambda+v)^2}-\frac{1}{\lambda^2}
	.\end{align*}
	It embeds $V /\Lambda$ in $\P^2(\mathbb{C})$ via $[\wp:\wp':1]$ with image $y^2=P_\Lambda(x)$ where $P_\Lambda\in \bb{C}[X]$ has degree 3. The coefficients of $P_\Lambda$ are Eisenstein series (modular forms).

	For $n>1$, $X$ is ``almost never'' algebraic.
\end{example}
Recall that $X$ is algebraizable if and only if there exists $\CL\in \Pic(X)$ which is ample (see Mumford's Abelian Varieties). Recall that $\Pic(X)\simeq H^{1}(X,\OO_X^{\times})$. There is a short exact sequence
\begin{equation}
\begin{tikzcd}
	0 \ar[r] & \Zc_X \ar[r] & \OO_X \ar[r,"\exp(2\pi i-)"] & \OO_X^{\times} \ar[r] & 0
\end{tikzcd}
\end{equation}
so we get a long exact sequence. The map
\[
	H^{0}(X,\OO_X) \simeq \bb{C} \to \bb{C}^{\times}\simeq H^{0}(X,\OO_X^{\times})
\]
is surjective so we get
\begin{equation}
\begin{tikzcd}
	H^{1}(X,\Z) \ar[r,hook] \ar[d,"\simeq"] & H^{1}(X,\OO_X) \ar[d,"\simeq"',"Dolbeaut"] \ar[r] & H^{1}(X,\OO_X) \ar[d,"\simeq"] \ar[r,"\delta"] & \ker (H^2(X,\Z)\to H^2(X,\OO_X)) \ar[d,"\simeq"] \\
	H^{1}(\Lambda,\Z) \ar[d,equal] & \overline{T} & H^{1}(\Lambda,\OO(X)^{\times}) & \Hom\left( \bigwedge^2\Lambda,\Z \right)  \\
	\Hom(\Lambda,\Z) \ar[dr,hook] & T \oplus \overline{T} \ar[u,"\pr_2"] \\
				      & \Hom_{\bb{R}}(V,\bb{C}) \ar[u,"\simeq"]
\end{tikzcd}
\end{equation}
We have $H^{i}(V,\Z)=0$ for all $i>0$ and $H^{i}(V,\OO_V)=0$ for all $i>0$ so $\Pic(V)=0$. $\overline{T}$ are the antilinear maps $V\to \bb{C}$ and $T=\Hom_{\bb{C}}(V,\bb{C})$. Observe that
\[
\Pic^{0}(X) = \ker\delta \simeq \frac{\overline{T}}{\pr_2(\Hom(\Lambda,\Z))}
\]
is also a complex torus. The last term is the Neron-Severi group
\begin{align*}
	NS(X) &\simeq \{E:\Lambda\times \Lambda\to \Z\text{ alt.}: E(iv_1,iv_2)=E(v_1,v_2), \forall v_1,v_2\in V\} \\
	      &= \left\{ \Im H:\, H:V\times V\to \bb{C}\text{ Hermitian such that }(\Im H)(\Lambda\times \Lambda)\subset \Z \right\} 
.\end{align*}
The Appel-Humbert theorem completely describes $\Pic(X)$ as $\{L(H,\alpha)\} $ with $H$ as above and $\alpha$ an extra datum.

\begin{theorem}[Lefschetz]
	The following are equivalent:
	\begin{enumerate}[1)]
		\item $H$ is positive definite.
		\item $L(H,\alpha)$ is ample (in fact, $L(H,\alpha)^{\otimes 3}$ is enough to embed $X$).
	\end{enumerate}
\end{theorem}
Let $L\in \Pic(X)$ then
\begin{align*}
	\phi_L: X &\longrightarrow \Pic^{0}(X) = \wh{X} \\
	x &\longmapsto T_x^{*}L\otimes L^{-1}
\end{align*}
is a morphism of Lie groups (here $T_x$ is translation by $x$).

\begin{theorem}
	The following are equivalent:
	\begin{itemize}
		\item $L$ is ample.
		\item $\ker\phi_L$ is finite.
		\item $\phi_L$ is surjective (i.e. an isogeny).
	\end{itemize}
\end{theorem}
\begin{exercise}
	Check that $phi_L$ is an isomorphism if and only if $E(\cdot ,\cdot )$ is perfect ($\Lambda\simeq \Hom(\Lambda,\Z)$).
\end{exercise}
\begin{definition}
	Say that such $\phi_L$ is a \emph{polarization}. If $\phi_L$ is an isomorphism, then it is called a \emph{principal polarization}.
\end{definition}
\begin{remark}
	Not every algebraic $X$ admits a principal polarization, but is isogenous to one that does.
\end{remark}
We can define the moduli problem
\[
	\CA_n(\bb{C}) = \left\{ (X,\phi):X = V /\Lambda\text{ of dimension }n,  \phi:X\to \wh{X}\text{ a principal polarization} \right\} 
.\] 
Let $(V,\Lambda,H)$ be a principally polarized complex torus. Choose a symplectic basis $(e_1,\dots,e_{2n})$ of $\Lambda$, i.e.
\[
	(E(e_i,e_j))_{i,j} = J_n = \begin{pmatrix} 0 & I_n \\ -I_n & 0 \end{pmatrix} 
.\] 
\begin{exercise}
	$L=L(H,\alpha)$ is ample if and only if $e_{n+1},\dots,e_{2n}$ is a basis of $V$ over $\bb{C}$ such that
	\[
	\tau = \Mat_{e_{n+1},\dots,e_{2n}}(e_1,\dots,e_n)
	\] 
	satisfies $\tau = ^{t}\tau$ and $\Im(\tau)$ is positive definite.
\end{exercise}
\begin{definition}
	$\H_n^{+}$ is the set of such $\tau\in M_n(\bb{C})$. There is an algebraic group
	\[
	\Sp_{2n,\Z}:R \mapsto \left\{ g\in M_{2n}(R) : ^{t}gJ_ng = K_n \right\} 
	.\] 
\end{definition}
There is an action of $\Sp_{2n}(\Z)$ on $\H_n^{+}$ such that given
\[
	\gamma = \begin{pmatrix} a & b \\ c & d \end{pmatrix} 
\] 
we have
\[
\tau \gamma = (\tau b + d)^{-1}(\tau a + c)
\]
(this corresponds to replacing $\underline{e}=(e_1,\dots,e_{2n})$ by $\underline{e}\gamma$).

We prefer left actions: let $^{t}\gamma$ act so that $\gamma\tau = \tau*^{t}\gamma$, i.e.
\[
	(\tau^{t}c + d)^{-1}(\tau^{t}a+^{t}b) = (a\tau + b)(c\tau + d)
.\] 
This extends to an action of $\Sp_{2n}(\bb{R})$ on $\H_n^{+}$. This action is transitive and
\begin{align*}
	\Stab_{\Sp_{2n}(\bb{R})}(iI_n) &\longrightarrow U(n) \\
	\begin{pmatrix} a & b \\ -b & a \end{pmatrix}  &\longmapsto a + ib
\end{align*}
is an isomorphism (this is a maximal compact subgroup).

So $\CA_n(\bb{C})\simeq \Gamma_n\setminus \H_n^{+}$ wher $\Gamma_n\simeq \Sp_{2n}(\Z)$.
\begin{remark}
	There exists $\gamma\in \Gamma_n\setminus \{\pm 1\} $and $\tau\in \H_n^{+}$ such that $\gamma\tau=\tau$.
\end{remark}

There is a universal object
\[
	\CX_n(\bb{C}) = \Z^{2n}\rtimes \Gamma_n\setminus\bb{C}^{n}\times \H_n^{+}
\] 
where
\[
	\gamma(v,\tau) = \begin{pmatrix} a & b \\ c & d \end{pmatrix} (v,\tau) = ((\tau^{t}c+^{t}d)^{-1}v, \gamma\tau)
\] 
and
\[
	(\lambda_1,\lambda_2)(v,\tau) = (v+\lambda_1+\tau\lambda_2,\tau)
\]
for $\lambda_i\in \Z^{n}$.

There is a morphism $\pi:\CX(\bb{C})\to \CA_n(\bb{C})$ which admits a section $e$. The fiber of $\tau$ is $[\tau]\simeq\bb{C}^{n} /\Lambda_\tau$ where $\Lambda_\tau = \Z^{n}\oplus\tau \Z^{n}$. We get the \emph{Hodge bundle}: take $\Omega^{1}(V /\Lambda)$ which are translaton invariant 1-forms, which is isomorphic to $V^{*}$ via $e^{*}$, then the Hodge bundle is
\[
	\CE_n = e^{*}\Omega^{1}_{\CX(\bb{C}) / \CA_n(\bb{C})} \simeq \Gamma_n\setminus\b{C}^{n}\times \H_n^{+}
\]
for action (??).

We can apply Schur functors (use the action of $\mathfrak{S}_k$ on on $\CE_n^{\otimes k}$ to act on subbundles, e.g. $\bigwedge^{k}\CE_n$ for $0\le k\le n$ ). (Equivalently see $\CE_n$ as a $\GL_n(\bb{C})$-bundle on $\CA_n(\bb{C})$ and apply a holomorphic representation $\rho:(\GL_n(\bb{C})\to \GL(W)$.)
Sections of such vector bundles on $\CA_k(\bb{C})$ are (level $\Gamma_n$, weight $\rho$) Siegel modular forms on $\CA_n(\bb{C})$.

\emph{Notation:} Write
\[
	M_\rho(\Gamma_n) = \left\{ f\in \Gamma(A_n(\bb{C}),\rho(\CE_n):f\text{ is holomorphic at }\infty \right\} 
\]
(the last condition is automatic if $n>1$). We write
\[
S_\rho(\Gamma_n) = \{f:\text{vanish at }\infty\} \subset M_\rho(\Gamma_n)
\]
for the set of \emph{cusp forms}.

We want a group theoretic description of the complex structure on $\CA_n(\bb{C})$ and these vector bundles on $\CA_m(\bb{C})$.

We have $Z(U(n))\simeq U(1)$ and its centralizer in $\Sp_{2n}(\bb{R})$ is $U(n)=K(\R)$ where $K\injto \Sp_{2n,\bb{R}}$ is an algebraic subgroup.

Over $\bb{C}$ we have
\[
	Z(K_{\bb{C}}) \simeq \GL_{1,\bb{C}} \injtpo \Sp_{2n,\bb{C}}.
\] 
This determines two opposite parabolic subgroups $Q_+=K_{\bb{C}}N_+$, $Q_{-}K_{\bb{C}}N_-$

\end{document}
