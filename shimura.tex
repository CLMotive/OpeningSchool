\documentclass[oneside]{amsart}

\usepackage[all]{xy}
\usepackage{tikz-cd}
\usepackage[T1]{fontenc}
\usepackage{xstring}
\usepackage{xparse}
\usepackage{xr-hyper}
\usepackage{xcolor}
\definecolor{brightmaroon}{rgb}{0.76, 0.13, 0.28}
\usepackage[linktocpage=true,colorlinks=true,hyperindex,citecolor=blue,linkcolor=brightmaroon]{hyperref}
\usepackage[nameinlink]{cleveref}
\usepackage[left=1.25in,right=1.25in,top=0.75in,bottom=0.75in]{geometry}
%\usepackage[charter,greekfamily=didot]{mathdesign}
%\usepackage{Baskervaldx}
\usepackage{amssymb}
\usepackage{stmaryrd}
\usepackage{mathrsfs}
\usepackage{mathpazo}
\linespread{1.05}

\usepackage[nobottomtitles]{titlesec}
\usepackage{marginnote}
\usepackage{enumerate}
\usepackage{longtable}
\usepackage{aurical}
\usepackage{microtype}

\newtheoremstyle{ega-env-style}%
{}{}{\rmfamily}{}{\bfseries}{.}{ }{\thmnote{(#3)}}%

\newtheoremstyle{ega-thm-env-style}%
{}{}{\itshape}{}{\bfseries}{. --- }{ }{\thmname{#1}\thmnumber{ #2}\thmnote{ (#3)}}%

\newtheoremstyle{ega-defn-env-style}%
{}{}{\rmfamily}{}{\bfseries}{. --- }{ }{\thmname{#1}\thmnumber{ #2}\thmnote{ (#3)}}%

\theoremstyle{ega-env-style}
\newtheorem*{env}{---}

\theoremstyle{ega-thm-env-style}
\newtheorem{theorem}{Theorem}[subsection]
\newtheorem{proposition}{Proposition}[subsection]
\newtheorem{lemma}{Lemma}[subsection]
\newtheorem{corollary}{Corollary}[subsection]
\newtheorem{stheorem}{Theorem}[section]
\newtheorem{slemma}[stheorem]{Lemma}
\newtheorem{skey}[stheorem]{Key Formula}
\newtheorem{conjecture}{Conjecture}[subsection]

\theoremstyle{ega-defn-env-style}
\newtheorem*{definition}{Definition}
\newtheorem{example}{Example}[subsection]
\newtheorem*{examples}{Examples}
\newtheorem*{remark}{Remark}
\newtheorem*{remarks}{Remarks}
\newtheorem*{notation}{Notation}
\newtheorem*{exercise}{Exercise}
\newtheorem*{properties}{Properties}
\newtheorem*{consequences}{Consequences}
\newtheorem*{note}{Note}
\newtheorem*{fact}{Fact}

\makeatletter
\def\l@subsection{\@tocline{2}{0pt}{2.5pc}{2.2pc}{}}
\def
\section{\@startsection{section}{1}%
	\z@{.7\linespacing\@plus\linespacing}{.5\linespacing}%
{\normalfont\bfseries\Large\scshape\centering}}
\renewcommand{\@seccntformat}[1]{%
	\ifnum\pdfstrcmp{#1}{section}=0\textsection\fi%
\csname the#1\endcsname.~}
\makeatother

\def\mathcal{\mathscr}
%% Fonts

\def\sh{\mathcal}                   % sheaf font
\def\bb{\mathbf}                    % bold font
\def\cat{\mathsf}                   % category font

%% Font Letters

\def\CL{\mathcal{L}}
\def\OO{\mathcal{O}}
\def\CX{\mathcal{X}}

%% Cohomology

\def\CH{\mathrm{H}}                 % cohomology H
\def\CHH{\check{\HH}}               % Čech cohomology H
\def\RD{\mathrm{R}}                 % right derived R
\def\LD{\mathrm{L}}                 % left derived L
\def\dual#1{{#1}^\vee}              % dual
\def\Tor{\operatorname{Tor}}        % Tor
\def\Ext{\operatorname{Ext}}        % Ext
\def\HdR{\mathrm{H}_{\mathrm{dR}}}  % de Rham cohomology
\def\Zc{\underline{\mathbb{Z}}}     % constant sheaf with integer coeffs

%% Categories

\def\A{\cat{A}}                     % category A, usually abelian
\def\C{\cat{C}}                     % category C
\def\op{^\cat{op}}                  % opposite category
\def\Set{\cat{Sets}}                % category of sets
\def\Grp{\cat{Gps}}                 % category of groups
\def\Alg{\cat{Alg}}                 % category of algebras
\def\QCoh{\cat{QCoh}}               % category of quasicoherent sheaves
\def\supertilde{{\,\widetilde{\,}\,}}   % use \supertilde instead of ^\sim
\def\Hom{\operatorname{Hom}}        % morphisms
\def\End{\operatorname{End}}        % endomorphisms
\def\Aut{\operatorname{Aut}}        % automorphisms
\def\ker{\operatorname{ker}}        % kernel
\def\img{\operatorname{im}}         % image
\def\coker{\operatorname{coker}}    % cokernel
\def\pr{\operatorname{pr}}          % projection
\def\EssIm{\operatorname{EssIm}}    % essential image
\DeclareMathOperator*{\colim}{colim}   % colimit


%% Schemes

\def\Proj{\operatorname{Proj}}      % Proj
\def\Supp{\operatorname{Supp}}      % support
\def\Spec{\operatorname{Spec}}      % Spec
\def\Spf{\operatorname{Spf}}        % formal Spec
\def\Aff{\mathbb A}                 % affine space
\def\P{\mathbb{P}}                  % projective space
\def\Pic{\operatorname{Pic}}        % Picard group

%% Standard Operators

\def\codim{\operatorname{codim}}    % codimension
\def\id{\operatorname{id}}          % identity

%% Arrows
\renewcommand{\to}{\mathchoice{\longrightarrow}{\rightarrow}{\rightarrow}{\rightarrow}}
\newcommand{\from}{\mathchoice{\longleftarrow}{\leftarrow}{\leftarrow}{\leftarrow}}
\let\mapstoo\mapsto
\renewcommand{\mapsto}{\mathchoice{\longmapsto}{\mapstoo}{\mapstoo}{\mapstoo}}
\def\isoto{\simeq}                  % isomorphism
\def\simto{\xrightarrow{\sim}}      % isomorphism arrow
\def\surjto{\twoheadrightarrow}     % surjetion
\def\injto{\hookrightarrow}         % injection

%% Under/Over Accents

\newcommand{\wh}[1]{\widehat{#1}}   % hat
\newcommand{\wt}[1]{\widetilde{#1}}    % tilde
\def\ul{\underline}                % underline

%% Spaces

\def\Z{\mathbb Z}                  % integers
\def\Q{\mathbb Q}                  % rationals
\def\N{\mathbb{N}}                 % naturals

%% Groups

\def\GL{\bb{GL}}                   % general linear group
\def\SL{\bb{SL}}                   % special linear group
\def\det{\operatorname{det}}       % determinant

%% Sub/Superscripts

\def\et{^\text{\'et}}              % \'etale
\def\an{^{\text{an}}}              % analytic

%% Misc

\newcommand{\defn}[1]{\textbf{#1}}  % definition highlighting
\def\defeq{:=}                     % definition equation
\def\eps{\varepsilon}              % correct epsilon
\newcommand{\gen}[1]{\left\langle\!\left\langle #1 \right\rangle\!\right\rangle}



\def\shHom{\sh{H}\textup{\kern-2.2pt{\Fontauri\slshape om}}}   % sheaf Hom
\def\shProj{\sh{P}\textup{\kern-2.2pt{\Fontauri\slshape roj}}} % sheaf Proj
\def\shExt{\sh{E}\textup{\kern-2.2pt{\Fontauri\slshape xt}}}   % sheaf Ext
\def\shGr{\sh{G}\textup{\kern-2.2pt{\Fontauri\slshape r}}}     % sheaf Gr
\def\shDer{\sh{D}\,\textup{\kern-2.2pt{\Fontauri\slshape er}}} % sheaf Der
\def\shDiff{\sh{D}\,\textup{\kern-2.2pt{\Fontauri\slshape if{}f}}\,} % sheaf Diff
\def\shHomcont{\sh{H}\textup{\kern-2.2pt{\Fontauri\slshape om.\,cont}}}   % sheaf Hom.cont
\def\shAut{\sh{A}\textup{\kern-2.2pt{\Fontauri\slshape ut}}}   % sheaf Aut
\def\shSym{\sh{S}\textup{\kern-2.2pt{\Fontauri\slshape ym}}}   % sheaf Sym

\makeatletter
\newcommand{\cbigoplus}{\DOTSB\cbigoplus@\slimits@}
\newcommand{\cbigoplus@}{\mathop{\widehat{\bigoplus}}}
\makeatother

\def\Mat{\operatorname{Mat}}
\def\Stab{\operatorname{Stab}}
\def\Lie{\operatorname{Lie}}
\def\kg{\mathfrak{g}}
\def\A{\mathbb{A}}
\def\R{\bb{R}}
\def\GSp{\bb{GSp}}
\def\gl{\mathfrak{gl}}
\def\sp{\mathfrak{sp}}
\def\Sch{\cat{Sch}}
\def\Ab{\cat{Ab}}
\def\Gal{\operatorname{Gal}}
\def\tr{\operatorname{tr}}
\def\Frob{\operatorname{Frob}}
\def\S{\mathbb{S}}
\def\Res{\operatorname{Res}}
\def\kz{\mathfrak{z}}
\def\Rep{\cat{Rep}}
\def\Ad{\operatorname{Ad}}
\def\C{\bb{C}}
\def\SU{\bb{SU}}
\def\Sh{\operatorname{Sh}}
\def\der{_{\mathrm{der}}}
\def\et{_{\text{\'et}}}
\def\real{\operatorname{real}}
\def\PGL{\bb{PGL}}
\def\SO{\bb{SO}}
\def\GSpin{\bb{GSpin}}

\title{Introduction to Shimura Varieties and Their Cohomology}
\author{Lecturer: Olivier Taibi,\quad Typesetter: Micha{\l} Mruga{\l}a}

\begin{document}
\maketitle
\emph{Plan:}
\begin{enumerate}[I)]
	\item Siegel modular varieties
	\item General Shimura varieties
	\item (\'Etale) Cohomology: Kottwitz conjecture
\end{enumerate}
(The cohomology gives a realization of the global Langlands correspondence for certain number fields.)

The main reference are Sophie Morel's notes from the IHES '22 summer school.

\section{Siegel Modular Varieties}
Analytically: we want to parametrize complex tori. This is fairly simple. Let $V$ be a $\C$-vector space of dimension $m\ge 1$, $\Lambda\subset V$ a lattice (a discrete subgroup such that $V /\Lambda$ is compact), then $X=V /\Lambda$ is a complex Lie group, which is a complex torus. 

\begin{exercise}
	A morphism $f:X=V /\Lambda \to X' = V' /\Lambda'$ of complex Lie groups is given by a $\C$-linear map $V\to V'$ mapping $\Lambda$ to $\Lambda'$.
\end{exercise}

\emph{Question:} Which complex tori are algebraizable, i.e. $X\injto \P^{n}(\C)$ (equivalent to $X\isoto \ul{X}\an$ for some projective $\ul{X}$ by Chow). Can we find a parametrization?

\begin{example}
	Let $n=1$ complex tori are always algebraic. There is the Weierstrass $\wp$-function
	\begin{align*}
		\wp: V /\Lambda &\longrightarrow \P^{1}(\C) \\
		v &\longmapsto \wp(v) = \frac{1}{v^2} + \sum_{\lambda\in \Lambda-0}\frac{1}{(\lambda+v)^2}-\frac{1}{\lambda^2}
	.\end{align*}
	It embeds $V /\Lambda$ in $\P^2(\C)$ via $[\wp:\wp':1]$ with image $y^2=P_\Lambda(x)$ where $P_\Lambda\in \C[X]$ has degree 3. The coefficients of $P_\Lambda$ are Eisenstein series (modular forms).

	For $n>1$, $X$ is ``almost never'' algebraic.
\end{example}
Recall that $X$ is algebraizable if and only if there exists $\CL\in \Pic(X)$ which is ample (see Mumford's Abelian Varieties). Recall that $\Pic(X)\isoto H^{1}(X,\OO_X^{\times})$. There is a short exact sequence
\begin{equation}
\begin{tikzcd}
	0 \ar[r] & \Zc_X \ar[r] & \OO_X \ar[r,"\exp(2\pi i-)"] & \OO_X^{\times} \ar[r] & 0
\end{tikzcd}
\end{equation}
so we get a long exact sequence. The map
\[
	H^{0}(X,\OO_X) \isoto \C \to \C^{\times}\isoto H^{0}(X,\OO_X^{\times})
\]
is surjective so we get
\begin{equation}
\begin{tikzcd}
	H^{1}(X,\Z) \ar[r,hook] \ar[d,"\isoto"] & H^{1}(X,\OO_X) \ar[d,"\isoto"',"Dolbeaut"] \ar[r] & H^{1}(X,\OO_X) \ar[d,"\isoto"] \ar[r,"\delta"] & \ker (H^2(X,\Z)\to H^2(X,\OO_X)) \ar[d,"\isoto"] \\
	H^{1}(\Lambda,\Z) \ar[d,equal] & \overline{T} & H^{1}(\Lambda,\OO(X)^{\times}) & \Hom\left( \bigwedge^2\Lambda,\Z \right)  \\
	\Hom(\Lambda,\Z) \ar[dr,hook] & T \oplus \overline{T} \ar[u,"\pr_2"] \\
				      & \Hom_{\R}(V,\C) \ar[u,"\isoto"]
\end{tikzcd}
\end{equation}
We have $H^{i}(V,\Z)=0$ for all $i>0$ and $H^{i}(V,\OO_V)=0$ for all $i>0$ so $\Pic(V)=0$. $\overline{T}$ are the antilinear maps $V\to \C$ and $T=\Hom_{\C}(V,\C)$. Observe that
\[
\Pic^{0}(X) = \ker\delta \isoto \frac{\overline{T}}{\pr_2(\Hom(\Lambda,\Z))}
\]
is also a complex torus. The last term is the Neron-Severi group
\begin{align*}
	NS(X) &\isoto \{E:\Lambda\times \Lambda\to \Z\text{ alt.}: E(iv_1,iv_2)=E(v_1,v_2), \forall v_1,v_2\in V\} \\
	      &= \left\{ \Im H:\, H:V\times V\to \C\text{ Hermitian such that }(\Im H)(\Lambda\times \Lambda)\subset \Z \right\} 
.\end{align*}
The Appel-Humbert theorem completely describes $\Pic(X)$ as $\{L(H,\alpha)\} $ with $H$ as above and $\alpha$ an extra datum.

\begin{theorem}[Lefschetz]
	The following are equivalent:
	\begin{enumerate}[1)]
		\item $H$ is positive definite.
		\item $L(H,\alpha)$ is ample (in fact, $L(H,\alpha)^{\otimes 3}$ is enough to embed $X$).
	\end{enumerate}
\end{theorem}
Let $L\in \Pic(X)$ then
\begin{align*}
	\phi_L: X &\longrightarrow \Pic^{0}(X) = \wh{X} \\
	x &\longmapsto T_x^{*}L\otimes L^{-1}
\end{align*}
is a morphism of Lie groups (here $T_x$ is translation by $x$).

\begin{theorem}
	The following are equivalent:
	\begin{itemize}
		\item $L$ is ample.
		\item $\ker\phi_L$ is finite.
		\item $\phi_L$ is surjective (i.e. an isogeny).
	\end{itemize}
\end{theorem}
\begin{exercise}
	Check that $phi_L$ is an isomorphism if and only if $E(\cdot ,\cdot )$ is perfect ($\Lambda\isoto \Hom(\Lambda,\Z)$).
\end{exercise}
\begin{definition}
	Say that such $\phi_L$ is a \defn{polarization}. If $\phi_L$ is an isomorphism, then it is called a \defn{principal polarization}.
\end{definition}
\begin{remark}
	Not every algebraic $X$ admits a principal polarization, but is isogenous to one that does.
\end{remark}
We can define the moduli space
\[
	\CA_n(\C) = \left\{ (X,\phi):X = V /\Lambda\text{ of dimension }n,  \phi:X\to \wh{X}\text{ a principal polarization} \right\} 
.\] 
Let $(V,\Lambda,H)$ be a principally polarized complex torus. Choose a symplectic basis $(e_1,\dots,e_{2n})$ of $\Lambda$, i.e.
\[
	(E(e_i,e_j))_{i,j} = J_n = \begin{pmatrix} 0 & I_n \\ -I_n & 0 \end{pmatrix} 
.\] 
\begin{exercise}
	$L=L(H,\alpha)$ is ample if and only if $e_{n+1},\dots,e_{2n}$ is a basis of $V$ over $\C$ such that
	\[
	\tau = \Mat_{e_{n+1},\dots,e_{2n}}(e_1,\dots,e_n)
	\] 
	satisfies $\tau = ^{t}\tau$ and $\Im(\tau)$ is positive definite.
\end{exercise}
\begin{definition}
	$\H_n^{+}$ is the set of such $\tau\in M_n(\C)$. There is an algebraic group
	\[
	\Sp_{2n,\Z}:R \mapsto \left\{ g\in M_{2n}(R) : ^{t}gJ_ng = J_n \right\} 
	.\] 
\end{definition}
There is an action of $\Sp_{2n}(\Z)$ on $\H_n^{+}$ such that given
\[
	\gamma = \begin{pmatrix} a & b \\ c & d \end{pmatrix} 
\] 
we have
\[
\tau \gamma = (\tau b + d)^{-1}(\tau a + c)
\]
(this corresponds to replacing $\ul{e}=(e_1,\dots,e_{2n})$ by $\ul{e}\gamma$).

We prefer left actions: let $^{t}\gamma$ act so that $\gamma\tau = \tau*^{t}\gamma$, i.e.
\[
	(\tau^{t}c + d)^{-1}(\tau^{t}a+^{t}b) = (a\tau + b)(c\tau + d)
.\] 
This extends to an action of $\Sp_{2n}(\R)$ on $\H_n^{+}$. This action is transitive and
\begin{align*}
	\Stab_{\Sp_{2n}(\R)}(iI_n) &\longrightarrow U(n) \\
	\begin{pmatrix} a & b \\ -b & a \end{pmatrix}  &\longmapsto a + ib
\end{align*}
is an isomorphism (this is a maximal compact subgroup).

So $\CA_n(\C)\isoto \Gamma_n\setminus \H_n^{+}$ where $\Gamma_n\isoto \Sp_{2n}(\Z)$.
\begin{remark}
	There exists $\gamma\in \Gamma_n\setminus \{\pm 1\} $and $\tau\in \H_n^{+}$ such that $\gamma\tau=\tau$.
\end{remark}

There is a universal object
\[
	\CX_n(\C) = \Z^{2n}\rtimes \Gamma_n\setminus\C^{n}\times \H_n^{+}
\] 
where
\[
	\gamma(v,\tau) = \begin{pmatrix} a & b \\ c & d \end{pmatrix} (v,\tau) = ((\tau^{t}c+^{t}d)^{-1}v, \gamma\tau)
\] 
and
\[
	(\lambda_1,\lambda_2)(v,\tau) = (v+\lambda_1+\tau\lambda_2,\tau)
\]
for $\lambda_i\in \Z^{n}$.

There is a morphism $\pi:\CX(\C)\to \CA_n(\C)$ which admits a section $e$. The fiber of $\tau$ is $[\tau]\isoto\C^{n} /\Lambda_\tau$ where $\Lambda_\tau = \Z^{n}\oplus\tau \Z^{n}$. We get the \defn{Hodge bundle}: take $\Omega^{1}(V /\Lambda)$ which are translaton invariant 1-forms, which is isomorphic to $V^{*}$ via $e^{*}$, then the Hodge bundle is
\[
	\CE_n = e^{*}\Omega^{1}_{\CX(\C) / \CA_n(\C)} \isoto \Gamma_n\setminus\C^{n}\times \H_n^{+}
\]
for action (??).

We can apply Schur functors (use the action of $\mathfrak{S}_k$ on on $\CE_n^{\otimes k}$ to act on subbundles, e.g. $\bigwedge^{k}\CE_n$ for $0\le k\le n$ ). (Equivalently see $\CE_n$ as a $\GL_n(\C)$-bundle on $\CA_n(\C)$ and apply a holomorphic representation $\rho:(\GL_n(\C)\to \GL(W)$.)
Sections of such vector bundles on $\CA_k(\C)$ are (level $\Gamma_n$, weight $\rho$) Siegel modular forms on $\CA_n(\C)$.

\emph{Notation:} Write
\[
	M_\rho(\Gamma_n) = \left\{ f\in \Gamma(A_n(\C),\rho(\CE_n):f\text{ is holomorphic at }\infty \right\} 
\]
(the last condition is automatic if $n>1$). We write
\[
S_\rho(\Gamma_n) = \{f:\text{vanish at }\infty\} \subset M_\rho(\Gamma_n)
\]
for the set of \defn{cusp forms}.

We want a group theoretic description of the complex structure on $\CA_n(\C)$ and these vector bundles on $\CA_m(\C)$.

We have $Z(U(n))\isoto U(1)$ and its centralizer in $\Sp_{2n}(\R)$ is $U(n)=K(\R)$ where $K\injto \Sp_{2n,\R}$ is an algebraic subgroup.

Over $\C$ we have
\[
	Z(K_{\C}) \isoto \GL_{1,\C} \injto \Sp_{2n,\C}.
\] 
This determines two opposite parabolic subgroups $Q_+=K_{\C}N_+$, $Q_{-}K_{\C}N_-$.

\subsection{Siegel modular forms as automorphic forms}
Let $\rho:\GL_n(\C)\to \GL(W)$ be a holomorphic (equivalently algebraic) representation. \defn{Siegel modular forms} are
\begin{align*}
	M_\rho(\Gamma_n) &= \left\{ \begin{array}{c} f:\H_n^{+}\to W \\ \text{holomorphic} \end{array} \middle| \begin{array}{c}
		\forall \gamma = \begin{pmatrix} a & b \\ c & d \end{pmatrix} \in \Gamma_n,\forall \tau\in \H_n^{+}, f(\gamma \tau) = \rho(c\tau+d)f(\tau)\\
		\text{and }f\text{ holomorphic at }\infty
	\end{array} \right\} \\
			 &\subset H^{0}(\CA_n(\C),^{\rho}\CE_n).
\end{align*}
$^{\rho}\CE_m$ comes from a $\Sp_{2n}(\R)$-equivariant vector bundle on 
\[
\begin{tikzcd}
	& \H_n^{+} \ar[r,hook] & \Sp_{2n}(\C) /Q_-(\C) \\
	\Sp_{2n}(\R) \ar[r] & \Sp_{2n}(\R) /U(n) \ar[u,"\isoto"]
\end{tikzcd}
\] 
Define
\begin{align*}
	j: \Sp_{2n}(\R)\times \H_n^{+} &\longrightarrow \GL_n(\C) \\
	\begin{pmatrix} a & b \\ c & d \end{pmatrix} , \tau &\longmapsto c\tau + d
.\end{align*}
This is a cocycle
\[
	j(gg',\tau) = j(g,g'\tau)j(g',\tau)
\] 
(so $j(-,i)|_{U(n)}:U(n)\to \GL_n(\C)$ is a morphism). To $f\in M_\rho(\Gamma_n)$ associate
\begin{align*}
	\phi_f: \Gamma_n \setminus \Sp_{2n}(\R) &\longrightarrow W \\
	g &\longmapsto \phi_f(g) = \rho(j(g,i))^{-1}f(gi)
\end{align*}
a smooth function. Let $g\in \Sp_2n(\R)$ and $k\in U(n)$, then
\[
\phi_f(gk) = \rho(j(k,i))^{-1}f(gi).
\]
Assume $W=\C$ for simplicity, e.g. $^{\rho}\CE_n = \left( \bigwedge^{n}\CE_n \right) ^{\otimes k}$. Then 
\[
	\phi_f\in \CA(\Gamma_n\setminus \Sp_{2n}(\R)) \subset C^{\infty}(\Gamma_n\setminus \Sp_{2n}(\R, \C)	
\]
(? details). This space has actions by $\kg$ and $U(n)$. By the Cauchy-Riemann equations $f$ is holomorphic if and only if $\phi_f$ is killed by $\Lie N_-\subset \kg=\C\otimes_{\R}\Lie \Sp_{2n}(\R)$. Note that $\Lie(\Sp_{2n}(\R))$ acts on $C^{\infty}(\Gamma_n\setminus \Sp_{2n}(\R))$ by
\[
\left( X\cdot \phi \right) (g) = \left.\frac{d}{dt}\right|_{t=0}\phi(ge^{tX}).
\] 
$\phi_f$ lies in some generalized Verma module in $\CA(\Gamma_n\setminus \Sp_{2n}(\R))$.

If $f\in S_\rho(\Gamma_m)$ (vanishes at $\infty$) then
\[
	\phi_f\in \CA_{\text{cusp}}(\Gamma_n\setminus\Sp_{2n}(\R)) \subset \CA^2(-) \subset \CA(-)
\] 
and $\CA^2(-)$ decomposes inside $L^2(\Gamma_n\setminus\Sp_{2n}(\R))$ with the action of $\Sp_{2n}(\R)$. This means that cusp forms have fast decay at cusps.

As a $(\kg,U(n))$-module,
\[
	\CA_{cusp}\subset \CA^2(\Gamma_n\setminus\Sp_{2n}(\R)) \isoto \bigoplus_{\substack{\pi\text{ irr} \\ (\kg,U(n))\text{-mod}}}\pi^{\oplus m(\pi)}.
\] 
Siegel cusp forms correpond to special vectors in some of these $\pi$s ($U(n)$-equivariant and killed by $\Lie N$).

\subsection{Level structures}
Let $X = V /\Lambda$ be a complex torus with a principal polarization $phi:X\simto \wh{X}$. For $M\ge 1$ 
\[
	X[M] \defeq \ker\left( X\xrightarrow{\times M}X \right) = \frac{1}{M}\Lambda /\Lambda \isoto (\Z /M)^{2n}.
\] 
The map $[M]_X:X\to X$ is an isogeny (i.e. surjective with finite kernel). For all isogenenies $f:X\to Y$ inducing $\wh{f}=f^{*}:\wh{Y}\to \wh{X}$, also an isogeny. We get the Weil pairing
\begin{align*}
	\ker f\times \ker\wh{f} &\longrightarrow \C^{\times} \\
	(x,[L]) &\longmapsto \left<x,[L] \right>
.\end{align*}
Choose $t:f^{*}L\simto \OO_X$ we have
\[
\begin{tikzcd}
	T_x^{*}f^{*}L \ar[r,"{T_x^{*}(t)}"] \ar[d,"\isoto"] & T_x^{*}\OO_X \ar[d,"\isoto"] \\
	f^{*}L \ar[r,"{t\times \left<x,[L] \right>}"] & \OO_X.
\end{tikzcd}
\] 
$f=[M]_X$ is a special case, then we get $X[M]\times \wh{X}[M]\to \mu_M(\C)$ and usaing a polarization we get
\[
	\left<\cdot ,\cdot  \right>_\phi:X[M]\times X[M]\to \mu_M(\C).
\] 
\begin{proposition}
	$\left<\cdot ,\cdot  \right>_\phi$ is alternating and non-degenerate.
\end{proposition}
\begin{proof}
	Recall that $\phi$ is $\phi_L$ for some $L=L(H,\alpha)$, let $E=\Im H:\Lambda\times \Lambda\to \Z$. Then
	\[
	\begin{tikzcd}
		X[M] \times X[M] \ar[r,"{\left<\cdot ,\cdot  \right>_\phi}"] \ar[d,"\isoto"] & \mu_M(\C) \\
		\left( \frac{1}{M}\Lambda /\Lambda \right) ^2 \ar[r,"{ME(\cdot ,\cdot )}"] & \frac{1}{M}\Z /\Z \ar[u,"{\exp(2\pi i -)}"']
	\end{tikzcd}
	\] 
\end{proof}
\begin{definition}
	Temporarily we define a level structure on $(X,\phi)$ to be
	\[
		(\Z /M)^{2n} \xrightarrow[\eta]{\sim} X[M]
	\] 
	such that $\eta^{*}\left<\cdot ,\cdot  \right>_\phi$ is the standard pairing for metric $J_n$.
\end{definition}
\begin{fact}
	By strong approximation $\Sp_{2n}(\Z)\surjto \Sp_{2n}(\Z /M\Z)$. Define $\Gamma_m(M)$ to be the kernel.
\end{fact}
\begin{corollary}
	There is a bijection
	\[
		\left\{ (X,\phi,\eta) \middle| \text{PPAV with a level }M\text{ structure} \right\} / \sim \isoto \Gamma_n(M) \setminus \H_n^{+} = \CA_n'(M)(\C).
	\] 
\end{corollary}
\begin{exercise}
	For $M\ge 3$, for all $\tau\in \H_n^{+}$ show that $\Stab_{\Gamma_n(M)}(\tau) = \{1\}$. (?)
\end{exercise}
We get a tower $(\CA'_n(M)(\C))_{M\ge 1}$ ordered by divisibility. For $M \mid |M'$ we get $\CA_n'(M')(\C)\to \CA_n'(M)(\C)$.

Given $(X,\phi)$ 
\[
\left\{ \text{level }M\text{ structures on }(X,\phi) \right\} 
\] 
is a right $\Sp_{2n}(\Z /M\Z)$-torsor which gives us an action of
\[
	\Sp_{2n}(\wh{\Z}) = \varprojlim_M \Sp_{2n}(\Z /M)
\] 
on this tower.

Also
\[
	\CA_n'(M)(\C) \isoto \CA'_n(M')(\C) /\left( K(M) /K(M') \right) 
\] 
where 
\[
	K(M) = \ker\left( \Sp_{2n}(\wh{Z}) \to \Sp_{2n}(\Z /M) \right) .
\] 
The quotient $K(M) /K(M')$ is a finite group.

\subsection{Hecke operators (adelically)}
The goal is to define more natural maps between $\CA_n'(M)(\C)$. The basic idea is that given $(X,\phi,\eta)$, we should also consider isogeneus complex tori (i.e. quotients of $X$ by finite subgroups). But there are some problems: this is not strictly compatible with principal polarizations. Let $f:X\to Y$be an isogeny, $\phi$ be a principal polarization for $Y$, then $f^{*}\phi\defeq \wh{f}\circ \phi \circ f$ has degree $(\deg f)^2$, so it is not principal unless $f$ is an isomorphism.

There are two solutions:
\begin{enumerate}[1)]
	\item Rescale polarizations.
	\item Consider quasi-isogenies 
		\[
			f\in \Q\otimes\Hom(X,Y) \text{ such that }\exists M\ge 1 \text{ with } Mf\in \Hom(X,Y) \text{ an isogeny}.
		\]
\end{enumerate}
Let's do both.

Recall the ring of adeles $\A = \R\times \A_f$ where
\[
	\A_f = \prod_p'(\Q_p,\Z_p) = \left\{ (x_p)_{p\text{ prime}} \middle| \begin{array}{c}
		x_p\in \Q_p \\
		\exists \text{ finite }S\text{ such that}\forall p\not\in S,x_p\in \Z_p
	\end{array}\right\} .
\] 
Recall that
\begin{align*}
	\text{lattices in }\Q\Lambda &\leftrightarrow \left\{ (\Lambda'_p)_p \middle| \begin{array}{c}
		\Lambda'_p\subset \Q_p\otimes_\Z\Lambda\text{ is a }\Z_p\text{-lattice} \\
		\exists \text{ finite }S\text{ such that }\forall p\not\in S, \Lambda'_p = \Z_p\Lambda  
	\end{array}\right\} \\
				     &\leftrightarrow \GL(\A_f\otimes\Lambda) /\GL(\wh{Z}\otimes \Lambda).
\end{align*}
\begin{proof}
	Reduce to the case where
	\[
	M\Lambda \subset \Lambda' \subset \frac{1}{M}\Lambda
	\] 
	and use the chinese remainder theorem.
\end{proof}
\begin{proposition}
	Let $(X,\phi)$ be a principally polarized abelian variety.
	\begin{enumerate}[(a)]
		\item Let $L$ be the set of principally polarized abelian varieties $(X',\phi')$ quasi-isogeneous to $(X, /\phi)$, i.e. there exists a quasi-isogeny $f:X'\dashrightarrow X$ such that $f^{*}\phi = c\phi'$, where $c\in \Q_{>0}$.
		\item Let $R$ be the set of $(\Lambda'_p)_p$ such that 
			\[
			\Lambda'_p \subset V_pX \defeq \Q_p \otimes_{\Z_p}T_pX
			\] 
			is a $\Z_p$-lattice such that there exists $k_p$ making $p^{k_p}\left<\cdot ,\cdot  \right>|_{\Lambda_p'\times \Lambda_p'}$ take values in $\Z_p(1)\defeq = \varprojlim_k \mu_{p^{k}}(?)$ and is perfect, as well as there is a finite $S$ such that for all $p\not\in S$
			\[
				\Lambda_p' = T_pX \defeq \varprojlim_k X[p^{k}].
			\] 
	\end{enumerate}
	Then
	\[
	L /\sim \isoto R.
	\] 
	This is also isomorphic to the set of $\GSp_{2n}(\wh{\Z})$-orbits of symplectic trivializations
	\[
		(\A_f^{2n},\text{standard }\left<\cdot ,\cdot  \right>) \simto \left( \Q\otimes \prod_p T_p X, \left<\cdot ,\cdot  \right>_\phi \right) .
	\] 
\end{proposition}
Here $\GSp_{2n}$ is the $\Z$-group scheme
\[
\GSp_{2n}(R) = \left\{ (g,c) | g\in M_{2n}(R),c\in R^{\times }, ^{t}gJ_ng = cJ_n \right\} .
\] 

\begin{definition}
	A \defn{level structure} for $(X,\phi)$ is an isomorphism $(\Z /M)^{2n}\simto X[M]$. $\Z /M\simto \mu_M(\C)$ such that the obvious diagram commutes.
\end{definition}
We have
\begin{align*}
	\CA_n(M)(\C) &\isoto \left\{ (X,\phi,\eta) \middle| \text{PPAV with level }M\text{ structure} \right\} / \sim \\
			 &\isoto \left\{ (X',\phi') \middle| K(M)\text{-orbit of trivalization of }\Q\otimes \prod_p T_pX' \right\} / \text{quasi-isogeny} \\
			 &\isoto \GSp_{2n}(\Q) \setminus \left( \H_n^{\pm}\times \GSp_{2n}(\A_f) /K(M) \right) 
\end{align*}
where
\[
	\H_n^{\pm} = \H_n^{+} \coprod \H_n^{-}
\] 
has an action of $\GSp_{2n}(\R)$ and
\[
	K(M) \defeq \ker \left( \GSp_{2n}(\wh{\Z}) \to \GSp_{2n}(\Z /M) \right) .
\] 

From now on we write $G$ for $\GSp_{2n}$.
\marginpar{Lecture 3}
We have a tower $(\CA_m(M)(\C))_{M\ge 1}$ (a $\GSp_2(\wh{\Z})$-torsor over $\CA_m(\C)$ with a right action of $G(\A_f)$ and
\[
	\CA_m(M)(\C) \isoto G(\Q) \setminus \left( \H_n^{\pm} \times G(\A_f) /K(M) \right) .
\] 
For $g\in G(\A_f)$ and $M,M'$ satisfying $K(M')\subset gK(M)g^{-1}$ define
\begin{align*}
	T_g: \CA_m(M')(\C) &\longrightarrow \CA_m(M)(\C) \\
	[\tau, h] &\longmapsto [\tau, hg].
\end{align*}

There is also an action on Siegel modular forms. Note that $T_g^{*}\CE_m\isoto \CE_m$ and on $\CA_m(M)(\C)$, $\CE_m=T_1^{*}\CE_m$. Hence  for $\rho:\GL_m(\C)\to \GL(W)$ there is an acton of $G(\A_f)$ on 
\[
M_\rho \defeq \varinjlim_M M_\rho(K(M))
\] 
where
\[
M_\rho(K(M)) \defeq H^{0}(\CA(M)(\C), ^{\rho}\CE_m)) + \text{holomorphy at }\infty\text{ if }m=1.
\] 
$M_\rho$ contains cusp forms $S_\rho$ and by unitarity
\[
S_\rho \simeq \bigoplus_{\pi_f\text{ irrep of }G(\A_f)}\pi_f^{\oplus m(\pi_f)}
\] 
(note that $\pi_f$ are infinite-dimensional).

We recover 
\[
	\CH_B^{k}(\CA_m(M)(\C),\Q) \isoto (\CH_B^{k})^{K(M)}
\] 
where the right hand side admits an action of Hecke operators $H(G(\A_f),K(M))$. (Trace map?)

\begin{theorem}[Franke, Generalization of Matsushima's formula]
	We have
	\begin{align*}
		\C\otimes_\Q \CH^{\bullet}_B &\xrightarrow[G(\A_f)\text{-equiv.}]{\sim} \CH^{\bullet}( \sp_{2m}, U(m); \overbrace{\CA(G(\Q) \setminus G(\A) / \bb{R}_{>0}}^{\CA(G)} )  \\
						 &\defeq \CH^{\bullet} \left( \Hom_{U(m)}\left( \bigwedge^{\bullet}\sp_{2n} / \gl_m \right) , \CA(G) \right) .
	\end{align*}
\end{theorem}
\begin{remark}\leavevmode
	\begin{enumerate}[1)]
		\setcounter{enumi}{-1}
		\item It's ``easy'' if we replace $\CA(G)$ by $C^{\infty}$ and use de Rham cohomology for the LHS.
		\item If $\Gamma_m\setminus \H_m^{+}$ was compact, this would be obtained from the Hodge decomposition for Riemannian manifolds.
		\item $\CA(G)$ is not semi-simple at all.
		\item If $m=1$ we can use this to recover the Eichler-Shimura isomorphism. Let $\CH^{1}_{B,\text{cusp}}$ be the subspace of $\CH^{1}_B$ defined by ``vanishing at cusps''. Then
			\[
				\C\otimes_\Q \CH^{1}_{B,\text{cusp}} \xrightarrow[\GL_2(\A_f)]{\sim} S_2 \oplus \overline{S}_2.
			\] 
			If $\Gamma_1\setminus\H_1^{+}$ was compact (thus a projective curve over $\C$) this would follow from the Hodge decomposition because $\CE_1^{\otimes 2}\isoto \Omega^{1}$ on $\CA_1(\C)$.
	\end{enumerate}
\end{remark}
\subsection{Siegel modular varieties, algebraically}
\begin{definition}
	Let $S$ be a scheme. An \defn{abelian scheme} over $S$ is an $S$-group scheme $X\to S$ which is smooth, proper with connected geometric fibers. If $S=\Spec k$ we call abelian schemes \defn{abelian varieties}.
\end{definition}
\begin{proposition}
	Automatically commutative.
\end{proposition}
\begin{definition}
	Let $X\to S$ be an abelian scheme and $e:S\to X$ be the identity section we define a functor
	\begin{align*}
		\Pic_{X /S,e}: \Sch_{S} &\longrightarrow \Ab \\
		T &\longmapsto \left\{ (L,\alpha): L\in \Pic(X\times_S T)\text{ and }\alpha \text{ trivializes }e^{*}L \right\} .
	\end{align*}
	There is a subfunctor $\Pic_{X /S,e}^{0}$ defined by the data such that for all $t\in T$ and all smooth projective curves $C$ over $K(t)$, for all $f:C\to X\times_S K(t)$, $f^{*}L$ has degree 0.
\end{definition}
\begin{theorem}[Artin, Raynaud]
	$\Pic^{0}_{X /S,e}$ is represented by an abelian scheme over $S$.
\end{theorem}
We write $\wh{X}$ for this scheme.
\begin{definition}
	For $L\in \Pic(X)$, we have 
	\begin{align*}
		\phi_L: X &\longrightarrow \wh{X} \\
		x\in X(T) &\longmapsto T_x^{*}L\otimes L^{-1}.
	\end{align*}
	A \defn{polarization} is an isogeny (i.e. finite, faithfully flat) $\phi:X\to \wh{X}$ such that for all geometric points $p:\Spec k\to S$, $\phi_p=\phi_L$ for some ample $L$. A polarization is \defn{principal} if it is an isomorphism. A \defn{principally polarized abelian variety} (PPAV) is the data $(X,\phi)$ of an abelian variety $X$ and a principal polarization $\phi$.
\end{definition}
\begin{proposition}
	If $M\ge 1$ is invertible on $S$ then $X[M]$ (defined to be the kernel of $[M]_X$) is \'etale locally isomorphic to $\ul{(\Z /M)^{2m}}$.
\end{proposition}
\begin{definition}
	Let $M\ge 1$, we define a functor
	\begin{align*}
		\CA_m(M): \Sch_{\Z\left[ \frac{1}{M} \right] } &\longrightarrow \Set \\
		S &\longmapsto \{\text{PPAV }(X,\phi)\text{ with a level }M\text{ structure}\} /\sim
	\end{align*}
\end{definition}
(Groupoid when $M\le 2$?)
\begin{theorem}[Mumford]
	For $M\ge 3$, $\CA_m(M)$ is represented by a smooth quasiprojective scheme over $\Z\left[ \frac{1}{M} \right] $ of relative dimension $\frac{m(m+1)}{2}$.
\end{theorem}
By the previous proposition, for all $M \mid M'$ with $M\ge 3$ there is a map
\[
\CA_m(M') \to \CA_m(M) \times_{\Z\left[ \frac{1}{M} \right] }\Z\left[ \frac{1}{M'} \right] 
\] 
which is finite \'etale and a $\ker(G(\Z /M')\to G(\Z /M))$-torsor.

We stil have an action of $G(\A_f)$ on the tower $(\CA_m(M)\times \Q)_{M\ge 1}$ using the same interpretation of the moduli problem as in the analytic case (quasi-isogenies).

\emph{Variant:} Let $p$ be a prime and consider the tower $(\CA_m(M)\times \Z_{(p)})_{(M,p)=1}$. It admits an actioon of $G(\A_f^{(p)})$, where $\A_f^{(p)}$ are finite adeles with $\Q_p$ omitted.

\emph{Applications:}
\begin{enumerate}[1)]
	\item We have a $\Q$-structure on modular forms.
	\item \'Etale cohomology: the comparison theorem tells us that
		\[
			\CH_{\text{\'et}}^{\bullet}(\CA_m(M)_{\overline{\Q}}, \Q_l) \isoto \Q_l \otimes_{\Q}\CH^{\bullet}_B\left( \CA_m(M)(\C), \Q \right) .
		\] 
		The LHS has an action of $G(\Aff)\times \Gal_\Q$ where
		\[
		\Gal_\Q \defeq \Gal(\overline{\Q} /\Q).
		\] 
\end{enumerate}
\begin{example}
	Take $m=1$. Eichler-Shimura and Deligne associated Galois representations to eigenforms of weight $\ge 2$. There eigenforms correspond to automorphic representations
	\[
	\pi = \pi_{\infty}\otimes \bigotimes_p' \pi_{p} \injto S_k
	\] 
	(such that $\pi_{\infty}\isoto D_k$, an irreducible $(\gl_2,U(1))$-module). For almost all $p$, $\pi_p$ is unramified: 
	\[
		\underbrace{\pi^{\GL_2(\Z_p)}}_{\dim 1} \neq 0
	\] 
	with an action of $\mathcal{H}(\GL_2(\Q_p),\GL_2(\Z_p))$ which is commutative. There correspond to $c(\pi_p)$ which are semi-simple conjugacy classes in $\GL_2(\C)$.

	Suitably normalized, there exists a number field $F\subset \C$ such that for almost any $p$,
	\[
	\tr c(\pi_p)\in F,\quad \det c(\pi_p)\in F.
	\] 
\end{example}
\begin{theorem}
	For all $\iota:F\to \overline{\Q_l}$ there is a continuous irreducible representation $\rho_{\pi,\iota}:\Gal_\Q\to \GL_2(\overline{\Q_l})$ such that for almost any $p$, $\rho_{\pi,\iota}$ is unramified at $p$ and
	\[
	\tr \rho_{\pi,\iota}(\Frob_p) = \iota(\tr(c(\pi_p)))
	\] 
	where the Frobenius on the right is geometric. (??)
\end{theorem}

\section{General Shimura varieties}
\begin{definition}
	Let $\S = \Res_{\C /\R}(\GL_{1,\C})$, so
	\[
		\S(A) = (A\otimes_\R\C)^{\times }
	\] 
	for an $\R$-algebra $A$.
\end{definition}
$\Rep(\S)$ correspond to real Hodge structures, i.e. finite dimensional real vector spaces $V$ with a decomposition
\[
	V_{\C}=\C\otimes_\R V = \bigoplus_{p,q\in \Z}V^{p,q}
\] 
such that $\overline{V^{p,q}} = V^{q,p}$ and there is an action of $\kz\in \S(\R)=\C^{\times }$ via $\kz^{-p}\overline{\kz}^{-q}$.

\begin{definition}
	A \defn{Shimura datum} is a pair $(G,X)$ of a commutative reductive group $G$ over $\Q$ and a $G(\R)$-orbit $X$ of morphisms $h:\S\to G_{\R}$ such that
	\begin{enumerate}[1)]
		\item $\S$ acs via $\Ad(h)$ on
			\[
				\kg \defeq \C\otimes_\Q \Lie G = \bigoplus_{p,q}\kg^{p,q}
			\] 
			has kernel of type $\{(-1,1), (0,0), (1,-1)\} $, i.e. $\kg^{p,q}=0$ unless $(p,q)$ lies in this set. (This implies that $\GL_{1,\R}\injto \S$ maps to $Z(G)$.)
		\item $\Ad(h(i))$ is a Cartan involution of $G_{\text{ad},\R} = (G /Z(G))_{\R}$, i.e. an ivolution $\theta$ of $H = G_{\text{ad},\R}$ such that
			\[
				\{g\in H(\C) | \theta(g)=g\} 
			\] 
			is compact.
		\item $G_{\text{ad}}\isoto\prod_i G_{\text{ad},i}$ (with $G_{\text{ad},i}$ simple over $\Q$) has no factor $G_{\text{ad},i}$ such that $G_{\text{ad},i}(\R)$ is compact.
	\end{enumerate}
\end{definition}
\begin{example}
	Let $G=\GSp_{2n}$ and
	\[
		h(a+ib) = \begin{pmatrix} aI_n & bI_n \\ -bI_n & aI_n \end{pmatrix} .
	\] 
\end{example}
\begin{proposition}
	$X$ (a priori a real manifold) is a finite disjoint unon of Hermitian symmetric spaces (of non-compact type).
\end{proposition}
A symmetric space is a quotient $H /K_H$ where $H$ is a commutative semisimple Lie group (real) and $K_H\subset H$ is a maximal compact subgroup, with an $H$-invariant Riemannian structure. The Hermitian part requires a complex structure. These were classified by Cartan (Harish-Chandra, Borel).

Complex structure as in the Siegel case:
\begin{definition}
	Let $h\in X$, define
	\begin{align*}
		h_{\C}: \S_{\C} &\longrightarrow G_{\C} \\
		(\kz_1,\kz_2) &\longmapsto \mu_h(\kz_1)\mu_h'(\kz_2)
	\end{align*}
\end{definition}
From $\mu_h$ we get two opposite parabolic subgroups $Q_{\mu_h}^{+},Q_{\mu_h}^{-}$. Embed $X\injto G(\C) / Q_{\mu_h}^{-}(\C)$ which endows $X$ with a complex structure. 

Now comes a definition motivated by variations of Hodge structure, e.g. if $G\to \GL(V)$ is some representation (over $\Q)$ we get a rational Hodge structure on $V$ for all $h\in X$. There exists a classification (see Deligne in Corvallis), $G_{\C}$ can only have a Dynkin diagram of type $A,B,C,D,E_6,E_7$. In the $A$ case $G_{\R}$ is isogeneous to $\prod \SU(p,q)$ (no $\SL_{n,\R}$!).

\begin{definition}
	Let $K\subset G(\A_f)$ be a compact open subgroup, define
	\[
	\Sh(G,X,K)(\C) \defeq G(\Q) \setminus \left( X\times G(\A_f) / K \right) .
	\] 
	Choose a commutative compact subgroup $X_0$ of $X$ and define
	\[
	G(\Q)^{+} \defeq \Stab_{G(\Q)}(X_0).
	\] 
\end{definition}
Then
\[
	\Sh(G,X,K)(\C) \isoto \coprod_{[g]\in G(\Q)\setminus G(\A_f) / K}G(\Q)^{+}\cap gKg^{-1}\setminus X_0.
\] 
If $K$ is small enough (``neat'') then for all $h\in X_0$ 
\[
\Stab_{?}(h) = \{1\} 
\] 
so $\Sh(G,X,K)(\C)$ is a complex manifold. We still have the maps $T_g$ (right multiplication by $g$).
\begin{theorem}[Baily-Borel, Borel]
	For $K$ neat, $\Sh(G,X,K)(\C)$ has a canonical structure of a smooth group scheme over $\C$. 
\end{theorem}
We still have finite \'etale maps between them.
\begin{definition}
	A \defn{morphism of Shimura data} $(G_1,X_1)\to (G_2,X_2)$ is a morphism $f:G_1\to G_2$ such that $f_\R$ maps $X_1$ to $X_2$.
\end{definition}
In this case, Borel's uniqueness theorem implies that for all $K_1,K_2$ in $G_1(\A_f), G_2(\A_f)$ respectively such that $f(K_1)\subset K_2$ we get
\[
\Sh(G_1,X_1,K_1)(\C) \to \Sh(G_2,X_2,K_2)(\C)
\] 
(``obviously'' ?) is algebraic.

\subsection{Canonical model}
\begin{definition}
	A \defn{model} of $(\Sh(G,X,K))_{K\text{ neat}}$ over some number field $E$ are $(S_K)_{K\text{ neat}}$ of smooth quasiprojective schemes and isomorphisms $S_K\times_E \C\isoto \Sh(G,X,K)$ such that all $T_g$ are defined over $E$ (so finite \'etale).
\end{definition}
\begin{definition}
	Let $(G,X)$ be a Shimura datum. The \defn{reflex field} $E(G,X)$ is the smallest field $E\subset \C$ over which the conjugacy class of $\mu_h$ (any $h$) is defined.
\end{definition}
This is always a number field.

In the case of tori: if $G$ is a torus, each $\Sh(G,X,\C)$ is finite. From global class field theory we get a model of $(\Sh(G,X,K))_K$ over $E(G,X)$.

\begin{definition}
	Let $(G,X)$ be a Shimura datum. A \defn{canonical model} of $(\Sh(G,X,K))_K$ is a model over $E(G,X)$ such that for all  $T\subset G$ (over $\Q$) all $h\in X$ factor through $T_\R$.
\end{definition}
So $(T,\{h\} )$ is a Shimura datum $E(G,X)\subset E(T,\{h\} )$. For all $K$ 
\[
\Sh(T,\{h\} , K\cap T(\A_f))(\C) \to \Sh(G,X,K)(\C)
\] 
are defined over $E(T,\{h\} )$.

\begin{proposition}[Shimura, Deligne]
	Such models are unique.
\end{proposition}
\begin{theorem}[Shimura, Deligne, Tamigawa]
	In the Siegel case, the moduli space $(\CA_m(M)\times \Q)_{M\ge 1}$ is a canonical model.
\end{theorem}
Note that $(K(M))_{M\ge 1}$ is cofial in all compact open subgroups. This is equivalent to the main theorem of complex multiplication.

\begin{theorem}[Shimura, Deligne, ?, Milne, Shih, Moonen]
	For every Shimura datum $(G,X)$ there exists a canonical model.
\end{theorem}
In some cases, proved by reduction to the Siegel case:
\begin{itemize}
	\item Hodge type Shimura varieties: $(G,X) \injto (\GSp_{2n}, \H_n^{\pm})$.
	\item Abelian type Shimura varieties: ``isogeneous'' to Hodge type.
	\item PEL type Shimura varieties which are included in Hodge type Shimura varieties. P stands for polarizations, E stands for endomorphism, L stands for level. These are moduli spaces of polarized abelian varieties with an ``endomorphism''. Canonical models come from the moduli interpretation (follows from Mumford's representability theorem). There are also integral models over $\OO_{E,S}$.
\end{itemize}

\section{The Kottwitz conjecture}
Let $(G,X)$ be a Shimura datum and $E=E(G,X)$ be the reflex field. Assume $G\der$ is anisotropic (i.e. no non-trivial $\GL_{1,\Q}\to G\der$), by reduction theory this is equivalent to $\Sh(G,X,K)$ being projective. Let $l$ be a prime and define
\[
\CH^{k}\et \defeq \varinjlim_K \CH^{k}\et\left( \Sh(G,X,K)_{\overline{E}}, \overline{\Q_l} \right) 
\] 
which admits an action of $G(\A_f)\times \Gal_E$ (where $\Gal_E\defeq \Gal(\overline{E} /E)$). Recall the comparison theorem
\[
\CH^{k}\et \isoto \overline{\Q_l} \otimes_\Q \CH_B^{k}.
\] 
$\CH_B^{k}$ is computed using de Rham cohomology (real) and Hodge decomposition. This gives us $(\kg,K_{\infty})$-cohomology of automorphic forms ($\CA=\CA_{\text{cusp}}$ is semisimple).

\subsection{Conjectural picture}
The Langlands group of $E$ is a topological group $L_E$ extending the Weil group $W_E$ that fits into an sequence
\[
\begin{tikzcd}
	L_E \ar[dr,"|\cdot |"] \ar[r] & W_E \ar[d,"|\cdot |"] \ar[r,twoheadrightarrow] & \Gal_E \\
	 & \R_{>0}
\end{tikzcd}
\] 
and $\ker(L_E\to \R_{>0})$ should be compact. For all places $v$ of $E$ we should have
\[
\begin{tikzcd}
	L_{E_v} \ar[r] \ar[d] & L_E \ar[d] \\
	W_{E_v} \ar[r] \ar[d,hook] & W_E \ar[d,twoheadrightarrow] \\
	\Gal_{E_v} \ar[r,hook] & \Gal_E
\end{tikzcd}
\]  
Also $E$ is functorial in $E$. If $v$ is archimedean:
\begin{itemize}
	\item If $E_v\isoto \C$ we have $L_{E_v}=E_v^{\times }$ ($=W_{E_v}$).
	\item If $E_v\isoto \R$ we have a short exact sequence
		\[
		\begin{tikzcd}
			1 \ar[r] & \C^{\times } \ar[r] & W_\R \ar[r] & \Gal(\C /\R) \ar[r] & 1
		\end{tikzcd}
		\] 
\end{itemize}
There is a relation to (pure) motives (over $E$).
\begin{definition}
	A continuous and semisimple representation $\rho:L_E\to \GL(W)$ is \defn{algebraic} if for all archimedean places $v$ of $E$ 
	\[
		(\rho|_{L_{E_v}})|_{\C^{\times }} \isoto \left( \kz \mapsto \begin{pmatrix} \kz^{a_1}\overline{\kz}^{b_1} \\ & \ddots \\ & & \kz^{a_n}\overline{\kz}^{b_n} \end{pmatrix}  \right) 
	\] 
	for $a_i,b_i\in \Z$.
\end{definition}
\begin{conjecture}
	There exists $L_E\to \cat{GMot}_E(\C)$ ($\cat{GMot}_E$ are pro-reductive groups over $\Q$ whose representations are pure motives over $E$) such that $\rho$ is continuous and semisimple. $\rho$ is algebraic if and only if it factors through $L_E\to \cat{GMot}_E(\C)$. In particular, for all algebraic $\rho$ and all $\iota:\C\isoto \overline{\Q_l}$ (only $\iota_{\overline{\Q}}$ matters) we have
	\[
	\real_L(\rho):\Gal_E \to \GL_n(\overline{\Q_l})
	\] 
	continuous, semisimple with $n=\dim_\C W$ which is ``geometric'' such that for almost all non-archimedeal places $v$ of $E$, both $\rho$ and $\real_L(\rho)$ are unramified at $v$ and 
	\[
	\real_L(\rho)(\Frob_V) \sim \iota(\rho(\Frob_V))
	\] 
	(by Chebotarev this characterizes $\real_L(\rho)$).
\end{conjecture}

\subsection{Langlands dual group}
The isomorphism class of $G_{\overline{\Q}}$ is parametrized by a root datum $(X,R,X^{\vee},R^{\vee})$. Choose a maximal torus $T\subset G_{\overline{\Q}}$. The dual root datum $(X^{\vee},R^{\vee},X,R)$ corresponds to a dual group $\hat{G}_{\overline{\Q}}$ with a Galois action of $\Gal(E' /E)$ for some finite $E'$. Define
\[
	^{L}G \defeq \wh{G}(\C)\rtimes \Gal_\Q.
\] 
For example
\[
\begin{array}{c|c}
	G & \wh{G} \\ \hline
	\GL_m / U(n) & \GL_m \\
	\SL_m & \PGL_m \\
	\Sp_{2n} & \SO_{2n+1} \\
	\GSp_{2n} & \GSpin_{2n+1}
\end{array}
\] 
\begin{definition}
	A \defn{Langlands parameter} is a continuous semisimple
	\[
	\begin{tikzcd}
		L_\Q \ar[rr,"\phi"] \ar[dr] & & L_G \ar[dl] \\
					    & \Gal_\Q.
	\end{tikzcd}
	\] 
	$\phi$ is \defn{discrete} if $\operatorname{Cont}(\phi,\wt{G}(\C)) /Z(\wh{G}(\C))^{\Gal_\Q}$ is finite.
\end{definition}
If $G=\GL_n$, $\phi$ is irreducible.

\subsection{\'Etale cohomology}
Let $d=\dim \Sh(G,X,K)$. Define
\[
P^{d}\et \defeq \text{primitive part of }\CH^{d}\et
\] 
(hard Lefschetz, ``not coming from cohomology of a lower dimensional variety''). Recall $\mu=\mu_h:\Gal_{1,\C}\to G_{\C}$ (up to conjugation). We can see $\mu$ as a character of a maximal torus $T\subset \wh{G}$. This gives an irreducible (algebraic and finite dimensional) representation $V_\mu$ of $\wh{G}(\C)$ extends to
\[
	^{L}(G_E) \xrightarrow{r_\mu} \GL(V_\mu).
\] 
A discrete Langlands parameter $\phi:L_\Q\to L_G$ induces an action of $L_E$ on $V_\mu(\phi)\defeq V_\mu$ by $|\cdot |^{-d /2}(r_{\mu}\circ\phi|_{L_E}$. The group
\[
	C_\phi \defeq \operatorname{Cont}(\phi,\wh{G}(\C))
\] 
also acts on $V_\mu(\phi)$ 
\[
V_\mu(\phi) = \bigoplus_{\chi\in X^{*}(C_\phi)}V_\mu(\phi)_\chi
\] 
(in our case $C_\phi$ is abelian insider the maximal torus of $\wh{G}$).

Now only consider $\phi$ such that
\[
	(\phi|_{W_\R})|_{\C^{\times }} \sim \kz \mapsto (\kz / |\kz|)^{2\rho}
\] 
where $\rho = \frac{1}{2}\sum_{\alpha > 0}\alpha$.
\begin{lemma}
	Each $V_\mu(\phi)_\chi$ is algebraic.
\end{lemma}
\begin{conjecture}[Kottwitz, simplified and vague]
	We have
	\[
		P^{d}\et \isoto \bigoplus_{\text{these }\phi}\bigoplus_{\substack{\pi_f = \bigotimes_p'\pi_p \\ (\pi_f)_p\in \prod'\prod(\phi|_{L_{\Q_p}})}} \bigoplus_{\chi}\real_L(V_\mu(\phi)_\chi)^{\oplus m(\phi,p_f,\chi)}
	\] 
\end{conjecture}
where $m(\phi,\pi_f,\chi)$ are ``explicit'' integers.

\end{document}
