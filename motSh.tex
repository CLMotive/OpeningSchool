\documentclass[oneside]{amsart}

\usepackage[all]{xy}
\usepackage{tikz-cd}
\usepackage[T1]{fontenc}
\usepackage{xstring}
\usepackage{xparse}
\usepackage{xr-hyper}
\usepackage{xcolor}
\definecolor{brightmaroon}{rgb}{0.76, 0.13, 0.28}
\usepackage[linktocpage=true,colorlinks=true,hyperindex,citecolor=blue,linkcolor=brightmaroon]{hyperref}
\usepackage[nameinlink]{cleveref}
\usepackage[left=1.25in,right=1.25in,top=0.75in,bottom=0.75in]{geometry}
%\usepackage[charter,greekfamily=didot]{mathdesign}
%\usepackage{Baskervaldx}
\usepackage{amssymb}
\usepackage{stmaryrd}
\usepackage{mathrsfs}
\usepackage{mathpazo}
\linespread{1.05}

\usepackage[nobottomtitles]{titlesec}
\usepackage{marginnote}
\usepackage{enumerate}
\usepackage{longtable}
\usepackage{aurical}
\usepackage{microtype}

\newtheoremstyle{ega-env-style}%
  {}{}{\rmfamily}{}{\bfseries}{.}{ }{\thmnote{(#3)}}%

\newtheoremstyle{ega-thm-env-style}%
  {}{}{\itshape}{}{\bfseries}{. --- }{ }{\thmname{#1}\thmnumber{ #2}\thmnote{ (#3)}}%

\newtheoremstyle{ega-defn-env-style}%
  {}{}{\rmfamily}{}{\bfseries}{. --- }{ }{\thmname{#1}\thmnumber{ #2}\thmnote{ (#3)}}%

\theoremstyle{ega-env-style}
\newtheorem*{env}{---}

\theoremstyle{ega-thm-env-style}
\newtheorem{theorem}{Theorem}[subsection]
\newtheorem{proposition}{Proposition}[subsection]
\newtheorem{lemma}{Lemma}[subsection]
\newtheorem{corollary}{Corollary}[subsection]
\newtheorem{stheorem}{Theorem}[section]
\newtheorem{slemma}[stheorem]{Lemma}
\newtheorem{skey}[stheorem]{Key Formula}

\theoremstyle{ega-defn-env-style}
\newtheorem*{definition}{Definition}
\newtheorem{example}{Example}[subsection]
\newtheorem*{examples}{Examples}
\newtheorem*{remark}{Remark}
\newtheorem*{remarks}{Remarks}
\newtheorem*{notation}{Notation}
\newtheorem*{exercise}{Exercise}
\newtheorem*{properties}{Properties}
\newtheorem*{consequences}{Consequences}
\newtheorem*{note}{Note}
\newtheorem*{fact}{Fact}

\makeatletter
\def\l@subsection{\@tocline{2}{0pt}{2.5pc}{2.2pc}{}}
\def\section{\@startsection{section}{1}%
  \z@{.7\linespacing\@plus\linespacing}{.5\linespacing}%
  {\normalfont\bfseries\Large\scshape\centering}}
\renewcommand{\@seccntformat}[1]{%
  \ifnum\pdfstrcmp{#1}{section}=0\textsection\fi%
  \csname the#1\endcsname.~}
\makeatother

\def\mathcal{\mathscr}
%% Fonts

\def\sh{\mathcal}                   % sheaf font
\def\bb{\mathbf}                    % bold font
\def\cat{\mathsf}                   % category font

%% Font Letters

\def\CA{\mathcal{A}}
\def\CE{\mathcal{E}}
\def\CF{\mathcal{F}}
\def\CG{\mathcal{G}}
\def\CL{\mathcal{L}}
\def\OO{\mathcal{O}}
\def\CX{\mathcal{X}}
\def\b1{\mathbb{1}}

%% Cohomology

\def\CH{\mathrm{H}}                 % cohomology H
\def\CHH{\check{\HH}}               % Čech cohomology H
\def\RD{\mathrm{R}}                 % right derived R
\def\LD{\mathrm{L}}                 % left derived L
\def\dual#1{{#1}^\vee}              % dual
\def\Tor{\operatorname{Tor}}        % Tor
\def\Ext{\operatorname{Ext}}        % Ext
\def\HdR{\mathrm{H}_{\mathrm{dR}}}  % de Rham cohomology
\def\Zc{\underline{\mathbb{Z}}}     % constant sheaf with integer coeffs

%% Categories

\def\A{\cat{A}}                     % category A, usually abelian
\def\C{\cat{C}}                     % category C
\def\op{^\cat{op}}                  % opposite category
\def\Set{\cat{Sets}}                % category of sets
\def\Grp{\cat{Gps}}                 % category of groups
\def\Alg{\cat{Alg}}                 % category of algebras
\def\QCoh{\cat{QCoh}}               % category of quasicoherent sheaves
\def\supertilde{{\,\widetilde{\,}\,}}   % use \supertilde instead of ^\sim
\def\Map{\operatorname{Map}}        % morphisms (any category)
\def\Hom{\operatorname{Hom}}        % morphisms (additive category)
\def\iHom{\underline{\Hom}}         % interal hom
\def\End{\operatorname{End}}        % endomorphisms
\def\Aut{\operatorname{Aut}}        % automorphisms
\def\ker{\operatorname{ker}}        % kernel
\def\img{\operatorname{im}}         % image
\def\coker{\operatorname{coker}}    % cokernel
\def\pr{\operatorname{pr}}          % projection
\def\EssIm{\operatorname{EssIm}}    % essential image
\DeclareMathOperator*{\colim}{colim}   % colimit

\def\DAet{\cat{DA}\et}              
\def\SH{\cat{SH}}                   
\def\Sh{\cat{Sh}}                   % category of sheaves
\def\PSh{\cat{PSh}}                 % category of presheaves

%% Schemes

\def\Proj{\operatorname{Proj}}      % Proj
\def\Supp{\operatorname{Supp}}      % support
\def\Spec{\operatorname{Spec}}      % Spec
\def\Spf{\operatorname{Spf}}        % formal Spec
\def\Aff{\mathbb A}                 % affine space
\def\P{\mathbb{P}}                  % projective space
\def\Pic{\operatorname{Pic}}        % Picard group

%% Standard Operators

\def\codim{\operatorname{codim}}    % codimension
\def\id{\operatorname{id}}          % identity

%% Arrows
\renewcommand{\to}{\mathchoice{\longrightarrow}{\rightarrow}{\rightarrow}{\rightarrow}}
\newcommand{\from}{\mathchoice{\longleftarrow}{\leftarrow}{\leftarrow}{\leftarrow}}
\let\mapstoo\mapsto
\renewcommand{\mapsto}{\mathchoice{\longmapsto}{\mapstoo}{\mapstoo}{\mapstoo}}
\def\isoto{\simeq}                  % isomorphism
\def\simto{\xrightarrow{\sim}}      % isomorphism arrow
\def\surjto{\twoheadrightarrow}     % surjetion
\def\injto{\hookrightarrow}         % injection

%% Under/Over Accents

\newcommand{\wh}[1]{\widehat{#1}}   % hat
\newcommand{\wt}[1]{\widetilde{#1}}    % tilde
\def\ul{\underline}                % underline

%% Spaces

\def\Z{\mathbb Z}                  % integers
\def\Q{\mathbb Q}                  % rationals
\def\N{\mathbb N}                 % naturals
\def\H{\mathcal H}

%% Groups

\def\GL{\bb{GL}}                   % general linear group
\def\SL{\bb{SL}}                   % special linear group
\def\Sp{\bb{Sp}}
\def\det{\operatorname{det}}       % determinant

%% Sub/Superscripts

\def\et{^\text{\'et}}              % \'etale
\def\an{^{\text{an}}}              % analytic

%% Misc

\newcommand{\defn}[1]{\textbf{#1}}  % definition highlighting
\def\defeq{:=}                     % definition equation
\def\eps{\varepsilon}              % correct epsilon
\newcommand{\gen}[1]{\left\langle\!\left\langle #1 \right\rangle\!\right\rangle}



\def\shHom{\sh{H}\textup{\kern-2.2pt{\Fontauri\slshape om}}}   % sheaf Hom
\def\shProj{\sh{P}\textup{\kern-2.2pt{\Fontauri\slshape roj}}} % sheaf Proj
\def\shExt{\sh{E}\textup{\kern-2.2pt{\Fontauri\slshape xt}}}   % sheaf Ext
\def\shGr{\sh{G}\textup{\kern-2.2pt{\Fontauri\slshape r}}}     % sheaf Gr
\def\shDer{\sh{D}\,\textup{\kern-2.2pt{\Fontauri\slshape er}}} % sheaf Der
\def\shDiff{\sh{D}\,\textup{\kern-2.2pt{\Fontauri\slshape if{}f}}\,} % sheaf Diff
\def\shHomcont{\sh{H}\textup{\kern-2.2pt{\Fontauri\slshape om.\,cont}}}   % sheaf Hom.cont
\def\shAut{\sh{A}\textup{\kern-2.2pt{\Fontauri\slshape ut}}}   % sheaf Aut
\def\shSym{\sh{S}\textup{\kern-2.2pt{\Fontauri\slshape ym}}}   % sheaf Sym

\makeatletter
\newcommand{\cbigoplus}{\DOTSB\cbigoplus@\slimits@}
\newcommand{\cbigoplus@}{\mathop{\widehat{\bigoplus}}}
\makeatother


\usepackage{mathtools}

\def\Cone{\operatorname{Cone}}
\def\Ch{\cat{Ch}}
\def\ft{^{\text{(f.t)}}}
\def\Sm{\cat{Sm}}
\DeclareMathOperator{\hocolim}{hocolim}
\def\Sing{\operatorname{Sing}}
\def\F{\bb{F}}
\def\Ga{\mathbb{G}_a}
\def\Gm{\mathbb{G}_m}
\def\Fr{\operatorname{Fr}}
\def\PSp{\operatorname{PSp}}
\def\Sp{\operatorname{Sp}}
\def\Ev{\operatorname{Ev}}
\def\Sus{\operatorname{Sus}}
\def\Et{\cat{Et}}
\def\Ind{\operatorname{Ind}}
\def\rank{\operatorname{rank}}
\def\Bl{\operatorname{Bl}}
\def\Sch{\cat{Sch}}
\def\kz{\mathfrak{z}}
\def\Chow{\operatorname{CH}}
\def\Sym{\operatorname{Sym}}
\def\kg{\mathfrak{g}}
\def\PrL{\operatorname{Pr}^L}

\title{Motivic Sheaves}
\author{Lecturer: Simon Pepin Lehalleur,\quad Typesetter: Micha{\l} Mruga{\l}a}

\begin{document}
\maketitle
\textbf{Plan:}
\begin{enumerate}[L1:]
	\item Motivation, construction, examples.
	\item Six-functor formalism.
	\item Motivic $t$-structures and weight structures.
	\item $\infty$-categorical methods.
\end{enumerate}
\section{Motivation from GRT and cohomology}
\subsection{Cohomology and sheaves for representation theory} \leavevmode

\marginpar{Lecture 1}

\emph{Question:} How do you construct interesting representations?

\emph{Answer:}
\begin{enumerate}[1)]
	\item Find interesting actions.
	\item Linearlize them.
\end{enumerate}

\begin{example}
	Let $K$ be a compact Lie group. The action of $K$ on itself gives us an action of $K$ on $L^2(K)$ with respect to a Haar measure. The Peter-Weyl theorem says that
	\[
	L^2(K) \isoto \cbigoplus_{\pi\text{ unitary}}\pi^{\oplus dim(\pi)}.
	\]
	``Lie theory $\subset $ algebraic geometry''. Reductive groups are algebraic groups with many associated varieties with group actions: flag varieties\dots
\end{example}
The linearizations we consider in this course are the many types of cohomology theories.

\begin{example}[Borel-Weil-Bott]
	Let $T\subset B\subset G$ be a reductive group over $\bb{C}$. Let $\lambda\in X^{\vee}(T)$ such that there exists $w\in W$ with $w*\lambda = w(\lambda+\rho)-\rho>0$ (where $\rho = (1 /2)\sum_{\alpha\in \Phi^{+}}\alpha$). Then
	\[
		\RD\Gamma(G /B,L_\lambda) \isoto \pi_{w*\lambda}[-\ell(w)]
	\] 
	where $\ell(w)$ is the length of $w$.
\end{example}
Cohomology fits in the wider context of sheaf theory. If $T$ is a locally contractible topological space, then
\[
	\CH^{n}_{\text{sing}}(T,\Z) \isoto \CH^{n}(T,\Zc_T) \isoto \RD^{n}(\pi_T)_*(\Zc_T)	
\]
where $\pi_T$ is the morphism $\pi_T:T\to \text{pt}$ with
\[
\RD\pi_{T*}:D(T,\Z) \to D(\Z) \isoto D(*,\Z)
.\] 

Cohomology (singular with $\Q$-coefficients) of algebraic varieties over $\bb{C}$ is \emph{\textbf{very}} special.
\begin{itemize}
	\item There is a weight filtration.
	\item There is a mixed hodge structure.
\end{itemize}
Sheaves on complex algebraic varieties are also very special:
\begin{itemize}
	\item Perverse sheaves;
	\item Decomposition theorem;
	\item Mixed Hodge modules.
\end{itemize}
This leads to great success stories in GRT:
\begin{itemize}
	\item Springer theory;
	\item Kazdhan-Lustig theory;
	\item geometric Satake\dots
\end{itemize}

\subsection{From sheaves to motivic sheaves}
There are situations which can't be directly studied using these tools:
\begin{itemize}
	\item Representation theory of reductive groups over other fields/rings/schemes.
	\item Modular/integral representation theory.
	\item $q$-deformations, quantum groups, canonical bases.
\end{itemize}
These can be attacked using:
\begin{itemize}
	\item $l$-adic sheaves,
	\item sheaves cohomology with $\Z$-coefficients,
	\item $K$-theory.
\end{itemize}
Motivic sheaves will give us a unified perspective.

\emph{Motivic dream:} There should exist universal cohomology/sheaf theories such that
\begin{enumerate}[1)]
	\item they unify and ``explain'' the special stracture in cohomology/sheaves;
	\item they are of algebro-geometric nature;
	\item they ``explain'' the realization of algebraic cycles and algebraic $K$-theory.
\end{enumerate}

\section{Construction of \texorpdfstring{$\DAet$}{DA\'et} and \texorpdfstring{$\SH$}{SH} (Morel-Voevodsky)}
\subsection{Triangulated categories and localization}
\begin{definition}
	A \defn{triangulated category} is the data:
	\begin{itemize}
		\item an additive category $\C$,
		\item an automorphism $\Sigma=(-)[1]:\C\simto \C$,
		\item a collection of distinguished triangles
			\[
				A \to B \to C \to A[1]
			.\] 
	\end{itemize}
	The data satisfies the following conditions:
	\begin{itemize}
		\item shifted distinguished triangles are distinguished up to isomorphism,
		\item for all $f:A\to B$ there exists
			\[
				A \xrightarrow{f} B \to \Cone(f) \xrightarrow{+}
			\] 
			unique up to isomorphism and functorial,
		\item 
			\begin{equation*}
			\begin{tikzcd}
				A \ar[r] \ar[d,"f","\simeq"'] & B \ar[r] \ar[d,"g","\simeq"'] & C \ar[d,dashed] \ar[r] & A[1] \ar[d,"{f[1]}","\simeq"'] \\
				A' \ar[r] & B' \ar[r] & C' \ar[r] & A'[1]
			\end{tikzcd}
			\end{equation*}	
		\item (??)
	\end{itemize}
\end{definition}

\begin{example}
	Let $\A$ be an abelian category, $\Ch(\A)$ be the abelian category of chain complexes in $\A$. We define $(A[1])_n = A_{n-1}$. Given $f:A_\bullet\to B_\bullet$ the maping cone is given by
	\[
		\Cone(f)_n = A_{n-1} \oplus B_n,\qquad d_n = \begin{pmatrix} -d_{n-1}^{A} & 0 \\ f & d_n^{B} \end{pmatrix} 
	.\] 
\end{example}
\begin{definition}
	$f:A_\bullet\to B_\bullet$ is a \defn{quasi-isomorphism} if for all $n\in \Z$, the map $H_n(A_\bullet)\simeq H_n(B_\bullet)$ is an isomorphism.
\end{definition}
\begin{definition}
	$D(\A)$ is defined as the localization of $\Ch(\A)$ by quasi-isomorphisms.
\end{definition}

Now we consider reflexive localizations (1-categorical ones lead to triangulated and $\infty$-categorical ones).
\begin{definition}
	Let $\C$ be a 1-category.
	\begin{enumerate}[1)]
		\item $\C'\subset \C$ is \defn{reflexive} if $\iota:\C'\to \C$ has a left adjoint.
		\item $L_W:\C\to \C[W^{-1}]$ is \defn{reflexive} if $L_W$ has a right adjoint.
	\end{enumerate}
\end{definition}

\begin{lemma}[Reflexive subcategories are the same thing as reflexive localizations] \leavevmode
	\begin{enumerate}[a)]
		\item Let $\C'\subset \C$ be reflexive, $L:\C\to \C'$ be the left adjoint to $\iota$. Define 
			\[
				W_L=\{f:L(f)\text{ is an isomorphism}\}.
			\]
			Then $\C'\simeq \C[W_L^{-1}]$ and $L\simeq L_{W_L}$.
		\item If $L$ is a reflexive localization, then its right adjoint $\iota$ is fully faithful and $\iota:\C[W^{-1}] \simto \EssIm(\iota)\subset \C$.
	\end{enumerate}
\end{lemma}

\begin{definition}
	Let $S\subset \C$ be a collection of morphisms.
	\begin{enumerate}[a)]
		\item $A\in \C$ is \defn{$S$-local} if for all $f:B\to C$ in $S$ 
			\[
			\Hom_\C(C,A) \simto \Hom_\C(B,A).
			\] 
		\item $f:B\to C$ is an \defn{$S$-equivalence} if for all $S$-local $A$ 
			\[
			\Hom_{\C}(C,A) \simto \Hom_\C(B,A).
			\] 
	\end{enumerate}
\end{definition}
\begin{lemma}
	If $L:\C \rightleftarrows \C':\iota$ is a reflexive localizaton, $W_L$ as before, then
	\begin{itemize}
		\item $\iota$ gives an isomorphism between $\C'$ and $W_L$-local objects.
		\item $W_L$ are the $W_L$-equivalences.
	\end{itemize}
\end{lemma}
\begin{definition}
	Let $\cat{D}$ be a triangulated category with all small products.
	\begin{itemize}
		\item Let $\kappa$ be a regular cardinal (for example $\kappa=\aleph_0$). Then $A\in \cat{D}$ is \defn{$\kappa$-small}/\defn{$\kappa$-compact} if and only if
			\[
				\colim_{\substack{I' \subset I \\ |I'|<\kappa}} \Hom\left( A,\bigoplus_{I'}B_i \right) \simto \Hom\left( A, \bigoplus_I B_i \right) 
			.\] 
		\item \defn{Compact} means $\aleph_0$-small. $A$ is compact if and only if
			\[
			\bigoplus_I \Hom(A,B_i) \simto \Hom \left( A,\bigoplus_{I}B_i \right) 
			.\] 
		\item $\cat{D}$ is \defn{presentable}/\defn{well-generated} if and only if there exist $\kappa$ and a set $S\subset \cat{D}$ of $\kappa$-small objects which generate $\cat{D}$:
			\[
				\forall B\in \cat{D},\left( \forall A\in S, \Hom(A,B)=0 \right) \implies B = 0
			.\] 
		\item $\cat{D}$ is \defn{compactly generated} if it is $\aleph_0$-presentable.
	\end{itemize}
\end{definition}

\begin{definition}
	$\cat{E}\subset \cat{D}$ is \defn{localizing} if it is
	\begin{itemize}
		\item triangulated,
		\item stable under coproducts,
		\item thick (stable under subobjects and subquotients).
	\end{itemize}
\end{definition}

\begin{theorem}[Adjoint Functor Theorem]
	Let $\cat{D},\cat{D}'$ be triangulated categories with all coproducts, $F:\cat{D}\to \cat{D}'$ be a triangulated functor and $\cat{D}$ be presentable. Then $F$ admits a right adjoint if and only if $F$ preserves all coproducts.
\end{theorem}
\begin{corollary}[Verdier Localization]
	Let $\cat{D}$ be a presentable category and $\cat{E}$ be a localizing subcategory. Define
	\[
		\cat{D} / \cat{E} = D[W_{\cat{E}}^{-1}],\quad W_{\cat{E}} = \{f:\Cone(f)\in \cat{E}\} 
	.\] 
	Then $\cat{D} \to \cat{D} /\cat{E}$ is a reflexive localization.
\end{corollary}
Let $S\subset \cat{D}$ be a subset of objects, then $\gen{S}$ is the smallest subcategory containing $S$ such that $\cat{D} /\gen{S}$ is a reflexive localization.

Let's get some inspiration from Betti homology, i.e. singular homology on complex algebraic varieties. Let $X\in \cat{Var}_{\bb{C}}\ft$, then we get
\[
	C_*^{\text{sing}}(X(\bb{C}),\Z)\in D(\Z)
.\]
This comes with some data:
\begin{enumerate}[(a)]
	\item 
		\begin{itemize}
			\item $D(\Z)$ has a symmetric monoidal structure: $\otimes^{\Z}$,
			\item (K\"unneth) $C_*(X\times Y) \simeq C_*(X)\otimes^{\Z}C_*(Y)$.
		\end{itemize}
\end{enumerate}
which satisfies sproperties:
\begin{enumerate}[(a)]
	\setcounter{enumi}{1}
	\item ($\Aff^{1}$-homotopy invariance) $C_*(X\times \Aff^{1})\simto C_*(X)$ ($(\Aff^{1})\an=\bb{C}$ is contractible).
	\item[(c')] (Mayer-Vietoris sequence) Let $X=U \cup V$ be a Zariski cover, then
		\[
			C_*(U\cap V) \to C_*(U) \oplus C_*(V) \to C_*(X) \xrightarrow{\partial} C_*(U \cap  V)[1]
		.\] 
	\item (\'Etale descent) Let $U\to X$ be \'etale surjective. Define
		\[
			\check{C}_n(U /X) = U^{n+1}
		.\] 
		Then $\check{C}_\bullet(U /X)$ is a simplicial scheme, so $C_*(\check{C}_\bullet(U /X))$ is a simplicial complex of abelian groups and $C(C_*(\check{C}_\bullet(U /X)))$ is a double complex. (??)

		Concretely we have a descent spectral sequence which gives us ($U=U \cup V$) Mayer Vietoris.
	\item ($\P^{1}$-stabilization)
		\begin{align*}
			C_*(\P_{\bb{C}}^{1}) &\simeq C_*(\text{pt}) \oplus \wt{C}_*(\P_{\bb{C}}^{1}) \\
					     &\simeq \Z[0] \oplus \Z(1)[2]
		.\end{align*}
		$\Z(1)$ is $\oplus$-invertible.
\end{enumerate}
Smooth varieties play a special role:
\begin{itemize}
	\item Poincar\'e duality
	\item Gysin sequences
	\item $C_*(-)$ also satisfies ``$h$-descent'', so $C_*(-)$ is ``determined'' by $C_*(-)_{|(?)}$.
\end{itemize}
There is an associated sheaf theory:
\[
	D_B(-):\cat{Var}_{\bb{C}}\to \cat{TriCat}^{\otimes},\quad X\mapsto D_B(X) = D(\Sh(X\an,\Z))
.\] 

\emph{Sketch of $\DAet$:} Let $S$ be a base scheme.
\begin{itemize}
	\item Start with
		\[
		\begin{cases}
			D(\PSh(\Sm_S, \Z)) = D_{\PSh}(S) \\
			\Z[-]: \Sm_S \to D_{\PSh}(S)
		\end{cases}
		.\] 
	\item Impose $\Aff^{1}$-invariance, \'etale descent, and $\P^{1}$-stability. This will give us $\DAet(S,\Z)$ and $M_S(-):\Sm_S\to \DAet(S,\Z)$.
\end{itemize}
The surprise is that the result satisfies many \emph{other} properties of singular (co)homology and derived categories of sheaves on complex varieties which can largely be packaged into the six-functor formalism. The result is closely related to algebraic cycles/higher Chow groups and algebraic $K$-theory.

\marginpar{Lecture 2}
(Fill in $H_*$ frmo the recall part)

Let $S$ be a qcqs scheme, $\Lambda$ be a coefficient ring. Define
\[
	\begin{cases}
		D_{\PSh}(S) \defeq D(\PSh(\Sm_S,\Lambda)) \text{a presentable, symmetric monoidal, triangulated category}\\
		\Lambda[\cdot ]:\Sm_S \to D_{\PSh}(S)
	\end{cases}
\] 
\emph{\'Etale descent:}
\begin{align*}
	D_{\text{\'et}}(S) &\defeq D(\Sh_{\text{\'et}}(\Sm_S,\Lambda)) \\
			   &= D_{\PSh}(S)[W_{\text{\'et}}^{-1}]
\end{align*}
where $W_{\text{\'et}}$ are \'etale-local weak equivalences, i.e. $(f:K_\bullet\to L_\bullet)\in W_{\text{\'et}}$ if for all $n$ we have
\[
	(?)_{\text{\'et}}\CH_n(K_\bullet) \simto (?)_{\text{\'et}}\CH_n(L_\bullet).
\] 

\emph{$\Aff^{1}$-invariance}
Let
\[
	I_{\Aff^{1},(\text{\'et})} = \{\dots\to 0\to \Lambda_{(\text{\'et})}[X\times \Aff^{1}]\to \Lambda_{(\text{\'et})}[X]\to 0\to \dots | X\in \Sm_S\} .
\] 
Define
\[
	D_{\Aff^{1}}(S) \defeq D_{\PSh}(S) / \gen{I_{\Aff^{1}}} = D_{\PSh}(S)[W_{\Aff^{1}}^{-1}].
\] 
We have
\[
L_{\Aff^{1}}: D_{\PSh}(S) \to D_{\PSh}(S)^{\Aff^{1}-\text{loc}} \injto D_{\PSh}(S).?
\] 
with the middle term isomorphic to $D_{\Aff^{1}}(S)$.

\begin{definition}
	Define
	\[
		\bb{\Delta}^{n}_{\text{alg},S} \defeq \Spec_S\left( \OO_S[X_0,\dots,X_n] / \left( \sum x_i-1 \right)  \right) \simeq \Aff_S^{n}
	\] 
	then $\bb{\Delta}_{\text{alg},S}^{\bullet}$ is a cosimplicial scheme over $S$.
\end{definition}
\begin{definition}[Suslin-Voevodsky]
	Define
	\[
		\Sing^{\Aff^{1}}(K_\bullet) = \hocolim_{\bb{\Delta}\op} K_\bullet(\bb{\Delta}_{\text{alg},S}^{\bullet}\times_S X)
	\] 
\end{definition}
\begin{example}
	Let $F\in \PSh$ then
	\[
		\Sing^{\Aff^{1}}(F)(U) = \left[ \dots \to  F(\bb{\Delta}^2\times U)\to F(\Aff^{(?)}\times U) \xrightarrow{i_0^{*}-i_1^{*}}  F(U) \to 0 \right] .
	\] 	
\end{example}
\begin{proposition}
	$L_{\Aff^{1}}\simeq \Sing^{\Aff^{1}}$.
\end{proposition}
\begin{proof}
	The idea is to use
	\begin{align*}
		m: \Aff^{1}\times \Aff^{1} &\longrightarrow \Aff^{1} \\
		(x,y) &\longmapsto xy
	\end{align*}
	to prove
	\begin{enumerate}[a)]
		\item $\Sing^{\Aff^{1}}(K_\bullet)$ is $\Aff^{1}$-local.
		\item $\Sing^{\Aff^{1}}(K_\bullet)\to K_\bullet$ is $\Aff^{1}$-weak equivalence.
	\end{enumerate}
\end{proof}
\begin{definition}
	The category of \defn{effective \'etale motivic sheaves} on $S$ is
	\[
		\cat{DA}^{\text{\'et},\text{eff}}(S,\Lambda) \defeq D_{\text{\'et}}(S) / \gen{I_{\Aff^{1},\text{\'et}}}.
	\] 
	Write $L_{\text{mot}}^{\text{eff}}$ for the associated localization functor.
\end{definition}
\begin{lemma}
	We have
	\[
		L_{\text{mot}}^{\text{eff}} = \underbrace{\dots \Sing^{\Aff^{1}}L_{\text{\'et}}\Sing^{\Aff^{1}}}_{\text{transfinie composition\dots}}
	\] 
\end{lemma}
\begin{definition}
	Let $X\in \Sm_S$. Define
	\[
		M_S^{\text{\'et},\text{eff}}(X) \defeq L_{\text{mot}}^{\text{eff}}\Lambda_{\text{\'et}}[X]\in \cat{DA}^{\text{eff},\text{\'et}}(S,\Lambda)
	\] 
	(effecive \'etale (? homological) motive/motivic sheaf of ?).
\end{definition}
We have
\[
M_S^{\text{\'et},\text{eff}}(X\times_S Y) \isoto M_S^{\text{\'et},\text{eff}}(X) \otimes M_S^{\text{\'et},\text{eff}}(Y).
\] 
\begin{proposition}[Artin-Shreier ${+\Lambda\left[ \frac{1}{p} \right] }$]
	Let $S$ be a $\F_p$-scheme, then
	\[
		\cat{DA}^{\text{\'et},\text{eff}}(S,\Lambda) \simto \cat{DA}^{\text{\'et},\text{eff}}\left( S,\Lambda\left[ \tfrac{1}{p} \right]  \right) .
	\] 
\end{proposition}
\begin{proof}
	We have the short-exact sequence
	\[
	\begin{tikzcd}
		0 \ar[r] & \Lambda / p\Lambda \ar[r] & \Ga \otimes \Lambda \ar[r,"\Fr-\id"] & \Ga \ar[r] & 0
	\end{tikzcd}
	\] 
	and hence an exact sequence
	\[
	\begin{tikzcd}
		\Lambda_{\text{\'et}}[\Ga] \otimes(\Ga\otimes \Lambda) \ar[r,"a_{\Ga}\otimes \id"] & \Ga\otimes \Ga\otimes \Lambda \ar[r,"m"] & \Ga\otimes \Lambda.
	\end{tikzcd}
	\] 
	(Some remark??) Thus
	\[
	L_{\Aff^{1}}(\Ga\otimes\Lambda) \isoto \Lambda(0)
	\] 
	and so
	\[
	L_{\text{mot}}^{\text{eff}}(\Gamma /p\Gamma) = 0.
	\]
\end{proof}

\emph{$\P^{1}$-stabilization:} Let $x\in X(S)$, we have
\[
M_S^{\text{eff}}(X) = \Lambda_S(0) \oplus M_{S}^{\text{eff}}(X,x).
\] 
\begin{definition}
	We define
	\[
		T \defeq M_S^{\text{eff}}(\P^{1},1)[-2]
	\] 
	which is equal to $\Lambda(?).$
\end{definition}
\begin{exercise}
	Any $x\in \P^{1}_S(S)$ gives the same decomposition.
\end{exercise}
We have a problem: $T$ is not $\oplus$-invertible.

\begin{definition}
	The category of \'etale motivic sheaves over $S$ is
	\[
		\DAet(S,\Lambda) \defeq \cat{DA}^{\text{\'et},\text{eff}}(S,\Lambda)[T^{\otimes -1}].
	\] 
\end{definition}
This is not a proper definition since we have not explained the construction. It's not clear what this means!

\emph{Spectra:}
\begin{definition}
	Let $\C$ be a closed, symmetric monoidal 1-category and $T$ be an object of $\C$. A \defn{$T$-prespectrum} is
	\[
	A = \left\{ (A_n,\sigma_n)_{n\in \N} | A_n\in \C, \sigma_n:T\otimes A_n\to A_{n+1} \right\} .
	\] 
	$A$ is a \defn{$T$-spectrum} if for all $n\in N$ 
	\[
	A_n \simto \iHom(T,A_{n+1}).
	\] 
	We write $\PSp_T(\C)$ and $\Sp_T(\C)$ for the $T$-prespectrum and $T$-spectrum respectively.
\end{definition}
The evaluation map
\[
\Ev_n(A) = A_n
\] 
has a left adjoint. We define
\[
\Sus^{n}(A)_m = \begin{cases}
	\emptyset & \text{if }m<n \\
	T^{\otimes (m-n)}\otimes A & \text{if }m>n
\end{cases}
\] 
and $\Sigma_T^{\infty}\defeq \Sus^{0}$ is the $\infty$-suspension functor.

\begin{proposition}
	Assume $\C$ is presentably, symmetrical monoidal. Then $\Sp_T(\C)\subset \PSp_T(\C)$ is a reflexive subcategory. $W_{\text{st}}$ is generated by
	\[
	\left\{ \Sus^{n+1}(T\otimes A) \to \Sus^{n}(A):n\in \N,A\in \C \right\} .
	\] 
\end{proposition}
\begin{definition}
	We define
	\[
		\DAet(S,\Lambda) \defeq \Sp_T\cat{DA}^{\text{eff},\text{\'et}}(S,\Lambda).
	\] 	
\end{definition}
(This definition is correct ``with $\infty$-categories''.)
We have
\begin{align*}
	M_S: \Sm_S &\longrightarrow \DAet(S,\Lambda) \\
	X &\longmapsto L_{(\Aff^{1},\text{\'et},?}\Sigma_T^{\infty}M_S^{\text{\'et},\text{eff}}(X)
.\end{align*}
\begin{remark}
	$M\in \DAet(S,\Lambda)$ is isomotphic to a stable $(\Aff^{1},\text{\'et})$-local (??)
	\[
		K_n\in \Ch(\Sh_{\text{\'et}}(\Sm_S,\Lambda)) + \sigma_n = \Lambda_{\text{\'et}}[\P^{1},1]\otimes K_n\to K_{n+1}
	\] 
	such that for all $X\in \Sm_S,i \in \Z$
	\begin{itemize}
		\item $\CH_{\text{\'et}}^{i}(X,K_n)\simto \CH_{\text{\'et}}^{i}(X\times_S\Aff^{1},K_n)$ 
		\item $\CH_{\text{\'et}}^{i}(X,K_n) \simto \CH_{\text{\'et}}^{i+2}(X\times_S(\P^{1},1),K_{n+1})$.
	\end{itemize}
\end{remark}

\subsection{Constructible motivic sheaves}
\begin{definition}
	We define \defn{constructible motivic sheaves}
	\begin{align*}
		\DAet_{\text{ct}} &= \left<M_S(X)(-n)|X\in \Sm_S,n\in \N \right>^{\text{d.f.}} \\
				  &\subset \DAet(S,\Lambda).
	\end{align*}
	and \defn{locally constructible motivic sheaves}
	\[
		\DAet_{\text{lct}}(S,\Lambda) \defeq \{M|\exists e:U\surjto S, e^{*}M\in \cat{DA}_{\text{ct}}\} .
	\] 
\end{definition}
There is a Betti realization for $S$ finite type over $\bb{C}$ 
\[
R_B: \DAet(S,\Lambda) \to D(S\an,\Lambda)
\] 
by the existence of relative homology and the universal property. If $X\in \Sm_S$ 
\[
R_B(M_S(X)) \isoto \CH_*^{\text{sing}}(X /S)
\] 
and
\[
R_B(\DAet_{\text{lct}}(S,\Lambda) \subset D^{b}_{\text{ct}}(S\an,\Lambda).
\] 

Another deep property is the \emph{rigidity theorem}. Define
\[
D_{\text{\'et}}(S,\Lambda) = D(\Sh_{\text{\'et}}(S,\Lambda))
\] 
and write
\[
\iota : (\Et_S, \text{\'et})\injto (\Sm_S,\text{\'et})
\] 
for the inclusion, then we get
\[
\iota_S^{*}:D_{\text{\'et}}(S,\Lambda) \to \DAet(S,\Lambda).
\] 
\begin{theorem}[Ayoub]
	Let $S$ be an excellent, Noetherian, finite dimensional, $\Lambda$-finite, with any prime invertible in $\Lambda$ or $\OO_S$. Then $\iota_S^{*}$ is an equivalence.
\end{theorem}

This procedure is very flexible and admits many \emph{variants}.
\begin{itemize}
	\item We can change the input. Instead of complexes we can work with presheaves of simplicial sets or $\infty$-groupoids.
	\item We can change the topology. Instead of \'etale use Nisnevich or use presheaves with transfers.
	\item Change the geometric context. For example, to rigid analytic motivic sheaves.
\end{itemize}

\begin{definition}
	The \defn{stable motivic homotopy category} over $S$ is
	\[
		\SH(S) \defeq \PSp_T(\PSh(\Sm_S,\cat{sSet}))[W^{-1}_{(\Aff^{1},\text{Nis},\P^{1})}].
	\] 
\end{definition}
Recall that the Nisnevich topology is between the Zariski and \'etale topologies. $\DAet(S)$ is the motivic version of $D(S\an,\Z)$ and $\SH(S)$ is the motivic version of sheaves of $S^{?}$-spectra on $S\an$. There is also $\cat{DM}(S,\Lambda)$ which are the Nisnevich-local motivic sheaves. Many important invariants of varieties only satisfy Nisnevich descent, but not \'etale descent; for example, $K$-theory or higher chow groups.

\section{Motives over a field}
Let $S=\Spec(k)$ and $\Lambda=\Q$. Define
\[
	\cat{DM}(k,\Q) \defeq \DAet(k,\Q).
\] 
The analogies you should have in mind are
\begin{itemize}
	\item $D(\Ind \cat{MHS}_{\Q})$,
	\item $D(\Ind \cat{Rep}_{\Q_l}^{\text{f.d.}}G_k)$.
\end{itemize}
Even though the construction was very formal there are some surprises. Define
\[
	\Q\left<i \right> \defeq \Q(i)[2i]
\] 
be pure Tate twists and $M\left<i \right>\defeq M\otimes \Q\left<i \right>$.
\begin{itemize}
	\item \emph{Projective bundle formula:} Let $E\to X$ be a vector bundle, then
		\begin{gather*}
			M(\P(E)) \isoto \bigoplus_{i=0}^{\rank E-1}M(X)\left<i \right> \\
			M(\P^{n}_l) = \Lambda(0)\oplus \Lambda\left<1 \right>\oplus \dots\oplus \Lambda\left<n \right>.
		\end{gather*}
	\item \emph{Gysin triangle:} Let $(c:Z\not\injto X)\in \Sm_k$, then we have
		\[
		\begin{tikzcd}
			M(X\setminus Z) \ar[r] & M(X) \ar[r] & M(Z)\left<c \right> \ar[r,"+"] & { }
		\end{tikzcd}
		\] 
	\item \emph{Smooth blow-up formula:}
		\[
		M(\Bl_Z(X)) \isoto M(X)\oplus \bigoplus_{i=1}^{c-1}M(Z)\left<i \right>.
		\] 
	\item \emph{Poincar\'e duality 1}: Let $X$ be smooth and projective over $k$, then $M(X)$ is \emph{dualizable} with
		\[
		M(X)^{\vee}\isoto M(X)\left<-\dim(X) \right>.
		\] 
\end{itemize}
We have $\cat{DM}(k,\Q)\isoto \Ind\cat{DM}_{\text{ct}}$.

From here on out
\[
	\cat{DM}(k,\Lambda) = \begin{cases}
		\DAet(k,\Lambda) & \Lambda\text{ a }\Q\text{-algebra} \\
		\cat{DM}(k,\Lambda) & \Lambda\text{ a }\Z\left[ \frac{1}{p} \right] \text{-algebra}.
	\end{cases}
\] 
For singular varieties $X\in \Sch_R^{\text{ft,sep}}$ we get $M(X)\in \cat{DM}(k,\Lambda)$. There are four theories
\begin{enumerate}[(i)]
	\item $M(X)$,
	\item Borel-Moore cohomoloy $M_{\mathrm{BM}}(X)$ (also denoted $M^{c}(X)$ in the literature),
	\item $M^{\text{coh}}(X)$,
	\item $M_{\text{c}}^{\text{coh}}(X)$.
\end{enumerate}

\emph{Localization:} Consider a closed immersion $Z\injto X$ and the open immersion $X\setminus Z\injto X$. We have
\[
\begin{tikzcd}
	M_{\text{BM}}(Z) \ar[r] & M_{\text{BM}}(X) \ar[r] & M_{\text{BM}}(?) \ar[r,"+"] & {} \\
	M^{\text{coh}}_{\text{c}}(X\setminus Z) \ar[r] & M^{\text{coh}}_{\text{c}}(X) \ar[r] & M^{\text{coh}}_{\text{c}}(Z) \ar[r,"+"] & {}
\end{tikzcd}
\]

\emph{Poincar\'e duality 2:} For $X\in \Sm_k$
\[
\begin{cases}
	M(X)^{\vee} \isoto M_{\text{BM}}(X)\left<-d \right> \\
	M^{\text{coh}}(X)^{\vee} \isoto M^{\text{coh}}(X)\left<d \right>.
\end{cases}
\] 
(??)

\subsection{Motivic cohomology and algebraic cycles}
\begin{definition}
	Let $X\in \Sm_k$, we define the \defn{Motivic cohomology groups}
	\begin{align*}
		\CH_{\text{mot}}^{p,q}(X) = \CH_{\text{mot}}^{p}(X,\Lambda(q)) &\defeq \Hom_{\cat{DM}(k,\Lambda)}(M(X), \Lambda(q)[p]) \\
									       &\isoto \Hom_{\cat{DM}(X,\Lambda)}(\Lambda_X(0), \Lambda_X(q)[p]).
	\end{align*}
	For $X\in \Sch_k^{\text{ft,sep}}$ define
	\[
		\CH_{p,q}^{\text{BM}} \defeq \Hom(\Lambda(q)[p], M_{\text{BM}}(X)).
	\] 
\end{definition}
\subsubsection{Weight 1 motivic cohomology}
\begin{lemma}
	We have
	\[
		M_S^{\text{eff}}(\Gm) \isoto \Lambda(0) \oplus \Lambda(1)[1].?
	\] 
\end{lemma}
\begin{proof}
	$\P^{1}=\Aff^{1}\cup \Aff^{1}$, so by Mayer-Vietoris we get
	\[
	\begin{tikzcd}
		M(\Gm) \ar[r] & M(\Aff)^{\oplus 2} \ar[r] & M(\P^{1}) \ar[r,"+"] & {}
	\end{tikzcd}
	\] 
	hence by $\Aff^{1}$-invariance
	\[
		M(\Gm,1) \isoto M(\P^{1},1)[-1].
	\] 
\end{proof}
The map $\alpha_{\Gm}:\Lambda_{\text{\'et}}[\Gm]\to \Gm\otimes \Lambda$ induces
\[
	\Lambda(1)[1] \xrightarrow{(*)} \Sigma_{T}^{\infty} (\Gm\otimes\Lambda).
\] 
\begin{theorem}\leavevmode
	\begin{enumerate}[1)]
		\item (*) is an isomorphism, so
			\[
				\Pic(s)\otimes \Lambda \xrightarrow{c_1} \CH_{\text{mot}}^{2,1}(S)
			\] 
		\item For $S$ regular
			\[
			\CH_{\text{mot}}^{n,1}(S) = \begin{cases}
				\OO_S^{\times }\otimes\Lambda & n=1 \\
				\Pic(S)\otimes\Lambda & n=2 \\
				0 & \text{otherwise}.
			\end{cases}
			\] 
	\end{enumerate}
\end{theorem}
\subsubsection{Higher Chow groups}
Let $\bb{\Delta}_{\text{alg},k}^{\bullet}\in (\Sm_k)^{\bb{\Delta}}$.
\begin{definition}
	Let $X\in \Sch_k^{\text{ft}}$ define
	\[
	\kz_n(X,r) \subseteq Z_n(X\times \bb{\Delta}_{\text{alg}}^{r})\otimes\Lambda
	\] 
	generated by integral subvarieties of dimension $n$ which intersect all faces properly.
\end{definition}
(Picture) We get $d:\kz_n(X,r)\to \kz_{n-1}(X,r-1)$ so $\kz_n(X,\bullet)$ is a \emph{Bloch cycle complex}.
(??)

\begin{theorem}[Voevodsky+\dots]
	Let $k$ be perfect, $X\in \Sch_k^{\text{ft,sep}}$ then
	\[
	\CH_{p,q}^{\text{BM}}(X)\isoto \Chow_q(X,p-2q,\Lambda).
	\] 
	If $X\in \Sm_k$ then
	\begin{align*}
		\CH^{p,q}_{\text{mot}}(X) &\isoto \Chow^{q}(X,2q-p,\Lambda) \\
		\CH_{\text{mot}}^{2n,n}(X) &\isoto \Chow^{n}(X,\Lambda).
	\end{align*}
\end{theorem}
(??)

\subsection{Examples (Tate)}
\begin{definition}
	Define
	\[
		\cat{DMT}(k,\Lambda) = \left<\Lambda(n) | n\in \Z \right>^{\text{df}}
	\] 
	the \defn{mixed Tate motives}. It constains $\bigoplus \Lambda\left<i \right>^{\oplus n_i}$ the \defn{pure Tate motives}.
\end{definition}
We have $M(\Aff^{n})=\Lambda(0)$ and $M_{\text{BM}}(\Aff^{n}) = \Lambda\left<n \right>$.
\begin{exercise}
	Show that
	\[
		M(\Aff^{n}\setminus \{0\}) \isoto \Lambda(0) \oplus \Lambda(n)[2n-1].
	\] 
\end{exercise}
\subsubsection{Cellular varieties}
\begin{definition}
	$X\in \Sch_k^{\text{ft}}$ is \defn{cellular} if there exists a closed subscheme $Z\injto X$ such that $X\setminus Z\simto \Aff_k^{i}$ and $Z$ is cellular.
\end{definition}
\begin{proposition}
	Suppose $X$ is cellular:
	\begin{enumerate}[a)]
		\item We have
			\[
			M_{\text{BM}}(X) \isoto \bigoplus_{i=0}^{d}\Lambda\left<i \right>^{n_i},
			\] 
			where $n_i$ is the number of cells of dimension $i$.
		\item If $X$ is also smooth
			\[
			M(X) \isoto \bigoplus_{j=0}^{d}\Lambda\left<j \right>^{m_j},
			\] 
			where $m_j$ is the number of cells of codimension $j$.
	\end{enumerate}
\end{proposition}
\begin{example}\leavevmode
	\begin{enumerate}[1)]
		\item Let $G$ be split reductive, $B\subset G$ be a Borel, then $G /B$ is cellular (Bruhat decomposition) and
			\[
			M(G /B) \isoto \bigoplus_{i\ge 0}\Lambda\left<i \right>^{n_i}
			\] 
			where $n_i$ is the number of $w\in W$ of length $i$.
		\item Let $X$ be quasiprojective and smooth (??)
	\end{enumerate}
\end{example}
\subsubsection{Reductive groups}
\begin{theorem}[Biglami]
	 If $G$ is split reductive, then
	 \[
		 M(G) \isoto \Sym^{*}\left( \bigoplus_{i\ge 1}\Lambda(i)[2?-i]^{\oplus n_i} \right) 
	 \] 
	 so by Chevalley
	 \[
		 R[\kg]^{G} = k[q_1,\dots,q_{?}]
	 \] 
	 where $\deg q_j = d_j$ and $n_i$ is the number of $j$ such that $d_j=i$.
\end{theorem}
\begin{example}
	We have
	\begin{align*}
		M(\GL_n) &= \Sym^{*}\left( \Lambda(1)[1] \oplus \Lambda(2)[3]\oplus \dots\oplus \Lambda(n)[2n-1] \right)  \\
		M(\SL_n) &= \times  (??)
	\end{align*}
\end{example}
\begin{exercise}
	What is $M(\Sp_{2n})$?
\end{exercise}
\subsection{Examples (non-Tate)}
\subsubsection{Curves}
\begin{proposition}
	Let $C$ be a smooth projective curve with a 0-cycle (with $\Lambda$-coefficients) of degree 1 (or if $\Lambda$ is a $\Q$-algebra)
	\[
		M(C) \isoto \Lambda(0) \oplus M_1(C) \oplus \Lambda \left<1 \right>.
	\] 
	If $g(C)>0$ then $M_n(C)\not\in \cat{DMT}(k,\Lambda)$.
\end{proposition}
\subsubsection{Commutative algebraic groups}
\begin{theorem}[?]
	We take $\Lambda=\Q$ and $G /k$ a smooth commutative group (e.g. a (semi-)abelian variety). Define
	\[
		M_1(G) \defeq \Sigma_T^{\infty}(G\otimes \Q) \in \cat{DM}(k,G).
	\] 
	Then
	\[
	M(G) \isoto \left( \bigoplus_{i=0}^{?}\Sym_i(M_1(G)) \right) \otimes M(?).
	\] 
\end{theorem}

\section{Six functor formalism}
\subsection{Betti sheaves}
\begin{definition}
	Define
	\begin{align*}
		D_B(-): \cat{Var}_{\bb{C}}\op &\longrightarrow \cat{TriCat}^{\otimes}\quad (\text{better }\cat{CAlg}(\PrL)) \\
		X &\longmapsto D\left( \Sh(X\an,\Lambda) \right) \\
		f &\longmapsto f^{*} = \LD f^{*}\quad \text{pullback}
	\end{align*}
\end{definition}
$D_B$ is symmetric monoidal
\[
f^{*}(F\otimes G) \isoto f^{*}(F)\otimes f^{*}(G) + \dots
\] 
(note that we write $\otimes = \otimes^{\mathbb{L}}$).
\begin{proposition}
	$(f^{*},f_*=\RD f_*)$ is an adjoint pair. And $D_B(X)$ is closed, i.e. there exists $\iHom(F,G)$.
\end{proposition}
\begin{definition}
	A \defn{sheaf theory} is a symmetric monoidal functor
	\[
		D(-):(\Sch_S^{\text{ft}})\op \to \cat{TriCat}^{\otimes} / \cat{CAlg}(\PrL)
	\] 
\end{definition}
So we have four functors $(\otimes,\iHom)$ and $(f^{*},f_*)$ which form adjoint pairs.
\begin{example}\leavevmode
	\begin{itemize}
		\item Derived categories of \'etale/$l$-adic sheaves.
		\item Dervied categories of (holonomic) $D$-modules.
		\item Derived categories of mixed Hodge modules.
		\item ??
		\item $D(\cat{QCoh}(-))$.
	\end{itemize}
\end{example}
Let $f:Y\to X$ be separated of finite type, then we have two functors $f_!:D_B(Y)\leftrightarrows D_B(Y):f^{!}$ and $f_!$ gives us relative cohomology with compact support.

These satisfy a bunch of natural transformations:
\begin{itemize}
	\item \emph{Base change:} Let
		\[
		\begin{tikzcd}
			Y' \ar[r,"\wt{f}"] \ar[d,"\wt{g}"] & X' \ar[d,"g"] \\ Y \ar[r,"f"] & X
		\end{tikzcd}
		\] 
		be Cartesian, then we get a natural transformaton $f^{*}g_*(-)\to \wt{g}_*\wt{f}^{*}(-)$.
	\item \emph{Projection:} We have a natural transformation
		\[
		f_*(-)\otimes - \to f_*(-\otimes f^{*}(-)).
		\] 
	\item \emph{K\"unneth:} We have the natural transformation
		\[
		f_*(-)\otimes g_*(-) \to (f\times g)_*(-\boxtimes_X -)
		\] 
		where $\boxtimes_X\defeq \pr_1^{*}(-)\otimes \pr_2^{*}(-)$.
\end{itemize}
\begin{theorem}
	Let $D=D_B$. Assume $g$ is proper the (BC) and (Proj) are isomorphisms. If $f$ is also proper then (K\"u) is also an isomorphism.
\end{theorem}
\begin{proposition}[Open base change]
	Assume $f$ is an open immersion. Then (BC) is an isomorphism.
\end{proposition}
\begin{definition}
	Let $f:Y\to X$ be separated of finite type and $F\in \Sh(X\an,\Lambda)$. Define
	\[
		(f_!F)(U) \defeq \left\{ s\in F(f^{-1}(U)) \middle| f|_{\Supp(s)}\text{ is proper} \right\} \subset (f_*F)(U)
	\] 
	is the \defn{pushforward with compact support}. We also write
	\[
	f_! \defeq \RD f_!:D(Y)\to D(X).
	\] 
\end{definition}
$f_!\to f_*$ is an isomorphism for $f$ proper (??).
\begin{lemma}
	Suppose $j:U\injto X$ is an open immersion.
	\begin{enumerate}[1)]
		\item $j_!:\Sh(U\an)\to \Sh(X\an)$ is ``extension by zero''
			\[
				(j_!F)_x = \begin{cases}
					F_x  & x\in U \\
					0 & \text{otherwise}.
				\end{cases}
			\] 
		\item $j_!$ is left adjoint to $j^{*}$.
		\item We have open BC: $f^{*}j_!\isoto \wt{j}_!\wt{f}^{*}$ and open Proj
			\[
			j_!(-\otimes j^{*}(-)) \isoto j_!(-) \otimes -.
			\] 
	\end{enumerate}
\end{lemma}
Let $f:Y\to X$ be a separated morphism of finite type, then there exists a Nagata compactification where $f$ factors as
\[
	Y \xhookrightarrow{j} \overline{Y} \xrightarrow{p} X
\] 
where $j$ is an open immersion and $p$ is proper. Then
\[
j_! \isoto p_!j_! \isoto p_*j_!.
\] 
\begin{theorem}
	(BC) We have $g^{*}f_! \simto \wt{f}_!\wt{g}^{*}$.

	(Proj) $f_!(-\otimes f^{*}(-)) \simto f_!(-)\otimes -$.
	
	(K\"u) $f_!(-)\otimes g_!(-) \simto (f\times g)_!(-\boxtimes -)$.
\end{theorem}
\begin{proposition}
	Let $f$ be a separated morphism of finite type. The functor $f_!:D_B(Y)\to D_B(X)$ commutes with all coproducts. So by the Adjoint Functor Theorem, $f_!$ has a right adjoint $f^!:D_B(X)\to D_B(Y)$ called the \defn{exceptional pullback}.
\end{proposition}
\begin{example}
	If $j$ is an open immersion (\'etale) then $j^{!}\isoto j^{*}$.
\end{example}
\begin{proposition}[Formal local duality]
	There is an isomorphism
	\[
	\iHom(f_!F,G_) \simto f_*\iHom(F,f^{!}G).
	\] 	
\end{proposition}
\begin{exercise}
	Prove this!	
\end{exercise}
\begin{example}
	Let $\pi:X\to \Spec(\bb{C})$, then
	\[
	\CH_c^{*}(X,\Q)^{\vee} \isoto \CH^{*}(X,\pi^{!}\Q).
	\] 
	To recover Poincar\'e duality, we need to compute $\pi^{!}\Q$ for $X$ smooth.
\end{example}
\begin{theorem}[Duality for smooth morphisms]
	Let $q:Y\to X$ be a separated morphism of finite type.
	\begin{enumerate}[1)]
		\item There is a canonical natural transformation
			\[
			\alpha_f:f^{!}\Lambda\otimes f^{*}(-) \to f^{!}(-).
			\] 
		\item Let $f$ be smooth separated of relative dimension $d$, then
			\begin{itemize}
				\item $\alpha_f$ is an isomorphism,
				\item $f^{!}\Lambda\isoto \Lambda\left<d \right>$.
			\end{itemize}
			(Better $\Lambda(1)\isoto\Lambda$.)
		\item If $f$ is smooth then $f^{*}$ has a left adjoint 
			\[
			f_\sharp = f_!\left<d \right>.
			\] 
	\end{enumerate}
\end{theorem}

\end{document}

