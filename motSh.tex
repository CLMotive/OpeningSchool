\documentclass[oneside]{amsart}

\usepackage[all]{xy}
\usepackage{tikz-cd}
\usepackage[T1]{fontenc}
\usepackage{xstring}
\usepackage{xparse}
\usepackage{xr-hyper}
\usepackage{xcolor}
\definecolor{brightmaroon}{rgb}{0.76, 0.13, 0.28}
\usepackage[linktocpage=true,colorlinks=true,hyperindex,citecolor=blue,linkcolor=brightmaroon]{hyperref}
\usepackage[nameinlink]{cleveref}
\usepackage[left=1.25in,right=1.25in,top=0.75in,bottom=0.75in]{geometry}
%\usepackage[charter,greekfamily=didot]{mathdesign}
%\usepackage{Baskervaldx}
\usepackage{amssymb}
\usepackage{stmaryrd}
\usepackage{mathrsfs}
\usepackage{mathpazo}
\linespread{1.05}

\usepackage[nobottomtitles]{titlesec}
\usepackage{marginnote}
\usepackage{enumerate}
\usepackage{longtable}
\usepackage{aurical}
\usepackage{microtype}

\newtheoremstyle{ega-env-style}%
{}{}{\rmfamily}{}{\bfseries}{.}{ }{\thmnote{(#3)}}%

\newtheoremstyle{ega-thm-env-style}%
{}{}{\itshape}{}{\bfseries}{. --- }{ }{\thmname{#1}\thmnumber{ #2}\thmnote{ (#3)}}%

\newtheoremstyle{ega-defn-env-style}%
{}{}{\rmfamily}{}{\bfseries}{. --- }{ }{\thmname{#1}\thmnumber{ #2}\thmnote{ (#3)}}%

\theoremstyle{ega-env-style}
\newtheorem*{env}{---}

\theoremstyle{ega-thm-env-style}
\newtheorem{theorem}{Theorem}[subsection]
\newtheorem{proposition}{Proposition}[subsection]
\newtheorem{lemma}{Lemma}[subsection]
\newtheorem{corollary}{Corollary}[subsection]
\newtheorem{stheorem}{Theorem}[section]
\newtheorem{slemma}[stheorem]{Lemma}
\newtheorem{skey}[stheorem]{Key Formula}
\newtheorem{conjecture}{Conjecture}[subsection]

\theoremstyle{ega-defn-env-style}
\newtheorem*{definition}{Definition}
\newtheorem{example}{Example}[subsection]
\newtheorem*{examples}{Examples}
\newtheorem*{remark}{Remark}
\newtheorem*{remarks}{Remarks}
\newtheorem*{notation}{Notation}
\newtheorem*{exercise}{Exercise}
\newtheorem*{properties}{Properties}
\newtheorem*{consequences}{Consequences}
\newtheorem*{note}{Note}
\newtheorem*{fact}{Fact}

\makeatletter
\def\l@subsection{\@tocline{2}{0pt}{2.5pc}{2.2pc}{}}
\def
\section{\@startsection{section}{1}%
	\z@{.7\linespacing\@plus\linespacing}{.5\linespacing}%
{\normalfont\bfseries\Large\scshape\centering}}
\renewcommand{\@seccntformat}[1]{%
	\ifnum\pdfstrcmp{#1}{section}=0\textsection\fi%
\csname the#1\endcsname.~}
\makeatother

\def\mathcal{\mathscr}
%% Fonts

\def\sh{\mathcal}                   % sheaf font
\def\bb{\mathbf}                    % bold font
\def\cat{\mathsf}                   % category font

%% Font Letters

\def\CL{\mathcal{L}}
\def\OO{\mathcal{O}}
\def\CX{\mathcal{X}}

%% Cohomology

\def\CH{\mathrm{H}}                 % cohomology H
\def\CHH{\check{\HH}}               % Čech cohomology H
\def\RD{\mathrm{R}}                 % right derived R
\def\LD{\mathrm{L}}                 % left derived L
\def\dual#1{{#1}^\vee}              % dual
\def\Tor{\operatorname{Tor}}        % Tor
\def\Ext{\operatorname{Ext}}        % Ext
\def\HdR{\mathrm{H}_{\mathrm{dR}}}  % de Rham cohomology
\def\Zc{\underline{\mathbb{Z}}}     % constant sheaf with integer coeffs

%% Categories

\def\A{\cat{A}}                     % category A, usually abelian
\def\C{\cat{C}}                     % category C
\def\op{^\cat{op}}                  % opposite category
\def\Set{\cat{Sets}}                % category of sets
\def\Grp{\cat{Gps}}                 % category of groups
\def\Alg{\cat{Alg}}                 % category of algebras
\def\QCoh{\cat{QCoh}}               % category of quasicoherent sheaves
\def\supertilde{{\,\widetilde{\,}\,}}   % use \supertilde instead of ^\sim
\def\Hom{\operatorname{Hom}}        % morphisms
\def\End{\operatorname{End}}        % endomorphisms
\def\Aut{\operatorname{Aut}}        % automorphisms
\def\ker{\operatorname{ker}}        % kernel
\def\img{\operatorname{im}}         % image
\def\coker{\operatorname{coker}}    % cokernel
\def\pr{\operatorname{pr}}          % projection
\def\EssIm{\operatorname{EssIm}}    % essential image
\DeclareMathOperator*{\colim}{colim}   % colimit


%% Schemes

\def\Proj{\operatorname{Proj}}      % Proj
\def\Supp{\operatorname{Supp}}      % support
\def\Spec{\operatorname{Spec}}      % Spec
\def\Spf{\operatorname{Spf}}        % formal Spec
\def\Aff{\mathbb A}                 % affine space
\def\P{\mathbb{P}}                  % projective space
\def\Pic{\operatorname{Pic}}        % Picard group

%% Standard Operators

\def\codim{\operatorname{codim}}    % codimension
\def\id{\operatorname{id}}          % identity

%% Arrows
\renewcommand{\to}{\mathchoice{\longrightarrow}{\rightarrow}{\rightarrow}{\rightarrow}}
\newcommand{\from}{\mathchoice{\longleftarrow}{\leftarrow}{\leftarrow}{\leftarrow}}
\let\mapstoo\mapsto
\renewcommand{\mapsto}{\mathchoice{\longmapsto}{\mapstoo}{\mapstoo}{\mapstoo}}
\def\isoto{\simeq}                  % isomorphism
\def\simto{\xrightarrow{\sim}}      % isomorphism arrow
\def\surjto{\twoheadrightarrow}     % surjetion
\def\injto{\hookrightarrow}         % injection

%% Under/Over Accents

\newcommand{\wh}[1]{\widehat{#1}}   % hat
\newcommand{\wt}[1]{\widetilde{#1}}    % tilde
\def\ul{\underline}                % underline

%% Spaces

\def\Z{\mathbb Z}                  % integers
\def\Q{\mathbb Q}                  % rationals
\def\N{\mathbb{N}}                 % naturals

%% Groups

\def\GL{\bb{GL}}                   % general linear group
\def\SL{\bb{SL}}                   % special linear group
\def\det{\operatorname{det}}       % determinant

%% Sub/Superscripts

\def\et{^\text{\'et}}              % \'etale
\def\an{^{\text{an}}}              % analytic

%% Misc

\newcommand{\defn}[1]{\textbf{#1}}  % definition highlighting
\def\defeq{:=}                     % definition equation
\def\eps{\varepsilon}              % correct epsilon
\newcommand{\gen}[1]{\left\langle\!\left\langle #1 \right\rangle\!\right\rangle}



\def\shHom{\sh{H}\textup{\kern-2.2pt{\Fontauri\slshape om}}}   % sheaf Hom
\def\shProj{\sh{P}\textup{\kern-2.2pt{\Fontauri\slshape roj}}} % sheaf Proj
\def\shExt{\sh{E}\textup{\kern-2.2pt{\Fontauri\slshape xt}}}   % sheaf Ext
\def\shGr{\sh{G}\textup{\kern-2.2pt{\Fontauri\slshape r}}}     % sheaf Gr
\def\shDer{\sh{D}\,\textup{\kern-2.2pt{\Fontauri\slshape er}}} % sheaf Der
\def\shDiff{\sh{D}\,\textup{\kern-2.2pt{\Fontauri\slshape if{}f}}\,} % sheaf Diff
\def\shHomcont{\sh{H}\textup{\kern-2.2pt{\Fontauri\slshape om.\,cont}}}   % sheaf Hom.cont
\def\shAut{\sh{A}\textup{\kern-2.2pt{\Fontauri\slshape ut}}}   % sheaf Aut
\def\shSym{\sh{S}\textup{\kern-2.2pt{\Fontauri\slshape ym}}}   % sheaf Sym

\makeatletter
\newcommand{\cbigoplus}{\DOTSB\cbigoplus@\slimits@}
\newcommand{\cbigoplus@}{\mathop{\widehat{\bigoplus}}}
\makeatother


\usepackage{stmaryrd}
\usepackage[nameinlink]{cleveref}

\newcommand{\SC}{\mathcal{C}}
\newcommand{\OO}{\mathcal{O}}
\def\coker{\operatorname{coker}}
\def\bL{\mathbf{L}}
\def\CT{\mathcal{T}}
\def\Aut{\operatorname{Aut}}
\def\wt{\widetilde}
\def\ko{\mathfrak{o}}
\def\kt{\mathfrak{t}}
\newcommand{\CX}{\mathcal{X}}
\def\coker{\operatorname{coker}}
\def\bL{\mathbf{L}}
\def\wh{\widehat}
\def\CI{\mathcal{I}}
\def\et{^\text{\'et}}
\def\sep{^\text{sep}}
\def\pr{\operatorname{pr}}
\def\img{\operatorname{im}}
\def\Sym{\operatorname{Sym}}
\def\CG{\mathcal{G}}
\def\N{\mathbb{N}}
\def\th{^{\mathrm{th}}}
\def\CN{\mathcal{N}}
\def\Inf{\operatorname{Inf}}
\def\rank{\operatorname{rank}}
\def\Pic{\operatorname{Pic}}
\def\i{\operatorname{inv}}
\def\Diag{\operatorname{Diag}}
\def\G{\mathbb{G}}
\def\End{\operatorname{End}}
\def\det{\operatorname{det}}
\def\SL{\bb{SL}}
\def\Rep{\operatorname{Rep}}
\def\Repf{\operatorname{Rep}^{\text{fd}}}
\def\Res{\operatorname{Res}}
\def\Ind{\operatorname{Ind}}
\def\CL{\mathcal{L}}
\def\Zc{\underline{\mathbb{Z}}}
\def\DAet{\operatorname{DA}^{\text{\'et}}}
\def\SH{\operatorname{SH}}
\def\Sh{\operatorname{Sh}}
\def\PSh{\operatorname{PSh}}
\def\Cone{\operatorname{Cone}}
\def\A{\cat{A}}
\def\Ch{\operatorname{Ch}}
\def\EssIm{\operatorname{EssIm}}
\def\colim{\operatorname{colim}}
\def\ft{^{\text{(f.t)}}}
\def\an{^{\text{an}}}
\def\P{\mathbb{P}}

\title{Motivic Sheaves}
\author{Micha{\l} Mruga{\l}a}

\begin{document}
\maketitle
\textbf{Plan:}
\begin{enumerate}[L1:]
	\item Motivation, construction, examples.
	\item Six-functor formalism.
	\item Motivic $t$-structures and weight structures.
	\item $\infty$-categorical methods.
\end{enumerate}
\section{Motivation from GRT and cohomology}
\subsection{Cohomology and sheaves for representation theory}
\emph{Question:} How do you construct interesting representations?

\emph{Answer:}
\begin{enumerate}[1)]
	\item Find interesting actions.
	\item Linearlize them.
\end{enumerate}

\begin{example}
	Let $K$ be a compact Lie group. The action of $K$ on itself gives us an action of $K$ on $L^2(K)$ with respect to a Haar measure. The Peter-Weyl theorem says that
	\[
	L^2(K) \simeq \bigoplus_{\pi\text{ unitary}}\pi^{\oplus dim(\pi)}.
	\]
	``Lie theory $\subset $ algebraic geometry''. Reductive groups are algebraic groups with many associated varieties with group actions: flag varieties\dots
\end{example}
The linearization we consider in this course are the many types of cohomology theories.

\begin{example}[Borel-Weil-Bott]
	Let $T\subset B\subset G$ be a reductive group over $\bb{C}$. Let $\lambda\in X^{\vee}(T)$ such that there exists $w\in W$ with $w*\lambda = w(\lambda+\rho)-\rho>0$ (where $\rho = (1 /2)\sum_{\alpha\in \Phi^{+}}\alpha$). Then
	\[
		R\Gamma(G /B,L_\lambda) \simeq \pi_{w*\lambda}[-\ell(w)]
	\] 
	where $\ell(w)$ is the length of $w$.
\end{example}
Cohomology fits in the wider context of sheaf theory. If $T$ is a locally contractible topological space, then
\[
	H^{n}_{\text{sing}}(T,\Z) \simeq H^{n}(T,\Zc_T) \simeq R^{n}(\pi_T)_*(\Zc_T)	
\]
where $\pi_T$ is the morphism $\pi_T:T\to \text{pt}$ with
\[
R\pi_{T*}:D(T,\Z) \to D(\Z) \simeq D(*,\Z)
.\] 

\emph{Facts:} Cohomology (singular with $\Q$-coefficients) of algebraic varieties over $\bb{C}$ is \emph{\textbf{very}} special.
\begin{itemize}
	\item There is a weight filtration.
	\item There is a mixed hodge structure.
\end{itemize}
Sheaves on complex algebraic varieties are also very special:
\begin{itemize}
	\item Perverse sheaves;
	\item Decomposition theorem;
	\item Mixed Hodge modules.
\end{itemize}

This leads to great success stories in GRT:
\begin{itemize}
	\item Springer theory;
	\item Kazdhan-Lustig theory;
	\item geometric Satake\dots
\end{itemize}

\subsection{From sheaves to motivic sheaves}
There are situations which can't be directly studied using these tools:
\begin{itemize}
	\item Representation theory of reductive groups over other fields/rings/schemes.
	\item Modular/integral representation theory.
	\item $q$-deformations, quantum groups, canonical bases.
\end{itemize}
These can be attacked using:
\begin{itemize}
	\item $l$-adic sheaves,
	\item sheaves cohomology with $\Z$-coefficients,
	\item $K$-theory.
\end{itemize}
Motivic sheaves will give us a unified perspective.

\emph{Motivic dream:} There should exist universal cohomology/sheaf theories such that
\begin{enumerate}[1)]
	\item they unify and ``explain'' the special stracture in cohomology/sheaves;
	\item they are of algebro-geometric nature;
	\item they ``explain'' the realization of algebraic cycles and algebraic $K$-theory.
\end{enumerate}

\section{Construction of $\DAet$ and $\SH$ (Morel-Voevodsky)}
\subsection{Triangulated categories and localization}
\begin{definition}
	A triangulated category is the data:
	\begin{itemize}
		\item an additive category $\C$,
		\item an automorphism $\Sigma=(-)[1]:\C\simto \C$,
		\item a collection of distinguished triangles
			\[
				A \to B \to C \to A[1]
			.\] 
	\end{itemize}
	Satisfying
	\begin{itemize}
		\item shifted distinguished triangles are distinguished up to isomorphism,
		\item for all $f:A\to B$ there exists
			\[
				A \xrightarrow{f} B \to \Cone(f) \xrightarrow{+}
			\] 
			unique up to isomorphism and functorial,
		\item 
			\begin{equation*}
			\begin{tikzcd}
				A \ar[r] \ar[d,"f","\simeq"'] & B \ar[r] \ar[d,"g","\simeq"'] & C \ar[d,dashed] \ar[r] & A[1] \ar[d,"{f[1]}","\simeq"'] \\
				A' \ar[r] & B' \ar[r] & C' \ar[r] & A'[1]
			\end{tikzcd}
			\end{equation*}	
		\item (??)
	\end{itemize}
\end{definition}

\begin{example}
	Let $\A$ be an abelian category, $\Ch(\A)$ be the abelian category of chain complexes in $\A$. We define $(A[1])_n = A_{n-1}$. Given $f:A_\bullet\to B_\bullet$ the maping cone is given by
	\[
		\Cone(f)_n = A_{n-1} \oplus B_n,\qquad d_n = \begin{pmatrix} -d_{n-1}^{A} & 0 \\ f & d_n^{B} \end{pmatrix} 
	.\] 
\end{example}
\begin{definition}
	$f:A_\bullet\to B_\bullet$ is a \emph{quasi-isomorphism} if for all $n\in \Z$, the map $H_n(A_\bullet)\simeq H_n(B_\bullet)$ is an isomorphism.
\end{definition}
\begin{definition}
	$D(\A)$ is defined as the localization of $\Ch(\A)$ by quasi-isomorphisms.
\end{definition}

Now we consider reflexive localizaitons (1-categorical ones lead to triangulated and $\infty$-categorical ones).
\begin{definition}
	Let $\C$ be a 1-category.
	\begin{enumerate}[1)]
		\item $\C'\subset \C$ is reflexive if $\iota:\C'\to \C$ has a left adjoint.
		\item $L_W:\C\to \C[W^{-1}]$ is reflexive if $L_W$ has a right adjoint.
	\end{enumerate}
\end{definition}

\begin{lemma}[Reflexive subcategories are the same thing as reflexive localizations]
	\begin{enumerate}[a)]
		\item Let $\C'\subset \C$ be reflexive, $L:\C\to \C'$ be the left adjoint to $\iota$. Define 
			\[
				W_L=\{f:L(f)\text{ is an isomorphism}\}.
			\]
			Then $\C'\simeq \C[W_L^{-1}]$ and $L\simeq L_{(W_L)}$.
		\item If $L$ is a reflexive localization, then $\iota$ is fully faithful and $\iota:\C[W^{-1}] \simto \EssIm(\iota)\subset \C$.
	\end{enumerate}
\end{lemma}

\begin{definition}
	Let $S\subset \C$ be a collection of morphisms.
	\begin{enumerate}[a)]
		\item $A\in \C$ is \emph{$S$-local} if for all $f:B\to C$ in $S$ 
			\[
			\Hom_\C(C,A) \simto \Hom_\C(B,A).
			\] 
		\item $f:B\to C$ is an \emph{$S$-equivalence} if for all $S$-local $A$ 
			\[
			\Hom_{\C}(C,A) \simto \Hom_\C(B,A).
			\] 
	\end{enumerate}
\end{definition}
\begin{lemma}
	If $L:\C \rightleftarrows \C':\iota$ is a reflexive localizaton, $W_L$ as before, then
	\begin{itemize}
		\item $\iota$ gives an isomorphism between $\C'$ and $W_L$-local objects.
		\item $W_L$ are the $W_L$-equivalences.
	\end{itemize}
\end{lemma}
\begin{definition}
	Let $\cat{D}$ be a triangulated category with all small products.
	\begin{itemize}
		\item Let $\kappa$ be a regular  cardinal (for example $\kappa=\aleph_0$). Then $A\in \cat{D}$ is \emph{$\kappa$-small}/\emph{$\kappa$-compact} if and only if
			\[
			\colim_{I' \subset I, |I'|<\kappa} \Hom\left( A,\bigoplus_{I'}B_i \right) \simto \Hom\left( A, \bigoplus_I B_i \right) 
			.\] 
		\item \emph{Compact} means $\aleph_0$-small. $A$ is compact if and only if
			\[
			\bigoplus_I \Hom(A,B_i) \simto \Hom \left( A,\bigoplus_{I}B_i \right) 
			.\] 
		\item $\cat{D}$ is \emph{presentable}/\emph{well-generated} if and only if there exist $\kappa$ and a set $S\subset \cat{D}$ of $\kappa$-small objects which generate $\cat{D}$:
			\[
				\forall B\in \cat{D},\left( \forall A\in S, \Hom(A,B)=0 \right) \implies B = 0
			.\] 
		\item $\cat{D}$ is \emph{compactly generated} if it is $\aleph_0$-presentable.
	\end{itemize}
\end{definition}

\begin{definition}
	$\cat{E}\subset \cat{D}$ is \emph{localizing} if it is
	\begin{itemize}
		\item triangulated,
		\item stable under coproducts,
		\item thick (stable under subobjects and subquotients).
	\end{itemize}
\end{definition}

\begin{theorem}[Adjoint Functor Theorem]
	Let $\cat{D},\cat{D}'$ be triangulated categories with all coproducts, $F:\cat{D}\to \cat{D}'$ be a triangulated functor and $\cat{D}$ be presentable. Then $F$ admits a right adjoint if and only if $F$ preserves all coproducts.
\end{theorem}
\begin{corollary}[Verdier Localization]
	Let $\cat{D}$ be a presentable category and $\cat{E}$ be a localizing subcategory. Define
	\[
		\cat{D} / \cat{E} = D[W_{\cat{E}}^{-1}],\quad W_{\cat{E}} = \{f:\Cone(f)\in \cat{E}\} 
	.\] 
	Then $\cat{D} \to \cat{D} /\cat{E}$ is a reflexive localization.
\end{corollary}
Let $S\subset \cat{D}$ be a subset of objects, then $\ll S \gg$ is the smallest subcategory containing $S$ such that $\cat{D} /\ll S\gg$ is a reflexive localization.

Let's get some inspiration from Betti homology, i.e. singular homology on complex algebraic varieties. Let $X\in \cat{Var}_{\bb{C}}\ft$, then we get
\[
	C_*^{\text{sing}}(X(\bb{C}),\Z)\in D(\Z)
.\]
This comes with some data:
\begin{enumerate}[(a)]
	\item 
		\begin{itemize}
			\item $D(\Z)$ has a symmetric monoidal structure: $\otimes^{\Z}$,
			\item (K\"unneth) $C_*(X\times Y) \simeq C_*(X)\otimes^{\Z}C_*(Y)$.
		\end{itemize}
\end{enumerate}
which satisfies sproperties:
\begin{enumerate}[(a)]
	\setcounter{enumi}{1}
	\item ($\Aff^{1}$-homotopy invariance) $C_*(X\times \Aff^{1})\simto C_*(X)$ ($(\Aff^{1})\an=\bb{C}$ is contractible).
	\item (Mayer-Vietoris sequence) Let $X=U \cup V$ be a Zariski cover, then
		\[
			C_*(U\cap V) \to C_*(U) \oplus C_*(V) \to C_*(X) \xrightarrow{\partial} C_*(U \cap  V)[1]
		.\] 
	\item (\'Etale descent) Let $U\to X$ be \'etale surjective. Define
		\[
			\check{C}_n(U /X) = U^{n+1}
		.\] 
		Then $\check{C}_\bullet(U /X)$ is a simplicial scheme, so $C_*(\check{C}_\bullet(U /X))$ is a simplicial complex of abelian groups and $C(C_*(\check{C}_\bullet(U /X)))$ is a double complex. (??)

		Concretely we have a descent spectral sequence which gives us ($U=U \cup V$) Mayer Vietoris.
	\item ($\P^{1}$-stabilization)
		\begin{align*}
			C_*(\P_{\bb{C}}^{1}) &\simeq C_*(\text{pt}) \oplus \wt{C}_*(\P_{\bb{C}}^{1}) \\
					     &\simeq \Z[0] \oplus \Z(1)[2]
		.\end{align*}
		$\Z(1)$ is $\oplus$-invertible.
\end{enumerate}
Smooth varieties play a special role:
\begin{itemize}
	\item Poincar\'e duality
	\item Gysin sequences
	\item $C_*(-)$ also satisfies ``$h$-descent'', so $C_*(-)$ is ``determined'' by $C_*(-)_{|(?)}$.
\end{itemize}
There is an associated sheaf theory:
\[
	D_B(-):\cat{Var}_{\bb{C}}\to \cat{TriCat}^{\otimes},\quad X\mapsto D_B(X) = D(\Sh(X\an,\Z))
.\] 

\emph{Sketch of $\DAet$:} Let $S$ be a base scheme.
\begin{itemize}
	\item Start with
		\[
		\begin{cases}
			D(\PSh(\cat{Sm}_S, \Z)) = D_{\PSh}(S) \\
			\Z[-]: \cat{Sm}_S \to D_{\PSh}(S)
		\end{cases}
		.\] 
	\item Impose $\Aff^{1}$-invariane, \'etale descent, and $\P^{1}$-stability. This will give us $\DAet(S,\Z)$ and $M_S(-):\cat{Sm}_S\to \DAet(S,\Z)$.
\end{itemize}
The surprise is that the result satisfies many \emph{other} properties of singular (co)homology and derived categories of sheaves on complex varieties which can largely be packaged into the six-functor formalism. The result is closely related to algebraic cycles/higher Chow groups and algebraic $K$-theory.

\end{document}

