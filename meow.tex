\documentclass[oneside]{amsart}

\usepackage[all]{xy}
\usepackage{tikz-cd}
\usepackage[T1]{fontenc}
\usepackage{xstring}
\usepackage{xparse}
\usepackage{xr-hyper}
\usepackage{xcolor}
\definecolor{brightmaroon}{rgb}{0.76, 0.13, 0.28}
\usepackage[linktocpage=true,colorlinks=true,hyperindex,citecolor=blue,linkcolor=brightmaroon]{hyperref}
\usepackage[nameinlink]{cleveref}
\usepackage[left=1.25in,right=1.25in,top=0.75in,bottom=0.75in]{geometry}
%\usepackage[charter,greekfamily=didot]{mathdesign}
%\usepackage{Baskervaldx}
\usepackage{amssymb}
\usepackage{stmaryrd}
\usepackage{mathrsfs}
\usepackage{mathpazo}
\linespread{1.05}

\usepackage[nobottomtitles]{titlesec}
\usepackage{marginnote}
\usepackage{enumerate}
\usepackage{longtable}
\usepackage{aurical}
\usepackage{microtype}

\newtheoremstyle{ega-env-style}%
{}{}{\rmfamily}{}{\bfseries}{.}{ }{\thmnote{(#3)}}%

\newtheoremstyle{ega-thm-env-style}%
{}{}{\itshape}{}{\bfseries}{. --- }{ }{\thmname{#1}\thmnumber{ #2}\thmnote{ (#3)}}%

\newtheoremstyle{ega-defn-env-style}%
{}{}{\rmfamily}{}{\bfseries}{. --- }{ }{\thmname{#1}\thmnumber{ #2}\thmnote{ (#3)}}%

\theoremstyle{ega-env-style}
\newtheorem*{env}{---}

\theoremstyle{ega-thm-env-style}
\newtheorem{theorem}{Theorem}[subsection]
\newtheorem{proposition}{Proposition}[subsection]
\newtheorem{lemma}{Lemma}[subsection]
\newtheorem{corollary}{Corollary}[subsection]
\newtheorem{stheorem}{Theorem}[section]
\newtheorem{slemma}[stheorem]{Lemma}
\newtheorem{skey}[stheorem]{Key Formula}
\newtheorem{conjecture}{Conjecture}[subsection]

\theoremstyle{ega-defn-env-style}
\newtheorem*{definition}{Definition}
\newtheorem{example}{Example}[subsection]
\newtheorem*{examples}{Examples}
\newtheorem*{remark}{Remark}
\newtheorem*{remarks}{Remarks}
\newtheorem*{notation}{Notation}
\newtheorem*{exercise}{Exercise}
\newtheorem*{properties}{Properties}
\newtheorem*{consequences}{Consequences}
\newtheorem*{note}{Note}
\newtheorem*{fact}{Fact}

\makeatletter
\def\l@subsection{\@tocline{2}{0pt}{2.5pc}{2.2pc}{}}
\def
\section{\@startsection{section}{1}%
	\z@{.7\linespacing\@plus\linespacing}{.5\linespacing}%
{\normalfont\bfseries\Large\scshape\centering}}
\renewcommand{\@seccntformat}[1]{%
	\ifnum\pdfstrcmp{#1}{section}=0\textsection\fi%
\csname the#1\endcsname.~}
\makeatother

\def\mathcal{\mathscr}
%% Fonts

\def\sh{\mathcal}                   % sheaf font
\def\bb{\mathbf}                    % bold font
\def\cat{\mathsf}                   % category font

%% Font Letters

\def\CL{\mathcal{L}}
\def\OO{\mathcal{O}}
\def\CX{\mathcal{X}}

%% Cohomology

\def\CH{\mathrm{H}}                 % cohomology H
\def\CHH{\check{\HH}}               % Čech cohomology H
\def\RD{\mathrm{R}}                 % right derived R
\def\LD{\mathrm{L}}                 % left derived L
\def\dual#1{{#1}^\vee}              % dual
\def\Tor{\operatorname{Tor}}        % Tor
\def\Ext{\operatorname{Ext}}        % Ext
\def\HdR{\mathrm{H}_{\mathrm{dR}}}  % de Rham cohomology
\def\Zc{\underline{\mathbb{Z}}}     % constant sheaf with integer coeffs

%% Categories

\def\A{\cat{A}}                     % category A, usually abelian
\def\C{\cat{C}}                     % category C
\def\op{^\cat{op}}                  % opposite category
\def\Set{\cat{Sets}}                % category of sets
\def\Grp{\cat{Gps}}                 % category of groups
\def\Alg{\cat{Alg}}                 % category of algebras
\def\QCoh{\cat{QCoh}}               % category of quasicoherent sheaves
\def\supertilde{{\,\widetilde{\,}\,}}   % use \supertilde instead of ^\sim
\def\Hom{\operatorname{Hom}}        % morphisms
\def\End{\operatorname{End}}        % endomorphisms
\def\Aut{\operatorname{Aut}}        % automorphisms
\def\ker{\operatorname{ker}}        % kernel
\def\img{\operatorname{im}}         % image
\def\coker{\operatorname{coker}}    % cokernel
\def\pr{\operatorname{pr}}          % projection
\def\EssIm{\operatorname{EssIm}}    % essential image
\DeclareMathOperator*{\colim}{colim}   % colimit


%% Schemes

\def\Proj{\operatorname{Proj}}      % Proj
\def\Supp{\operatorname{Supp}}      % support
\def\Spec{\operatorname{Spec}}      % Spec
\def\Spf{\operatorname{Spf}}        % formal Spec
\def\Aff{\mathbb A}                 % affine space
\def\P{\mathbb{P}}                  % projective space
\def\Pic{\operatorname{Pic}}        % Picard group

%% Standard Operators

\def\codim{\operatorname{codim}}    % codimension
\def\id{\operatorname{id}}          % identity

%% Arrows
\renewcommand{\to}{\mathchoice{\longrightarrow}{\rightarrow}{\rightarrow}{\rightarrow}}
\newcommand{\from}{\mathchoice{\longleftarrow}{\leftarrow}{\leftarrow}{\leftarrow}}
\let\mapstoo\mapsto
\renewcommand{\mapsto}{\mathchoice{\longmapsto}{\mapstoo}{\mapstoo}{\mapstoo}}
\def\isoto{\simeq}                  % isomorphism
\def\simto{\xrightarrow{\sim}}      % isomorphism arrow
\def\surjto{\twoheadrightarrow}     % surjetion
\def\injto{\hookrightarrow}         % injection

%% Under/Over Accents

\newcommand{\wh}[1]{\widehat{#1}}   % hat
\newcommand{\wt}[1]{\widetilde{#1}}    % tilde
\def\ul{\underline}                % underline

%% Spaces

\def\Z{\mathbb Z}                  % integers
\def\Q{\mathbb Q}                  % rationals
\def\N{\mathbb{N}}                 % naturals

%% Groups

\def\GL{\bb{GL}}                   % general linear group
\def\SL{\bb{SL}}                   % special linear group
\def\det{\operatorname{det}}       % determinant

%% Sub/Superscripts

\def\et{^\text{\'et}}              % \'etale
\def\an{^{\text{an}}}              % analytic

%% Misc

\newcommand{\defn}[1]{\textbf{#1}}  % definition highlighting
\def\defeq{:=}                     % definition equation
\def\eps{\varepsilon}              % correct epsilon
\newcommand{\gen}[1]{\left\langle\!\left\langle #1 \right\rangle\!\right\rangle}



\def\shHom{\sh{H}\textup{\kern-2.2pt{\Fontauri\slshape om}}}   % sheaf Hom
\def\shProj{\sh{P}\textup{\kern-2.2pt{\Fontauri\slshape roj}}} % sheaf Proj
\def\shExt{\sh{E}\textup{\kern-2.2pt{\Fontauri\slshape xt}}}   % sheaf Ext
\def\shGr{\sh{G}\textup{\kern-2.2pt{\Fontauri\slshape r}}}     % sheaf Gr
\def\shDer{\sh{D}\,\textup{\kern-2.2pt{\Fontauri\slshape er}}} % sheaf Der
\def\shDiff{\sh{D}\,\textup{\kern-2.2pt{\Fontauri\slshape if{}f}}\,} % sheaf Diff
\def\shHomcont{\sh{H}\textup{\kern-2.2pt{\Fontauri\slshape om.\,cont}}}   % sheaf Hom.cont
\def\shAut{\sh{A}\textup{\kern-2.2pt{\Fontauri\slshape ut}}}   % sheaf Aut
\def\shSym{\sh{S}\textup{\kern-2.2pt{\Fontauri\slshape ym}}}   % sheaf Sym

\makeatletter
\newcommand{\cbigoplus}{\DOTSB\cbigoplus@\slimits@}
\newcommand{\cbigoplus@}{\mathop{\widehat{\bigoplus}}}
\makeatother


\def\CC{\mathcal C}
\def\CS{\mathcal S}
\def\CT{\mathcal T}
\def\Sing{\operatorname{Sing}}

\title{Infinity Categories}
\author{Lecturer: Pramod Achar,\quad Typesetter: Micha{\l} Mruga{\l}a}

\begin{document}
\maketitle

\section{Simplicial sets}

\begin{definition}
	The \defn{simplex category} $\bb{\Delta}$ is the category of
	\begin{itemize}
		\item finite non-empty totally ordered sets
		\item order-preserving maps.
	\end{itemize}
\end{definition}
\begin{notation}
	$[n]=\{0<1<2<\dots<n\} $ for $n\in \Z_{\ge 0}$.
\end{notation}
Every object in $\bb{\Delta}$ is (uniquely) isomorphic to some $[n]$.
\begin{definition}
	A \defn{simplicial set} is a functor
	\[
		\CS:\bb{\Delta}\op \to \Set
	\] 
\end{definition}
\begin{notation}
	$\CS_n \defeq \CS([n])$, call this the \defn{set of $n$-simplices} of $\CS$. 0-simplices are called \defn{vertices}, 1-simplices are called \defn{edges}.
\end{notation}
\begin{example}
	Let $C$ be a set. Let $\ul{C}:\bb{\Delta}\op\to \Set$ be the constant functor:
	\begin{gather*}
		\ul{C}_n = C\quad \forall n, \\
		\ul{C}(\alpha) = \id \quad \forall \alpha:[m]\to [n] \text{ in }\bb{\Delta}.
	\end{gather*}
	This is called a \defn{discrete simplicial set}.
\end{example}
\begin{definition}
	Let $\CS$ be a simplicial set. Given $\alpha:[n]\to [n-1]$ we get $\CS(\alpha):\CS_{n-1}\to \CS_n$. The $n$-simplices in the eimage are called \defn{degenerate} simplices, i.e. $\sigma$ is degenerate if there is an $\alpha$ such that $\sigma\in \img(\CS(\alpha))$.
\end{definition}
\begin{lemma}
	A simplicial set is discrete if and only if for all $n\ge 1$ all $n$-simplices are degenerate.
\end{lemma}
\begin{exercise}
	Prove it.
\end{exercise}
\begin{example}
	Let $(P,\ge )$ be a poset. Define a simplicial set $N(P,\le )$ called the \defn{nerve} of $(P,\le )$ by
	\[
		N(P,\le )_k = \{\text{chains }p_0\le p_1\le \dots\le p_k: p_i\in P\} 
	\]
	where a chain is a totally ordered subset.
\end{example}
\begin{exercise}
	Finish the definition. Which simplices are degenerate?
\end{exercise}
\begin{example}[``Standard $n$-simplex'']
	The \defn{standard $n$-simplex} is
	\[
		\Delta^{n} \defeq N([n]).
	\] 
	(Pictures)
\end{example}
\begin{note}
	For $j\in [n]$, we get a subsimplicial set
	\[
		N([n]\setminus \{j\} ) \subset \Delta^{n}
	\] 
	isomorphic to $\Delta^{n-1}$ called the $j\th$ \defn{face} of $\Delta^{n}$. (Picture)
\end{note}
\begin{example}[Horns]
	Let $n\ge 0$ and $0\le j\le n$, define the \defn{horn}
	\[
		\Lambda^{n}_j \defeq \begin{array}{c}
			\text{subsimplicial set of }\Delta^{n} = N([n]) \\
			\text{consisting of chains }p_0\le p_1\le \dots\le p_k \\
			\text{such that } \{p_0,\dots,p_k\} \not\supset [n] \setminus \{j\} . 
		\end{array}
		(Pictures)
	\] 
\end{example}
\begin{example}[$(n-1)$-sphere $\partial\Delta^{n}$]
	We define the \defn{$(n-1)$-sphere}
	\[
		\partial \Delta^{n} \defeq \begin{array}{c}
			\text{subsimplicial set of }\Delta^{n} \\
			\text{chains }p_0\le \dots \le p_k\text{ such that }\{p_0\le \dots\le p_k\} \neq [n]
		\end{array}
	\] 
\end{example}
\begin{example}[Products]
	Let $\CS,\CT$ be simplicial sets. We define their \defn{product} $\CS\times \CT$ as
	\[
		(\CS\times \CT)_k = \CS_k \times \CT_k.
	\] 
	(Picture)
\end{example}
\begin{example}
	Let $\C$ be an ordinary category. We define its \defn{nerve} $N(\C)$ as
	\[
		N(\C)_k \defeq \left\{ \begin{array}{c}
			\text{composable sequences of morphisms} \\
			X_0 \xrightarrow{f_1} X_1\xrightarrow{f_2}X_2\to \dots\xrightarrow{f_k} X_k
		\end{array}\right\} . 
	\] 
\end{example}
\begin{example}
	Let $X$ be a topological space. The \defn{singular simplicial set} of $X$ is defined as
	\[
	\Sing(X)_k = \{\text{continuous maps }|\Delta^{k}|\to X\} ,
	\] 
	where $|\Delta^{k}|$ is the standard $k$-simplex
	\[
		|\Delta^{k}| = \left\{ (x_0,\dots,x_k)\in \bb{R}^{k+1} \middle| x_i\ge 0, \sum x_i=1 \right\} .
	\] 
\end{example}
\begin{exercise}
	What does this do to the morphisms in $\bb{\Delta}$?
\end{exercise}
\begin{definition}
	A \defn{Kan complex} is a simplicial set $X$ such that for every diagram
	\[
	\begin{tikzcd}
		\Lambda^{n}_j \ar[r,"\text{any map}"] \ar[d,hook] & X \\
		\Delta^{n} \ar[ur,dashed]
	\end{tikzcd}
	\] 
	we can fill the dashed arrow. This is called an \defn{extension problem}. If the arrow exists we say that the extension problem \defn{admits a solution}.
\end{definition}
Daniel Kan discovered Kan complexes in 1958. The \emph{key fact} is that $\Sing(X)$ \emph{is} a Kan complex. The theme from 1958 to today is that Kan complexes are a ``combinatorial model'' for algebraic topology which allows us to do homotopy theory.

\begin{definition}
	Let $X$ be a Kan complex and $\CS$ be any simplicial set. Two maps $f,g:\CS\to X$ are said to be \defn{homotopic} if there exists a map $H:\CS\times \Delta^{1}\to X$ such that
	\[
	H|_{\CS\times \{0\} } = f,\quad H|_{\CS\times \{1\} } = g.
	\] 
\end{definition}
\begin{lemma}
	This is an equivalence relation.
\end{lemma}
\begin{proof}
	Omitted, tricky for an exercise. This requires $X$ to be a Kan complex.
\end{proof}
\begin{definition}
	Let $X$ be a Kan complex and $x_0$ be a vertex of $X$. Let 
	\[
	\text{Loops}_{x_0} = \{\text{maps }\gamma:\Delta^{n}\to X\text{ such that }\gamma|_{\partial\Delta^{n}}\text{ is the constant map to }x_0\}. 
	\] 
	We say $\gamma,\gamma'\in \text{Loops}_{x_0}$ are \defn{relatively homotopic} (\defn{rel. homotopic}) if there exists $H:\Delta^{n}\times \Delta^{1}\to X$ such that
	\[
	H|_{\Delta^{n}\times \{0\} } = \gamma,\quad H|_{\Delta^{n}\times \{1\} } = \gamma',\quad H|_{\partial \Delta^{n}\times \Delta^{1}} = \text{const. map to }x_0.
	\] 
	Define
	\[
		\pi_n(X,x_0) \defeq \frac{\text{Loops}_{x_0}}{\text{rel. homotopy}}.
	\] 
\end{definition}
\begin{fact}
	For $n\ge 1$, $\pi_n(X,x_0)$ is a group. For $n\ge 2$, $\pi_n(X,x_0)$ is abelian.
\end{fact}
\begin{definition}
	An \defn{$\infty$-category} (or \defn{quasi-category}) is a simplicial set $\CC$ such that any extension problem
	\[
	\begin{tikzcd}
		\Lambda_{j}^{n} \ar[r] \ar[d,hook] & \CC \\
		\Delta^{n} \ar[ur, dashed]
	\end{tikzcd}
	\] 
	with $0<j<n$ (\defn{inner horns}) admits a solution. (Picture) An $\infty$-category is also called a \defn{weak Kan complex}.
\end{definition}
\begin{lemma}
	Let $\C$ be an ordinary category, then $N(\C)$ is an $\infty$-category.
\end{lemma}
\emph{Digression:} Let $I^{n}$ be the simplicial set consisting of $n$ consecutive 1-simplices (\defn{$n$-spine}) (Picture). A naive alternative definition is: $\CC$ is an infinity category if every
\[
\begin{tikzcd}
	I^{n} \ar[r] \ar[d,hook] & \CC \\
	\Delta^{n} \ar[ur]
\end{tikzcd}
\] 
has a solution. This is WRONG (but its wrongness is subtle), even though $N(\text{ord. cat.})$ satisfy this. There is a book by Markus Land ``Introduction to $\infty$-categories'' which explores this. The definition of $\infty$-categories was introduced by Boardman-Vogt in 1972. Joyal started generalizing results from ordinary category theory to $\infty$-categories in 2006. Lurie is largely responsible for how well this notion is developed in modern literature.

\begin{remark}
	Having a unique solution to the lifting problem characterizes nerves of ordinary categories.
\end{remark}
\begin{definition}
	Let $\CC$ be an $\infty$-category. An \defn{object} is a vertex. A \defn{morphism} is an edge. An \defn{identity morphism} is a degenerate edge. Say that $h$ is \emph{a} \defn{composition} of $g$ and $f$ if there exists a 2-simplex such that (Picture).
\end{definition}
\begin{remark}
	Compositions are NOT unique in $\infty$-categories.
\end{remark}
\begin{example}[$\infty$-categories] \leavevmode
	\begin{enumerate}[1)]
		\item Topological spaces $\cat{Top}$.
			\begin{itemize}
				\item Objects are topological spaces.
				\item Morphisms are continuous maps.
				\item A 2-simplex is a (not necessarily commutative) diagram
					\[
					\begin{tikzcd}
						& X_1 \ar[dr,"g"] \\
						X_0 \ar[ur,"f"] \ar[rr,"h"] & & X_2
					\end{tikzcd}
					\] 
					\emph{and} a homotopy $H:X_0\times [0,1]\to X_2$ from $gf$ to $h$.
				\item A 3-simplex is a diagram
					\[
					\begin{tikzcd}
						& X_0 \ar[dl] \ar[dr] \ar[dd] \\
						X_1 \ar[rr] \ar[dr] & & X_3 \\
								    & X_2 \ar[ur]
					\end{tikzcd}
					\]
					with continuous maps $f_{ij}:X_i\to X_j$ for $i<j$, homotopies $T_{ijk}X_i\times [0,1]\to X_k$ from $f_{jk}\circ f_{ij}$ to $f_{ik}$, and $H:X_0\times [0,1]^2\to X_3$ (\defn{higher homotopy}) such that $H|_{\text{bdry}}$ is
					\[
					\begin{tikzcd}
						(0,0) \ar[r,"T_{123}f_{01}"] \ar[d,"f_{23}T_{012}"'] & (0,1) \ar[d,"T_{013}"] \\
						(1,0) \ar[r,"T_{023}"'] & (1,0)
					\end{tikzcd}
					\] 
			\end{itemize}
		\item The $\infty$-category of ordinary categories $\cat{Cat}_1$.
			\begin{itemize}
				\item Objects are ordinary categories.
				\item Morphisms are functors. 
				\item A 2-simplex is a (not necessarily commutative) diagram
					\[
					\begin{tikzcd}
						& X_1 \ar[dr,"g"] \\
						X_0 \ar[ur,"f"] \ar[rr,"h"] & & X_2
					\end{tikzcd}
					\] 
					\emph{and} a natural isomorphism $T:g\circ f \simto h$.
				\item A 3-simplex is a diagram
					\[
					\begin{tikzcd}
						& X_0 \ar[dl] \ar[dr] \ar[dd] \\
						X_1 \ar[rr] \ar[dr] & & X_3 \\
								    & X_2 \ar[ur]
					\end{tikzcd}
					\]
					where $f_{ij}$ are functors and $T_{ijk}$ are natural isomorphism such that
					\[
					\begin{tikzcd}
						\bullet \ar[r,"T_{123}f_{01}"] \ar[d,"f_{23}T_{012}"'] & \bullet \ar[d,"T_{013}"] \\
						\bullet \ar[r,"T_{023}"'] & \bullet
					\end{tikzcd}
					\] 
					commutes
			\end{itemize}
	\end{enumerate}
\end{example}
A source of $\infty$-categories are
\begin{itemize}
	\item ordinary categories with an equivalence relation on morphisms,
	\item ordinary categories with inverting some morphisms.
\end{itemize}

\end{document}
