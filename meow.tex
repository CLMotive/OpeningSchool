\documentclass[oneside]{amsart}

\usepackage[all]{xy}
\usepackage{tikz-cd}
\usepackage[T1]{fontenc}
\usepackage{xstring}
\usepackage{xparse}
\usepackage{xr-hyper}
\usepackage{xcolor}
\definecolor{brightmaroon}{rgb}{0.76, 0.13, 0.28}
\usepackage[linktocpage=true,colorlinks=true,hyperindex,citecolor=blue,linkcolor=brightmaroon]{hyperref}
\usepackage[nameinlink]{cleveref}
\usepackage[left=1.25in,right=1.25in,top=0.75in,bottom=0.75in]{geometry}
%\usepackage[charter,greekfamily=didot]{mathdesign}
%\usepackage{Baskervaldx}
\usepackage{amssymb}
\usepackage{stmaryrd}
\usepackage{mathrsfs}
\usepackage{mathpazo}
\linespread{1.05}

\usepackage[nobottomtitles]{titlesec}
\usepackage{marginnote}
\usepackage{enumerate}
\usepackage{longtable}
\usepackage{aurical}
\usepackage{microtype}

\newtheoremstyle{ega-env-style}%
  {}{}{\rmfamily}{}{\bfseries}{.}{ }{\thmnote{(#3)}}%

\newtheoremstyle{ega-thm-env-style}%
  {}{}{\itshape}{}{\bfseries}{. --- }{ }{\thmname{#1}\thmnumber{ #2}\thmnote{ (#3)}}%

\newtheoremstyle{ega-defn-env-style}%
  {}{}{\rmfamily}{}{\bfseries}{. --- }{ }{\thmname{#1}\thmnumber{ #2}\thmnote{ (#3)}}%

\theoremstyle{ega-env-style}
\newtheorem*{env}{---}

\theoremstyle{ega-thm-env-style}
\newtheorem{theorem}{Theorem}[subsection]
\newtheorem{proposition}{Proposition}[subsection]
\newtheorem{lemma}{Lemma}[subsection]
\newtheorem{corollary}{Corollary}[subsection]
\newtheorem{stheorem}{Theorem}[section]
\newtheorem{slemma}[stheorem]{Lemma}
\newtheorem{skey}[stheorem]{Key Formula}

\theoremstyle{ega-defn-env-style}
\newtheorem*{definition}{Definition}
\newtheorem{example}{Example}[subsection]
\newtheorem*{examples}{Examples}
\newtheorem*{remark}{Remark}
\newtheorem*{remarks}{Remarks}
\newtheorem*{notation}{Notation}
\newtheorem*{exercise}{Exercise}
\newtheorem*{properties}{Properties}
\newtheorem*{consequences}{Consequences}
\newtheorem*{note}{Note}
\newtheorem*{fact}{Fact}

\makeatletter
\def\l@subsection{\@tocline{2}{0pt}{2.5pc}{2.2pc}{}}
\def\section{\@startsection{section}{1}%
  \z@{.7\linespacing\@plus\linespacing}{.5\linespacing}%
  {\normalfont\bfseries\Large\scshape\centering}}
\renewcommand{\@seccntformat}[1]{%
  \ifnum\pdfstrcmp{#1}{section}=0\textsection\fi%
  \csname the#1\endcsname.~}
\makeatother

\def\mathcal{\mathscr}
%% Fonts

\def\sh{\mathcal}                   % sheaf font
\def\bb{\mathbf}                    % bold font
\def\cat{\mathsf}                   % category font

%% Font Letters

\def\CA{\mathcal{A}}
\def\CE{\mathcal{E}}
\def\CF{\mathcal{F}}
\def\CG{\mathcal{G}}
\def\CL{\mathcal{L}}
\def\OO{\mathcal{O}}
\def\CX{\mathcal{X}}
\def\b1{\mathbb{1}}

%% Cohomology

\def\CH{\mathrm{H}}                 % cohomology H
\def\CHH{\check{\HH}}               % Čech cohomology H
\def\RD{\mathrm{R}}                 % right derived R
\def\LD{\mathrm{L}}                 % left derived L
\def\dual#1{{#1}^\vee}              % dual
\def\Tor{\operatorname{Tor}}        % Tor
\def\Ext{\operatorname{Ext}}        % Ext
\def\HdR{\mathrm{H}_{\mathrm{dR}}}  % de Rham cohomology
\def\Zc{\underline{\mathbb{Z}}}     % constant sheaf with integer coeffs

%% Categories

\def\A{\cat{A}}                     % category A, usually abelian
\def\C{\cat{C}}                     % category C
\def\op{^\cat{op}}                  % opposite category
\def\Set{\cat{Sets}}                % category of sets
\def\Grp{\cat{Gps}}                 % category of groups
\def\Alg{\cat{Alg}}                 % category of algebras
\def\QCoh{\cat{QCoh}}               % category of quasicoherent sheaves
\def\supertilde{{\,\widetilde{\,}\,}}   % use \supertilde instead of ^\sim
\def\Map{\operatorname{Map}}        % morphisms (any category)
\def\Hom{\operatorname{Hom}}        % morphisms (additive category)
\def\iHom{\underline{\Hom}}         % interal hom
\def\End{\operatorname{End}}        % endomorphisms
\def\Aut{\operatorname{Aut}}        % automorphisms
\def\ker{\operatorname{ker}}        % kernel
\def\img{\operatorname{im}}         % image
\def\coker{\operatorname{coker}}    % cokernel
\def\pr{\operatorname{pr}}          % projection
\def\EssIm{\operatorname{EssIm}}    % essential image
\DeclareMathOperator*{\colim}{colim}   % colimit

\def\DAet{\cat{DA}\et}              
\def\SH{\cat{SH}}                   
\def\Sh{\cat{Sh}}                   % category of sheaves
\def\PSh{\cat{PSh}}                 % category of presheaves

%% Schemes

\def\Proj{\operatorname{Proj}}      % Proj
\def\Supp{\operatorname{Supp}}      % support
\def\Spec{\operatorname{Spec}}      % Spec
\def\Spf{\operatorname{Spf}}        % formal Spec
\def\Aff{\mathbb A}                 % affine space
\def\P{\mathbb{P}}                  % projective space
\def\Pic{\operatorname{Pic}}        % Picard group

%% Standard Operators

\def\codim{\operatorname{codim}}    % codimension
\def\id{\operatorname{id}}          % identity

%% Arrows
\renewcommand{\to}{\mathchoice{\longrightarrow}{\rightarrow}{\rightarrow}{\rightarrow}}
\newcommand{\from}{\mathchoice{\longleftarrow}{\leftarrow}{\leftarrow}{\leftarrow}}
\let\mapstoo\mapsto
\renewcommand{\mapsto}{\mathchoice{\longmapsto}{\mapstoo}{\mapstoo}{\mapstoo}}
\def\isoto{\simeq}                  % isomorphism
\def\simto{\xrightarrow{\sim}}      % isomorphism arrow
\def\surjto{\twoheadrightarrow}     % surjetion
\def\injto{\hookrightarrow}         % injection

%% Under/Over Accents

\newcommand{\wh}[1]{\widehat{#1}}   % hat
\newcommand{\wt}[1]{\widetilde{#1}}    % tilde
\def\ul{\underline}                % underline

%% Spaces

\def\Z{\mathbb Z}                  % integers
\def\Q{\mathbb Q}                  % rationals
\def\N{\mathbb N}                 % naturals
\def\H{\mathcal H}

%% Groups

\def\GL{\bb{GL}}                   % general linear group
\def\SL{\bb{SL}}                   % special linear group
\def\Sp{\bb{Sp}}
\def\det{\operatorname{det}}       % determinant

%% Sub/Superscripts

\def\et{^\text{\'et}}              % \'etale
\def\an{^{\text{an}}}              % analytic

%% Misc

\newcommand{\defn}[1]{\textbf{#1}}  % definition highlighting
\def\defeq{:=}                     % definition equation
\def\eps{\varepsilon}              % correct epsilon
\newcommand{\gen}[1]{\left\langle\!\left\langle #1 \right\rangle\!\right\rangle}



\def\shHom{\sh{H}\textup{\kern-2.2pt{\Fontauri\slshape om}}}   % sheaf Hom
\def\shProj{\sh{P}\textup{\kern-2.2pt{\Fontauri\slshape roj}}} % sheaf Proj
\def\shExt{\sh{E}\textup{\kern-2.2pt{\Fontauri\slshape xt}}}   % sheaf Ext
\def\shGr{\sh{G}\textup{\kern-2.2pt{\Fontauri\slshape r}}}     % sheaf Gr
\def\shDer{\sh{D}\,\textup{\kern-2.2pt{\Fontauri\slshape er}}} % sheaf Der
\def\shDiff{\sh{D}\,\textup{\kern-2.2pt{\Fontauri\slshape if{}f}}\,} % sheaf Diff
\def\shHomcont{\sh{H}\textup{\kern-2.2pt{\Fontauri\slshape om.\,cont}}}   % sheaf Hom.cont
\def\shAut{\sh{A}\textup{\kern-2.2pt{\Fontauri\slshape ut}}}   % sheaf Aut
\def\shSym{\sh{S}\textup{\kern-2.2pt{\Fontauri\slshape ym}}}   % sheaf Sym

\makeatletter
\newcommand{\cbigoplus}{\DOTSB\cbigoplus@\slimits@}
\newcommand{\cbigoplus@}{\mathop{\widehat{\bigoplus}}}
\makeatother


\def\CC{\mathcal C}
\def\CS{\mathcal S}
\def\CT{\mathcal T}
\def\Sing{\operatorname{Sing}}
\def\map{\operatorname{map}}
\def\Mor{\operatorname{Mor}}
\def\1b{\mathbb{1}}

\title{Infinity Categories}
\author{Lecturer: Pramod Achar,\quad Typesetter: Micha{\l} Mruga{\l}a}

\begin{document}
\maketitle

\section{Simplicial sets}

\begin{definition}
	The \defn{simplex category} $\bb{\Delta}$ is the category of
	\begin{itemize}
		\item finite non-empty totally ordered sets
		\item order-preserving maps.
	\end{itemize}
\end{definition}
\begin{notation}
	$[n]=\{0<1<2<\dots<n\} $ for $n\in \Z_{\ge 0}$.
\end{notation}
Every object in $\bb{\Delta}$ is (uniquely) isomorphic to some $[n]$.
\begin{definition}
	A \defn{simplicial set} is a functor
	\[
		\CS:\bb{\Delta}\op \to \Set
	\] 
\end{definition}
\begin{notation}
	$\CS_n \defeq \CS([n])$, call this the \defn{set of $n$-simplices} of $\CS$. 0-simplices are called \defn{vertices}, 1-simplices are called \defn{edges}.
\end{notation}
\begin{example}
	Let $C$ be a set. Let $\ul{C}:\bb{\Delta}\op\to \Set$ be the constant functor:
	\begin{gather*}
		\ul{C}_n = C\quad \forall n, \\
		\ul{C}(\alpha) = \id \quad \forall \alpha:[m]\to [n] \text{ in }\bb{\Delta}.
	\end{gather*}
	This is called a \defn{discrete simplicial set}.
\end{example}
\begin{definition}
	Let $\CS$ be a simplicial set. Given $\alpha:[n]\to [n-1]$ we get $\CS(\alpha):\CS_{n-1}\to \CS_n$. The $n$-simplices in the image are called \defn{degenerate} simplices, i.e. $\sigma$ is degenerate if there is an $\alpha$ such that $\sigma\in \img(\CS(\alpha))$.
\end{definition}
\begin{lemma}
	A simplicial set is discrete if and only if for all $n\ge 1$ all $n$-simplices are degenerate.
\end{lemma}
\begin{exercise}
	Prove it.
\end{exercise}
\begin{example}
	Let $(P,\ge )$ be a poset. Define a simplicial set $N(P,\le )$ called the \defn{nerve} of $(P,\le )$ by
	\[
		N(P,\le )_k = \{\text{chains }p_0\le p_1\le \dots\le p_k: p_i\in P\} 
	\]
	where a chain is a totally ordered subset.
\end{example}
\begin{exercise}
	Finish the definition. Which simplices are degenerate?
\end{exercise}
\begin{example}[``Standard $n$-simplex'']
	The \defn{standard $n$-simplex} is
	\[
		\Delta^{n} \defeq N([n]).
	\] 
	(Pictures)
\end{example}
\begin{note}
	For $j\in [n]$, we get a subsimplicial set
	\[
		N([n]\setminus \{j\} ) \subset \Delta^{n}
	\] 
	isomorphic to $\Delta^{n-1}$ called the $j\th$ \defn{face} of $\Delta^{n}$. (Picture)
\end{note}
\begin{example}[Horns]
	Let $n\ge 0$ and $0\le j\le n$, define the \defn{horn}
	\[
		\Lambda^{n}_j \defeq \begin{array}{c}
			\text{subsimplicial set of }\Delta^{n} = N([n]) \\
			\text{consisting of chains }p_0\le p_1\le \dots\le p_k \\
			\text{such that } \{p_0,\dots,p_k\} \not\supset [n] \setminus \{j\} . 
		\end{array}
		(Pictures)
	\] 
\end{example}
\begin{example}[$(n-1)$-sphere $\partial\Delta^{n}$]
	We define the \defn{$(n-1)$-sphere}
	\[
		\partial \Delta^{n} \defeq \begin{array}{c}
			\text{subsimplicial set of }\Delta^{n} \\
			\text{chains }p_0\le \dots \le p_k\text{ such that }\{p_0\le \dots\le p_k\} \neq [n]
		\end{array}
	\] 
\end{example}
\begin{example}[Products]
	Let $\CS,\CT$ be simplicial sets. We define their \defn{product} $\CS\times \CT$ as
	\[
		(\CS\times \CT)_k = \CS_k \times \CT_k.
	\] 
	(Picture)
\end{example}
\begin{example}
	Let $\C$ be an ordinary category. We define its \defn{nerve} $N(\C)$ as
	\[
		N(\C)_k \defeq \left\{ \begin{array}{c}
			\text{composable sequences of morphisms} \\
			X_0 \xrightarrow{f_1} X_1\xrightarrow{f_2}X_2\to \dots\xrightarrow{f_k} X_k
		\end{array}\right\} . 
	\] 
\end{example}
\begin{example}
	Let $X$ be a topological space. The \defn{singular simplicial set} of $X$ is defined as
	\[
	\Sing(X)_k = \{\text{continuous maps }|\Delta^{k}|\to X\} ,
	\] 
	where $|\Delta^{k}|$ is the standard $k$-simplex
	\[
		|\Delta^{k}| = \left\{ (x_0,\dots,x_k)\in \bb{R}^{k+1} \middle| x_i\ge 0, \sum x_i=1 \right\} .
	\] 
\end{example}
\begin{exercise}
	What does this do to the morphisms in $\bb{\Delta}$?
\end{exercise}
\begin{definition}
	A \defn{Kan complex} is a simplicial set $X$ such that for every diagram
	\[
	\begin{tikzcd}
		\Lambda^{n}_j \ar[r,"\text{any map}"] \ar[d,hook] & X \\
		\Delta^{n} \ar[ur,dashed]
	\end{tikzcd}
	\] 
	we can fill the dashed arrow. This is called an \defn{extension problem}. If the arrow exists we say that the extension problem \defn{admits a solution}.
\end{definition}
Daniel Kan discovered Kan complexes in 1958. The \emph{key fact} is that $\Sing(X)$ \emph{is} a Kan complex. The theme from 1958 to today is that Kan complexes are a ``combinatorial model'' for algebraic topology which allows us to do homotopy theory.

\begin{definition}
	Let $X$ be a Kan complex and $\CS$ be any simplicial set. Two maps $f,g:\CS\to X$ are said to be \defn{homotopic} if there exists a map $H:\CS\times \Delta^{1}\to X$ such that
	\[
	H|_{\CS\times \{0\} } = f,\quad H|_{\CS\times \{1\} } = g.
	\] 
\end{definition}
\begin{lemma}
	This is an equivalence relation.
\end{lemma}
\begin{proof}
	Omitted, tricky for an exercise. This requires $X$ to be a Kan complex.
\end{proof}
\begin{definition}
	Let $X$ be a Kan complex and $x_0$ be a vertex of $X$. Let 
	\[
	\text{Loops}_{x_0} = \{\text{maps }\gamma:\Delta^{n}\to X\text{ such that }\gamma|_{\partial\Delta^{n}}\text{ is the constant map to }x_0\}. 
	\] 
	We say $\gamma,\gamma'\in \text{Loops}_{x_0}$ are \defn{relatively homotopic} (\defn{rel. homotopic}) if there exists $H:\Delta^{n}\times \Delta^{1}\to X$ such that
	\[
	H|_{\Delta^{n}\times \{0\} } = \gamma,\quad H|_{\Delta^{n}\times \{1\} } = \gamma',\quad H|_{\partial \Delta^{n}\times \Delta^{1}} = \text{const. map to }x_0.
	\] 
	Define
	\[
		\pi_n(X,x_0) \defeq \frac{\text{Loops}_{x_0}}{\text{rel. homotopy}}.
	\] 
\end{definition}
\begin{fact}
	For $n\ge 1$, $\pi_n(X,x_0)$ is a group. For $n\ge 2$, $\pi_n(X,x_0)$ is abelian.
\end{fact}
\begin{definition}
	An \defn{$\infty$-category} (or \defn{quasi-category}) is a simplicial set $\CC$ such that any extension problem
	\[
	\begin{tikzcd}
		\Lambda_{j}^{n} \ar[r] \ar[d,hook] & \CC \\
		\Delta^{n} \ar[ur, dashed]
	\end{tikzcd}
	\] 
	with $0<j<n$ (\defn{inner horns}) admits a solution. (Picture) An $\infty$-category is also called a \defn{weak Kan complex}.
\end{definition}
\begin{lemma}
	Let $\C$ be an ordinary category, then $N(\C)$ is an $\infty$-category.
\end{lemma}
\emph{Digression:} Let $I^{n}$ be the simplicial set consisting of $n$ consecutive 1-simplices (\defn{$n$-spine}) (Picture). A naive alternative definition is: $\CC$ is an infinity category if every
\[
\begin{tikzcd}
	I^{n} \ar[r] \ar[d,hook] & \CC \\
	\Delta^{n} \ar[ur]
\end{tikzcd}
\] 
has a solution. This is WRONG (but its wrongness is subtle), even though $N(\text{ord. cat.})$ satisfy this. There is a book by Markus Land ``Introduction to $\infty$-categories'' which explores this. The definition of $\infty$-categories was introduced by Boardman-Vogt in 1972. Joyal started generalizing results from ordinary category theory to $\infty$-categories in 2006. Lurie is largely responsible for how well this notion is developed in modern literature.

\begin{remark}
	Having a unique solution to the lifting problem characterizes nerves of ordinary categories.
\end{remark}
\begin{definition}
	Let $\CC$ be an $\infty$-category. An \defn{object} is a vertex. A \defn{morphism} is an edge. An \defn{identity morphism} is a degenerate edge. Say that $h$ is \emph{a} \defn{composition} of $g$ and $f$ if there exists a 2-simplex such that (Picture).
\end{definition}
\begin{remark}
	Compositions are NOT unique in $\infty$-categories.
\end{remark}
\begin{example}[$\infty$-categories] \leavevmode
	\begin{enumerate}[1)]
		\item Topological spaces $\cat{Top}$.
			\begin{itemize}
				\item Objects are topological spaces.
				\item Morphisms are continuous maps.
				\item A 2-simplex is a (not necessarily commutative) diagram
					\[
					\begin{tikzcd}
						& X_1 \ar[dr,"g"] \\
						X_0 \ar[ur,"f"] \ar[rr,"h"] & & X_2
					\end{tikzcd}
					\] 
					\emph{and} a homotopy $H:X_0\times [0,1]\to X_2$ from $gf$ to $h$.
				\item A 3-simplex is a diagram
					\[
					\begin{tikzcd}
						& X_0 \ar[dl] \ar[dr] \ar[dd] \\
						X_1 \ar[rr] \ar[dr] & & X_3 \\
								    & X_2 \ar[ur]
					\end{tikzcd}
					\]
					with continuous maps $f_{ij}:X_i\to X_j$ for $i<j$, homotopies $T_{ijk}X_i\times [0,1]\to X_k$ from $f_{jk}\circ f_{ij}$ to $f_{ik}$, and $H:X_0\times [0,1]^2\to X_3$ (\defn{higher homotopy}) such that $H|_{\text{bdry}}$ is
					\[
					\begin{tikzcd}
						(0,0) \ar[r,"T_{123}f_{01}"] \ar[d,"f_{23}T_{012}"'] & (0,1) \ar[d,"T_{013}"] \\
						(1,0) \ar[r,"T_{023}"'] & (1,0)
					\end{tikzcd}
					\] 
			\end{itemize}
		\item The $\infty$-category of ordinary categories $\cat{Cat}_1$.
			\begin{itemize}
				\item Objects are ordinary categories.
				\item Morphisms are functors. 
				\item A 2-simplex is a (not necessarily commutative) diagram
					\[
					\begin{tikzcd}
						& X_1 \ar[dr,"g"] \\
						X_0 \ar[ur,"f"] \ar[rr,"h"] & & X_2
					\end{tikzcd}
					\] 
					\emph{and} a natural isomorphism $T:g\circ f \simto h$.
				\item A 3-simplex is a diagram
					\[
					\begin{tikzcd}
						& X_0 \ar[dl] \ar[dr] \ar[dd] \\
						X_1 \ar[rr] \ar[dr] & & X_3 \\
								    & X_2 \ar[ur]
					\end{tikzcd}
					\]
					where $f_{ij}$ are functors and $T_{ijk}$ are natural isomorphism such that
					\[
					\begin{tikzcd}
						\bullet \ar[r,"T_{123}f_{01}"] \ar[d,"f_{23}T_{012}"'] & \bullet \ar[d,"T_{013}"] \\
						\bullet \ar[r,"T_{023}"'] & \bullet
					\end{tikzcd}
					\] 
					commutes
			\end{itemize}
	\end{enumerate}
\end{example}
A source of $\infty$-categories are
\begin{itemize}
	\item ordinary categories with an equivalence relation on morphisms,
	\item ordinary categories with inverting some morphisms.
\end{itemize}
\marginpar{Lecture 2}
\begin{definition} \leavevmode
\begin{enumerate}[1.]
	\item Let $\C$ be an $\infty$-category and $f:X\to Y$ be a morphism in $\C$. $f$ is called an \defn{isomorphism} if there exists $g:Y\to X$ and two 2-simplices
		\[
		\begin{tikzcd}
			 & Y \ar[dr,"g"] & & & X \ar[dr,"f"] \\
			X \ar[ur,"f"] \ar[rr,"\id_X"] & & X & Y \ar[ur,"g"] \ar[rr,"\id_Y"] & & Y
		\end{tikzcd}
		\] 
	\item An $\infty$-category is called an \defn{$\infty$-groupoid} if \emph{every} morphism is an isomorphism.
\end{enumerate}
\end{definition}
\begin{theorem}[Joyal]
	An $\infty$-category is an $\infty$-groupoid if and only if it is a Kan complex.
\end{theorem}
\begin{proof}
	The forward direction is hard, the converse is an exercise.
\end{proof}
\begin{definition}\leavevmode
	\begin{enumerate}[1.]
		\setcounter{enumi}{2}
		\item Say $f,g:\C\to \cat{D}$ are functors (morphisms of simiplicial sets) of $\infty$-categories. A \defn{natural transformation} from $f$ to $g$ is a functor $T:\C\times \Delta^{1}\to \cat{D}$ such that $T|_{\C\times \{0\} }=f$ and $T|_{\C\times \{1\}}=g$.

			A special case: the identity natural transformation $\id_f:f\to F$ is the map
			\[
			\begin{tikzcd}
				\C\times \Delta^{1} \ar[r,"\text{proj}"] & \C \ar[r,"f"] & \cat{D}.
			\end{tikzcd}
			\] 

			$T:f\to g$ is a \defn{natural isomorphism} if there exists $T':g\to f$ and two maps $H:\C\times \Delta^2\to \cat{D},H':\C\times \Delta^2\to \cat{D}$ such that
			\[
			\begin{tikzcd}
				& g \ar[dr,"T'"] & & & f \ar[dr,"T"] \\
				f \ar[ur,"T"] \ar[rr,"\id_f"] & & f & g \ar[ur,"T'"] \ar[rr,"\id_g"] & & g
			\end{tikzcd}
			\] 
	\end{enumerate}
\end{definition}
In ordinary category theory a natural transformation assigns objects in $\C$ to morphisms in $\cat{D}$ and morphisms in $\C$ to commutative squares in $\cat{D}$. For $\infty$-categories a natural transformation takes objects to morphisms, morphisms to diagrams of shape $\Delta^{1}\times \Delta^{1}$ and generally an $n$-simplex to a diagram of shape $\Delta^{n}\times \Delta^{1}$.

\begin{theorem}[Pointwise criterion for natural isomorphism]
	Let $f,g:\C\to \cat{D}$ be functors of $\infty$-categories and $T:f\to g$ be a natural transformation. $T$ is a natural isomorphism if and only if for all objects $x$ in $\C$, $T(\{x\}\times \Delta^{1})$ is an isomorphism in $\cat{D}$.
\end{theorem}
This is a consequence of Joyal's theorem.
\begin{definition}
	Define $\cat{Cat}_\infty$ as follows:
	\begin{itemize}
		\item Objects are $\infty$-categories.
		\item Morphisms are functors.
		\item 2-simplices are diagrams
			\[
			\begin{tikzcd}
				& X_1 \ar[dr,"g"] \\
				X_0 \ar[ur,"f"] \ar[rr,"h"] & & X_2
			\end{tikzcd}
			\] 
			and a natural isomorphism $T:g\circ f\simto h$.
		\item 3-simplices and higher: copy the data of $\cat{Top}$ and replace $[0,1]^{n}$ by $(\Delta^{1})^{n}$.
	\end{itemize}
\end{definition}
This is similar to $\cat{Top}$ and $\cat{Cat}_1$.
\begin{definition}
	Define $\cat{Spc}$ same as above, except objects are $\infty$-groupoids.
\end{definition}
In literature: $\infty$-groupoids, Kan complexes, spaces and anima are synonyms.
\begin{definition}
	A functor $f:\C\to \cat{D}$ is called a \defn{categorical equivalence} if there exists $g:\cat{D}\to \C$ such that $f\circ g\isoto \id_{\cat{D}}$ and $g\circ f\isoto \id_{\C}$.
\end{definition}
\begin{theorem}[Fundamental Theorem of Category Theory]
	A functor $f:\C\to \cat{D}$ is a categorical equivalence if and only it it's essentially surjective and fully faithful.
\end{theorem}
Note that we haven't defined essentially surjective or fully faithful. Let's pre-warm up first before we define them.

\begin{lemma}
	Let $X$ be a Kan complex. $X$ is \defn{contractible} (i.e., categorically equivalent to $\Delta^{0}$) if and only if every lifting problem
	\[
	\begin{tikzcd}
		\partial \Delta^{n} \ar[r] \ar[d] & X \ar[d] \\
		\Delta^{n} \ar[ur,dashed] \ar[r] & \Delta^{0}
	\end{tikzcd}
	\] 
	admits a solution.
\end{lemma}
Now we're warm enough to warm up, so lets do that. Let $f:X\to Y$ be a map of Kan complexes. Suppose every lifting problem
\[
\begin{tikzcd}
	\partial \Delta^{n} \ar[r] \ar[r] & X \ar[d,"f"] \\
	\Delta^{n} \ar[ur,dashed] \ar[r] & Y
\end{tikzcd}
\] 
has a solution. Then $f$ \emph{is} a categorical equivalence (think: homotopy equivalence of topological spaces). But this condition is too strong for the converse. A simple counter-example is to take $X$ contractible.

\begin{definition}
	Let $f:\C\to \cat{D}$ be a functor of $\infty$-categories. Given
	\begin{equation}\label{eq:star}
	\begin{tikzcd}
		\partial \Delta^{n} \ar[r,"r"] \ar[d] & \C \ar[d,"f"] \\
		\Delta^{n} \ar[r,"s"] & \cat{D}
	\end{tikzcd}
	\end{equation}
	we say it admits a \defn{solution up to isomorphism} if
	\begin{enumerate}[(i)]
		\item there exists $u:\Delta^{n}\to \C$ such that
			\[
			\begin{tikzcd}
				\partial \Delta^{n} \ar[r,"r"] \ar[d] & \C \\
				\Delta^{n} \ar[ur,"u"]
			\end{tikzcd}
			\] 
		\item $f\circ u:\Delta^{n}\to \cat{D}$ is naturally isomorphic to $s:\Delta^{n}\to \cat{D}$ \emph{relative} (of relative homotopy) to $\partial \Delta^{n}$.
	\end{enumerate}
\end{definition}
\begin{definition}
	Let $f:\C\to \cat{D}$ be a functor of $\infty$-categories. 
	\begin{itemize}
		\item It's \defn{essentially surjective} if every diagram \eqref{eq:star} with $n=0$ admits a solution up to isomorphism. 
		\item It's \defn{full} if every diagram \eqref{eq:star} with $n=1$ admits a solution up to isomorphism.
		\item It's \defn{fully faithful} if every diagram \eqref{eq:star} with $n\ge 1$ admits a solution up to isomorphism.
	\end{itemize}
\end{definition}
So a functor of $\infty$-categories is fully faithfull and essentially surjective if all \eqref{eq:star} admit a solution up to isomorphism.
\begin{remark}
	These definitions of fully and full faithful are \emph{nonstandard}. 
\end{remark}
Now the Fundamental Theorem makes sense.
\begin{proof}[Proof idea]
	The forward direction is easy. Conversely, we factor through
	\[
	\begin{tikzcd}
		\C \ar[r] & \C^{\text{enhanced}} \ar[r] & \cat{D}
	\end{tikzcd}
	\] 
	where an $n$-simplex in $\C^{\text{enhanced}}$ is the data of
	\begin{itemize}
		\item $n$-simplex in $\C$,
		\item a diagram of shape $\Delta^{n}\times \Delta^{1}$ in $\cat{D}$ satisfying some conditions.
	\end{itemize}
	The inverses of the intermediate maps are easy to construct.
\end{proof}
What is missing so far is \emph{mapping spaces}. Given objects $X,Y$ in an $\infty$-category $\C$ we expect to find a space (Kan complex) $\map_\C(X,Y)$ such that the objects of $\map_\C(X,Y)$ are morphisms $X\to Y$ \emph{and} it should extend to a functor 
\[
	\map_\C : \C\op\times \C\to \cat{Spc}.
\] 
For 1-categories this is usually called $\Hom$ or $\Mor$. Lurie uses $\Hom$ for a non-functorial, but easier, version of $\map$.

Here is one non-functorial approach to mapping spaces. Let $\C^{\Delta^{1}}$ be the simplicial set such that $(\C^{\Delta^{1}})_k$ is the sest of maps $\Delta^{1}\times \Delta^{k}\to \C$. By restricting to $\{0\} \times \Delta^{k}$ and $\{1\} \times \Delta^{k}$ we get a map $\C^{\Delta^{1}}\to \C\times \C$. Define $\map_\C(X,Y)$ as the fiber product (of simplicial sets)
\[
\begin{tikzcd}
	\map_\C(X,Y) \ar[r] \ar[d] & \C^{\Delta^{1}} \ar[d] \\
	\Delta^{0} \ar[r,"{(x,y)}"] & \C\times \C.
\end{tikzcd}
\] 
\begin{theorem}
	Let $f:\C\to \cat{D}$ be a functor of $\infty$-categories. Then $f$ is fully faithful if and only if for all objects $X,Y$, the induced map
	\[
		\map_\C(X,Y)\to \map_{\cat{D}}(f(x),f(y))
	\] 
	is a categorical equivalence of Kan complexes.
\end{theorem}
This theorem is actually the usual definition in the literature.

Somewhere along the way:
\begin{theorem}[Whitehead's Theorem]
	A map $f:X\to Y$ of Kan complexes is a categorical equivalence if and only if 
	\[
	\pi_n(X,x_0) \to  \pi_n(Y,f(x_0)
	\] 
	are bijections for all $n$ and all $x_0$.
\end{theorem}
\marginpar{Lecture 3}
A monoidal category in ordinary category theory consists of:
\begin{itemize}
	\item A category $\C$.
	\item A functor $\otimes:\C\times \C\to \C$.
	\item An object $\1b\in \C$.
	\item 3 natural transformations: the associator, left and right unitors.
\end{itemize}
We ask them to satisfy 3 axioms:
\begin{itemize}
	\item Triangle axioms (they say $\1b\otimes x = x = x\otimes\1b$.
	\item Pentagon axiom (various ways to group 4 objects).
\end{itemize}
Mac Lane's Coherence Theorem tells us that every ``reasonable'' diagram made from the 3 natural transformations commutes.

Let's try to mimic this for $\infty$-categories. The naive approach is to start with:
\begin{itemize}
	\item an $\infty$-category $\C$;
	\item a functor $\otimes:\C\times \C\to \C$;
	\item an object $\1b:\Delta^{0}\to \C$;
	\item an associator
		\[
		\begin{tikzcd}
			\C\times \C\times \C \ar[r,"\id\times \otimes"] \ar[d,"\otimes\times \id"] \ar[dr] & \C \times \C \ar[d,"\otimes"] \\
			\C\times \C \ar[r,"\otimes"] & \C
		\end{tikzcd}
		\] 
		a diagram of shape $\Delta^{1}\times \Delta^{1}$ in $\cat{Cat}_\infty$;
	\item a left unitor
		\[
		\begin{tikzcd}
			\Delta^{0}\times \C \ar[rr,"\1b\times \id"] \ar[dr,"\sim"] & & \C\times \C \ar[dl,"\otimes"] \\
										   & \C
		\end{tikzcd}
		\] 
		a 2-simplex in $\cat{Cat}_\infty$ and similarly a right unitor;
\end{itemize}
and ask it to satisfy
\begin{itemize}
	\item the triangle identity
		\[
		\begin{tikzcd}
			\C\times \Delta^{0}\times \C \ar[d,"\id\times \1b\times \id"'] \ar[ddr] \ar[ddrr] \ar[drr] \\
			\C\times \C\times \C \ar[dr] \ar[rr] \ar[drr] & & \C\times \C \ar[d] \\
								      & \C\times \C \ar[r] & \C
		\end{tikzcd}
		\] 
		two 3-simplices attached along a face.
\end{itemize}
We model these diagrams with totally ordered finite sets:
\begin{itemize}
	\item Associator
		\[
		\begin{tikzcd}
			0123 & 013 \ar[l] \\
			023 \ar[u] & 03 \ar[l] \ar[u] \ar[lu]
		\end{tikzcd}
		\] 
	\item Left unitor
		\[
		\begin{tikzcd}
			\frac{1}{2}2 & & 012 \ar[ll, "{(01, 2)\mapsto (\frac{1}{2}, 2)}"] \\
				     & 02 \ar[lu, "\sim"] \ar[ur,hook]
		\end{tikzcd}
		\] 
	\item Triangle identity
		\[
		\begin{tikzcd}
			0 \frac{3}{2}3 \\
			0123 \ar[u] & & 013 \ar[ull,"\sim"] \ar[ll] \\
				    & 023 \ar[ul] \ar[uul,"\sim"] & 013 \ar[l] \ar[u] \ar[uull]
		\end{tikzcd}
		\] 
\end{itemize}
The data for the naive approach is modeled by
\[
	N\left( \left( \begin{array}{c}
		\text{nonempty, finite, totally ordered} \\
		\text{sets of size} \le 4
\end{array}\right) \op \right) \to \cat{Cat}_{\infty}
\] 
We are still missing the pentagon axiom and Mac Lane's Coherence Theorem.

\begin{definition}
	Let $\C$ be an $\infty$-category. A \defn{monoidal structure} on $\C$ is a functor $F:N(\bb{\Delta}\op)\to \cat{Cat}_\infty$ such that
	\begin{enumerate}[1)]
		\item $F\left( [1] \right) = \C$,
		\item for all $n $
			\[
				[n] \leftarrow \{0,1\} , \{1,2\} , \dots, \{n-1, n\} 
			\] 
			induces
			\[
				F([n]) \to \C\times \C\times \dots\times \C = \C^{n}
			\] 
			which we require to be an equivalence of categories.
	\end{enumerate}
\end{definition}
Note that $F([0]) \simto \Delta^{0}$. \emph{The idea} is that the pentagon axiom and all ``higher coherences'' are encoded in $\bb{\Delta}\op$.

The problem with this definition is unusable. Actually writing down a functor $N(\bb{\Delta}\op)\to \cat{Cat}_\infty$ is too complicated. What do we do? Lurie will rescue us. 

Let's warm up. Suppose you have a $\Lambda^2_0$ horn
\[
\begin{tikzcd}[row sep=tiny]
	& 1 \ar[dd,dashed,"\exists ?"] \\
	0 \ar[ur,"f"] \ar[dr,"h"] \\
	& 2
\end{tikzcd}
\] 
so we are asking: given
\[
\begin{tikzcd}
	\Lambda_0^{2} \ar[r] \ar[d,hook] & \C \\
	\Delta^2 \ar[ur,dashed]
\end{tikzcd}
\] 
when does a solution exist? It exists if $f$ has a right inverse.

\begin{definition}
	Let $p:\C\to \cat{D}$ be a functor of $\infty$-categories, $f:x\to y$ be a morphism in $\C$.  We say $f$ is \defn{$p$-cocartesian} if
	\[
	\begin{tikzcd}
		\{0,1\} = \Delta^{1} \ar[r,hook] \ar[rr,bend left=20, "f"] & \Lambda_0^{n} \ar[r] \ar[d, hook] & \C \ar[d,"p"] \\
									   & \Delta^{n} \ar[r] \ar[ur,dashed] & \cat{D}
	\end{tikzcd}
	\] 
	has a solution.
\end{definition}
\begin{definition}
	$p:\C\to \cat{D}$ is called a \defn{cocartesian fibration} if lifting problems
	\[
	\begin{tikzcd}
		\Lambda_j^{n} \ar[r,"r"] \ar[d,hook] & \C \ar[d,"p"] \\
		\Delta^{n} \ar[r] \ar[ur,dashed] & \cat{D}
	\end{tikzcd}
	\] 
	have a solution when
	\begin{enumerate}[(1)]
		\item $0<j<n$,
		\item $j=0$ and $n\ge 2$ \emph{if} $r$ sends $\{0,1\} $ to a $p$-cocartesian edge,
		\item $j=0,n=1$; in this case we also \emph{require} the solution $u:\Delta^{1}\to \C$ to be a $p$-cocartesian edge.
	\end{enumerate}
\end{definition}
\emph{The idea} is that a cocartesian fibration $p:\C\to \cat{D}$ should be thought of as a ``functorial family of $\infty$-categories indexed by $\cat{D}$''. More precisely:
\begin{itemize}
	\item For each object $x$ in $\cat{D}$ let $\C_x = \{x\}\times_{\cat{D}}\C$. This is an $\infty$-category.
	\item For each edge $\gamma:x\to y$ in $\cat{D}$ and each $n$-simplex $\sigma:\Delta^{n}\to \C_x$ we can construct a map that under
		\[
			\wt{\sigma}: \Delta^{n}\times \Delta^{1} \to \C
		\] 
		sends $\{j\} \times \Delta^{1}$ to a $p$-cocartesian edge.
	\item Moreover, we get a \emph{functor}
		\begin{align*}
			\C_x &\longrightarrow \C_y \\
			\sigma &\longmapsto \wt{\sigma}_{\Delta^{n}\times \{1\} }
		\end{align*}
		(this is slightly sloppy).
\end{itemize}
The definition is precisely set up so you can carry this out. Let's keep going:
\begin{itemize}
	\item A 2-simplex in $\cat{D}$ gives a 2-simplex in $\cat{Cat}_\infty$ 
		\[
		\begin{tikzcd}[row sep=tiny]
			& C_y \ar[dd] \\
			C_x \ar[ur] \ar[dr] \\
			& C_z
		\end{tikzcd}
		\] 
\end{itemize}

\end{document}
