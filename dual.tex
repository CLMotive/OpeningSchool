\documentclass[oneside]{amsart}

\usepackage[all]{xy}
\usepackage{tikz-cd}
\usepackage[T1]{fontenc}
\usepackage{xstring}
\usepackage{xparse}
\usepackage{xr-hyper}
\usepackage{xcolor}
\definecolor{brightmaroon}{rgb}{0.76, 0.13, 0.28}
\usepackage[linktocpage=true,colorlinks=true,hyperindex,citecolor=blue,linkcolor=brightmaroon]{hyperref}
\usepackage[nameinlink]{cleveref}
\usepackage[left=1.25in,right=1.25in,top=0.75in,bottom=0.75in]{geometry}
%\usepackage[charter,greekfamily=didot]{mathdesign}
%\usepackage{Baskervaldx}
\usepackage{amssymb}
\usepackage{stmaryrd}
\usepackage{mathrsfs}
\usepackage{mathpazo}
\linespread{1.05}

\usepackage[nobottomtitles]{titlesec}
\usepackage{marginnote}
\usepackage{enumerate}
\usepackage{longtable}
\usepackage{aurical}
\usepackage{microtype}

\newtheoremstyle{ega-env-style}%
{}{}{\rmfamily}{}{\bfseries}{.}{ }{\thmnote{(#3)}}%

\newtheoremstyle{ega-thm-env-style}%
{}{}{\itshape}{}{\bfseries}{. --- }{ }{\thmname{#1}\thmnumber{ #2}\thmnote{ (#3)}}%

\newtheoremstyle{ega-defn-env-style}%
{}{}{\rmfamily}{}{\bfseries}{. --- }{ }{\thmname{#1}\thmnumber{ #2}\thmnote{ (#3)}}%

\theoremstyle{ega-env-style}
\newtheorem*{env}{---}

\theoremstyle{ega-thm-env-style}
\newtheorem{theorem}{Theorem}[subsection]
\newtheorem{proposition}{Proposition}[subsection]
\newtheorem{lemma}{Lemma}[subsection]
\newtheorem{corollary}{Corollary}[subsection]
\newtheorem{stheorem}{Theorem}[section]
\newtheorem{slemma}[stheorem]{Lemma}
\newtheorem{skey}[stheorem]{Key Formula}
\newtheorem{conjecture}{Conjecture}[subsection]

\theoremstyle{ega-defn-env-style}
\newtheorem*{definition}{Definition}
\newtheorem{example}{Example}[subsection]
\newtheorem*{examples}{Examples}
\newtheorem*{remark}{Remark}
\newtheorem*{remarks}{Remarks}
\newtheorem*{notation}{Notation}
\newtheorem*{exercise}{Exercise}
\newtheorem*{properties}{Properties}
\newtheorem*{consequences}{Consequences}
\newtheorem*{note}{Note}
\newtheorem*{fact}{Fact}

\makeatletter
\def\l@subsection{\@tocline{2}{0pt}{2.5pc}{2.2pc}{}}
\def
\section{\@startsection{section}{1}%
	\z@{.7\linespacing\@plus\linespacing}{.5\linespacing}%
{\normalfont\bfseries\Large\scshape\centering}}
\renewcommand{\@seccntformat}[1]{%
	\ifnum\pdfstrcmp{#1}{section}=0\textsection\fi%
\csname the#1\endcsname.~}
\makeatother

\def\mathcal{\mathscr}
%% Fonts

\def\sh{\mathcal}                   % sheaf font
\def\bb{\mathbf}                    % bold font
\def\cat{\mathsf}                   % category font

%% Font Letters

\def\CL{\mathcal{L}}
\def\OO{\mathcal{O}}
\def\CX{\mathcal{X}}

%% Cohomology

\def\CH{\mathrm{H}}                 % cohomology H
\def\CHH{\check{\HH}}               % Čech cohomology H
\def\RD{\mathrm{R}}                 % right derived R
\def\LD{\mathrm{L}}                 % left derived L
\def\dual#1{{#1}^\vee}              % dual
\def\Tor{\operatorname{Tor}}        % Tor
\def\Ext{\operatorname{Ext}}        % Ext
\def\HdR{\mathrm{H}_{\mathrm{dR}}}  % de Rham cohomology
\def\Zc{\underline{\mathbb{Z}}}     % constant sheaf with integer coeffs

%% Categories

\def\A{\cat{A}}                     % category A, usually abelian
\def\C{\cat{C}}                     % category C
\def\op{^\cat{op}}                  % opposite category
\def\Set{\cat{Sets}}                % category of sets
\def\Grp{\cat{Gps}}                 % category of groups
\def\Alg{\cat{Alg}}                 % category of algebras
\def\QCoh{\cat{QCoh}}               % category of quasicoherent sheaves
\def\supertilde{{\,\widetilde{\,}\,}}   % use \supertilde instead of ^\sim
\def\Hom{\operatorname{Hom}}        % morphisms
\def\End{\operatorname{End}}        % endomorphisms
\def\Aut{\operatorname{Aut}}        % automorphisms
\def\ker{\operatorname{ker}}        % kernel
\def\img{\operatorname{im}}         % image
\def\coker{\operatorname{coker}}    % cokernel
\def\pr{\operatorname{pr}}          % projection
\def\EssIm{\operatorname{EssIm}}    % essential image
\DeclareMathOperator*{\colim}{colim}   % colimit


%% Schemes

\def\Proj{\operatorname{Proj}}      % Proj
\def\Supp{\operatorname{Supp}}      % support
\def\Spec{\operatorname{Spec}}      % Spec
\def\Spf{\operatorname{Spf}}        % formal Spec
\def\Aff{\mathbb A}                 % affine space
\def\P{\mathbb{P}}                  % projective space
\def\Pic{\operatorname{Pic}}        % Picard group

%% Standard Operators

\def\codim{\operatorname{codim}}    % codimension
\def\id{\operatorname{id}}          % identity

%% Arrows
\renewcommand{\to}{\mathchoice{\longrightarrow}{\rightarrow}{\rightarrow}{\rightarrow}}
\newcommand{\from}{\mathchoice{\longleftarrow}{\leftarrow}{\leftarrow}{\leftarrow}}
\let\mapstoo\mapsto
\renewcommand{\mapsto}{\mathchoice{\longmapsto}{\mapstoo}{\mapstoo}{\mapstoo}}
\def\isoto{\simeq}                  % isomorphism
\def\simto{\xrightarrow{\sim}}      % isomorphism arrow
\def\surjto{\twoheadrightarrow}     % surjetion
\def\injto{\hookrightarrow}         % injection

%% Under/Over Accents

\newcommand{\wh}[1]{\widehat{#1}}   % hat
\newcommand{\wt}[1]{\widetilde{#1}}    % tilde
\def\ul{\underline}                % underline

%% Spaces

\def\Z{\mathbb Z}                  % integers
\def\Q{\mathbb Q}                  % rationals
\def\N{\mathbb{N}}                 % naturals

%% Groups

\def\GL{\bb{GL}}                   % general linear group
\def\SL{\bb{SL}}                   % special linear group
\def\det{\operatorname{det}}       % determinant

%% Sub/Superscripts

\def\et{^\text{\'et}}              % \'etale
\def\an{^{\text{an}}}              % analytic

%% Misc

\newcommand{\defn}[1]{\textbf{#1}}  % definition highlighting
\def\defeq{:=}                     % definition equation
\def\eps{\varepsilon}              % correct epsilon
\newcommand{\gen}[1]{\left\langle\!\left\langle #1 \right\rangle\!\right\rangle}



\def\shHom{\sh{H}\textup{\kern-2.2pt{\Fontauri\slshape om}}}   % sheaf Hom
\def\shProj{\sh{P}\textup{\kern-2.2pt{\Fontauri\slshape roj}}} % sheaf Proj
\def\shExt{\sh{E}\textup{\kern-2.2pt{\Fontauri\slshape xt}}}   % sheaf Ext
\def\shGr{\sh{G}\textup{\kern-2.2pt{\Fontauri\slshape r}}}     % sheaf Gr
\def\shDer{\sh{D}\,\textup{\kern-2.2pt{\Fontauri\slshape er}}} % sheaf Der
\def\shDiff{\sh{D}\,\textup{\kern-2.2pt{\Fontauri\slshape if{}f}}\,} % sheaf Diff
\def\shHomcont{\sh{H}\textup{\kern-2.2pt{\Fontauri\slshape om.\,cont}}}   % sheaf Hom.cont
\def\shAut{\sh{A}\textup{\kern-2.2pt{\Fontauri\slshape ut}}}   % sheaf Aut
\def\shSym{\sh{S}\textup{\kern-2.2pt{\Fontauri\slshape ym}}}   % sheaf Sym

\makeatletter
\newcommand{\cbigoplus}{\DOTSB\cbigoplus@\slimits@}
\newcommand{\cbigoplus@}{\mathop{\widehat{\bigoplus}}}
\makeatother

\def\coev{\operatorname{coev}}
\def\ev{\operatorname{ev}}
\def\PrL{\cat{Pr}^L}

\title{Dualizability}
\author{Lecturer: Jakob Scholbach,\quad Typesetter: Micha{\l} Mruga{\l}a}

\begin{document}
\maketitle

\section{Dualizable objects}
\begin{definition}
	Let $(\C, \otimes, \b1)$ be a symmetric monoidal category. An object $c\in \C$ is
	\defn{dualizable} if there exists $c^{\vee}\in C$ and maps
	\begin{align*}
		\coev &: \b1 \to c\otimes c^{\vee} \\
		\ev &: c^{\vee}\otimes c \to \b1
	\end{align*}
	such that
	\begin{align*}
		\left[ c \xrightarrow{\coev \otimes \id} c \otimes c^{\vee}\otimes c
		\xrightarrow{\id\otimes \ev} c \right] &= \id_{c} \\
		\left[ c^{\vee}\to c^{\vee}\otimes c\otimes c^{\vee}\to c^{\vee} \right] &= \id_{c^{\vee}}
		.
	\end{align*}
\end{definition}
\begin{example}
	Let $C=\cat{Vect}_k\ni V$. $V$ is dualizable if and only if $\dim V<\infty$.

	If $V$ is finite-dimensional define $V^{\vee} = \Hom(V,k)$, pick a basis $v_1,\dots,v_n$
	and a dual basis $v^{1},\dots,v^{n}\in V^{\vee}$. We can define
	\begin{align*}
		\coev: k &\longrightarrow V\otimes V^{\vee} \\
		1 &\longmapsto \sum v_i\otimes v^{i}
	\end{align*}
	and $\ev$ to be the usual evaluation map.

	For the inverse direction we get that $\coev(1)$ is a finite sum. One shows that the
	$v_i$ that show up form a basis.
\end{example}
\begin{example}
	In $(\cat{Set},\times ), (\cat{Top}, \times )$ and any $(\C,\times )$ the only
	dualizable object is $\{*\} $.
\end{example}
Recall that $\iHom(c,c')$ satisfies
\[
	\Hom_C(t,\iHom(c,c')) = \Hom_C(t\otimes c,c')
.\]
\begin{corollary}
	An object $c\in C$ is dualizable if and only if
\begin{enumerate}[a)]
	\item $\iHom(c,\b1)$ and  $\iHom(c,c)$ exist,
	\item $c\otimes \iHom(c,\b1)\to \iHom(c,c)$ is an iso.
\end{enumerate}
(Hence $c^{\vee} = \iHom(c,\b1)$.
\end{corollary}
\begin{example}
	Take $\cat{Mod}_R$ for a commutative ring $R$. Then $M\in \cat{Mod}_R$ is dualizable if
	and only if $M$ is projective.

	Dualizable objects are closed under retracts. If $M$ is a finite projective module, it
	is a retract of a finite free module, so it is dualizable.

	Conversely, if $M$ is dualizable there exists
	\[
		\begin{tikzcd}[row sep=tiny]
			M \ar[r] & R^{?} \ar[r] & M \\
			m \ar[r,mapsto] & (f_i(m))_i \\
			& (r_i) \ar[r, mapsto] & \sum r_im_i
		\end{tikzcd}
	\]
	We define
	\begin{align*}
		R &\longrightarrow M\otimes_R \Hom(M,R) \\
		1 &\longmapsto \sum m_i\otimes f_i
		.
	\end{align*}
\end{example}
\begin{example}
	Consider $D(R),\otimes_R^{\mathbb{L}}$. An object is dualizable if and only if it is a
	perfect complex, i.e. quasi-isomorphic to a finite complex of finite projective modules.
\end{example}
\begin{example}
	Let $X$ be a qcqs scheme, then $\CF\in D(\cat{QCoh}(X))$ is dualizable if and only if
	$\CF$ is perfect, i.e. $\CF|_{\Spec R}$ is as above.
\end{example}
\section{Dualizability as a finiteness condition}
\begin{definition}
	If $\C$ is a symmetric monoidal $\infty$-category, $c\in \C$ is dualizable if and only
	if $c\in H_0(\C)$ is dualizable.
\end{definition}
\begin{lemma}
	Suppose $\C$ is a symmetric monoidal $\infty$-category which has filtered colimits,
	which are preserved under $\otimes$. If $\b1\in \C$ is compact (i.e. $\Map_{\C}(\b1,-)$
	preserves filtered colimits), then any dualizable object is compact.
\end{lemma}
\begin{proof}
	$\Map_{\C}(c,-)=\Map_{\C}(\b1, c^{\vee}\otimes -)$.
\end{proof}
\begin{lemma}
	Suppose $\C$ is presentable and colimits are preserved under $\otimes$ (presentably
	symmetric monoidal). Then $c\in \C$ is dualizable if and only if $c\otimes -$ preserves limits.
\end{lemma}
\begin{proof}
	First suppose $\varphi=c\otimes -$ preserves limits. By the adjoint functor theorem
	$\varphi$ admits a left adjoint $\varphi^{L}$, then $\varphi^{L}(\b1)$ is a dual for $c$.
\end{proof}
\begin{lemma}
	Let $X$ be a topological space, $\CF\in D(\Sh(X,\cat{Ab}))$ is dualizable if and only if
	locally on $X$, $\CF$ is constant and associated to a perfect complex of abelian groups.
\end{lemma}
\begin{proof}
	Given an open subset $U$ of $X$ write $u:U\injto X$ for the open embedding. We claim that
	\[
		\colim_{x\in U}\Hom(u^{*}\CF,u^{*}\CG)\to \Hom(\CF_x, \CG_x)
	\]
	is an isomorphism. (For a proof look at Cisinski, D\'eglise \'Etale motives.)
\end{proof}
Let $X$ be a smooth, affine variety over $k$ we can associate to it the de Rham complex
$\Omega_{X /k}^{*}$ and de Rham cohomology $H_{\text{dR}}^{n}(X)=H^{n}(X,\Omega_{X
/k}^{*})$. If $X\an$ is compact, then $\Omega_{X\an}^{*}$ are holomorphic differential
forms and we define $H_{\text{dR}\an}^{n}(X) = H^{n}(X\an,\Omega^{*}_{X\an})$.
\begin{theorem}[Grothendieck, C-D ``Weil\dots'']
	Fix $k\subset \bb{C}$, then there is an isomorphism
	\[
		H_{\text{dR}}^{n}(X)\otimes_k \bb{C} \simto H_{\text{dR}\an}^{n}(X)
	.\] 	
\end{theorem}
\begin{proof}
	We have a commutative diagram
	\[
		\begin{tikzcd}
			\cat{Sm}_k\op \ar[r, bend left=20, "X\mapsto \Omega_{X /k}\otimes\bb{C}", name=F]
			\ar[r, bend right=20, "X\mapsto \Omega_{X\an}\an"', name=G]
			\ar[d,"X\mapsto M(X)"']
			& D(\cat{Vec}_{\bb{C}})
			\\
			\DAet(k)\op\text{ or }\SH(k)\op
		\end{tikzcd}
	\]
	(there is a natural transformation from top arrow to bottom, add this). The functors at
	the top are symmetric monoidal.

	\begin{lemma}
		Suppose we have
		\begin{equation}
			\begin{tikzcd}
				(\C,\otimes) \ar[r,shift left,"F"] \ar[r,shift right,"G"'] & (\cat{D},\otimes)
			\end{tikzcd}
		\end{equation}
		where $F,G$ are monoidal funcgtor and $\alpha:F\to G$ compatible with $\otimes$. There
		exists $c\in \C$ dualizable. Then
		\[
			\alpha(c): F(c) \simto G(c)
		\]
		is an isomorphism.
	\end{lemma}
	\textbf{Fact:} $\DAet(k)$ or $\SH(k)$ is generated by $M(X)(?)$ for $X /k$ smooth and
	proper. (6 functor formalism and resolution of singularities)

	$M(X)$ is dualizable for $X$ smooth and proper (by the 6 functor formalism).

	Look at Robalo's thesis to get a $F,G$ factoring through $\SH(k)\op$ as symmetric
	monoidal functors. (??)
\end{proof}
Recall (or wait until Friday) $(\PrL,\otimes)$ the category of presentable
$\infty$-categories and a colimit preserving functors
\begin{itemize}
	\item $P(C_0)\otimes P(C_1) = P(C_0\times C_1)$
	\item If $X,Y$ are (qcqs) schemes over $k$
		\[
			D(\cat{QCoh}(X)) \otimes_{D(k)}D(\cat{QCoh}(Y)) = D(\cat{QCoh}(X\times_k Y)).
		\]
\end{itemize}
\begin{example}
	In $\PrL$, $P(C_0)$ is dualizable with dual $P(C_0\op)$.
\end{example}

Now consider $\PrL_{\omega}$ the category of presentable, compactly generated categories
and functors that preserve compact objects. Define $\PrL_{\omega,k} =
\cat{Mod}_{D(k)}\PrL_\omega$. Any $\C=\cat{Ind}(C_0)$ is dual in $\PrL$ with dual
$\C^{\vee}=\cat{Ind}(C_0\op)$.

\begin{theorem}[Kontsevich]
	Let $X /k$ be an algebraic variety. Define $\C = D(\cat{QCoh}(X))\in \PrL_{\omega,k}$.
\begin{enumerate}[1)]
	\item $X$ is smooth if and only if (in $\PrL_k$)
		\begin{align*}
			\coev: D(k) &\longrightarrow \C\otimes\C^{\vee} = D(\cat{QCoh}(X\times X)) \\
			k &\longmapsto \Delta_*\OO_X
		\end{align*}
		preserves compact objects.
	\item $X$ is proper if and only if $p_*\Delta^{*}=\ev:D(\cat{QCoh}(X\times X))\to D(k)$
		preserves compact objects.
\end{enumerate}
Hence $X$ is smooth and proper if and only if $D(\cat{QCoh}(X))$ is dual in $\PrL_{\omega,k}$.
\end{theorem}
(Kadyrev, Prikodko proved Atiyah-Bott which implies Borel-Weil-Bott)

\end{document}
